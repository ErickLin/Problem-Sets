\documentclass[a4paper,12pt]{article}

\usepackage{amsfonts, amsmath, fancyhdr}
\usepackage[margin=3.5cm]{geometry}
\allowdisplaybreaks
\pagestyle{fancy}
\rhead{Erick Lin}

\begin{document}

\section*{MATH 4320 - HW1 Solutions}

\subsection*{1.3}
\begin{enumerate}
    \item[1.]
        \begin{enumerate}
            \item
                $\frac{(1 + 2i)(3 + 4i)}{(3 - 4i)(3 + 4i)} + \frac{(2 - i)(-5i)}{5i(-5i)} = \frac{3 + 4i + 6i + 8i^2}{9 - 16i^2} + \frac{-10i + 5i^2}{-25i^2} = \frac{-5 + 10i}{25} + \frac{-5 - 10i}{25} = -\frac{2}{5}$

            \item
                $\frac{5i(1 + i)(2 + i)(3 + i)}{(1 - i)(2 - i)(3 - i)(1 + i)(2 + i)(3 + i)} = \frac{(5i - 5)(5i + 5)}{(1 - i^2)(4 - i^2)(9 - i^2)} = \frac{25(i^2 - 1)}{2(5)(10)} = -\frac{1}{2}$

            \item
                $[(1 - i)(1 - i)]^2 = (1 - 2i + i^2)^2 = (-2i)^2 = 4i^2 = -4$
        \end{enumerate}

    \item[4.]
        We know that if a product of two complex numbers is zero, then so is at least one of the factors. Then $z_1 z_2 z_3$ can be grouped as $(z_1 z_2) z_3$, which is a product of two complex numbers; thus, we have that $z_3 = 0$ or $z_1 z_2 = 0$. If the latter is true, then we can apply the rule again and deduce that $z_1 = 0$ or $z_2 = 0$. We have the result that at least one of $z_1$, $z_2$, or $z_3$ is zero.
\end{enumerate}

\subsection*{1.5}
\begin{enumerate}
    \item[3.]
        From the Pythagorean theorem and the triangle inequality,
        \begin{align}
            \Re(z_1 + z_2) \leq |z_1 + z_2| \leq |z_1| + |z_2| \label{eq:ineq1}
        \end{align}
        and from the triangle inequality,
        \begin{align}
            |z_3 + z_4| &\geq ||z_3| - |z_4|| \\
            \frac{1}{|z_3 + z_4|} &\leq \frac{1}{||z_3| - |z_4||}. \label{eq:ineq2}
        \end{align}
        Multiplying (\ref{eq:ineq1}) and (\ref{eq:ineq2}) together gives
        \begin{align*}
            \frac{\Re(z_1 + z_2)}{|z_3 + z_4|} \leq \frac{|z_1| + |z_2|}{||z_3| - |z_4||}.
        \end{align*}

    \item[8.]
        Using the definition of the modulus of complex numbers,
        \begin{align*}
            |(x_1 + iy_1)(x_2 + iy_2)| &= |(x_1 x_2 - y_1 y_2) + i(x_1 y_2 + x_2 y_1)| \\
            &= \sqrt{(x_1 x_2 - y_1 y_2)^2 + (x_1 y_2 + x_2 y_1)^2} \\
            &= \sqrt{x_1^2 x_2^2 - 2 x_1 x_2 y_1 y_2 + y_1^2 y_2^2 + x_1^2 y_2^2 + 2 x_1 x_2 y_1 y_2 + x_2^2 y_1^2} \\
            &= \sqrt{(x_1^2 + y_1^2)(x_2^2 + y_2^2)}.
        \end{align*}
        Substituting $z_1 = x_1 + iy_1$ and $z_2 = x_2 + iy_2$, we have
        \begin{align*}
            |(x_1 + iy_1)(x_2 + iy_2)| = |z_1 z_2|
        \end{align*}
        and
        \begin{align*}
            \sqrt{(x_1^2 + y_1^2)(x_2^2 + y_2^2)} = \sqrt{x_1^2 + y_1^2} \sqrt{x_2^2 + y_2^2} = |z_1| |z_2|,
        \end{align*}
        which yields the identity
        \begin{align*}
            |z_1 z_2| = |z_1| |z_2|.
        \end{align*}
\end{enumerate}

\subsection*{1.6}
\begin{enumerate}
    \item[13.]
        Using identities of complex conjugates, we have that
        \begin{gather*}
            |z - z_0|^2 = R^2 \\
            (z - z_0)(\overline{z} - \overline{z_0}) = R^2 \\
            z \overline{z} - z \overline{z_0} - \overline{z \overline{z_0}} + z_0 \overline{z_0} = R^2 \\
            |z|^2 - 2\Re(z \overline{z_0}) + |z_0|^2 = R^2.
        \end{gather*}

    \item[15.]
        \begin{enumerate}
            \item
                \begin{align*}
                    |z_1 + z_2|^2 &= (z_1 + z_2)(\overline{z_1 + z_2}) \\
                    &= (z_1 + z_2)(\overline{z_1} + \overline{z_2}) \\
                    &= z_1 \overline{z_1} + z_1 \overline{z_2} + \overline{z_1} z_2 + z_2 \overline{z_2} \\
                    &= z_1 \overline{z_1} + (z_1 \overline{z_2} + \overline{z_1 \overline{z_2}}) + z_2 \overline{z_2}
                \end{align*}

            \item
                Using identities of complex conjugates,
                \begin{align*}
                    z_1 \overline{z_2} + \overline{z_1 \overline{z_2}} &= 2\Re(z_1 \overline{z_2})
                \end{align*}
                and
                \begin{align*}
                    2\Re(z_1 \overline{z_2}) &\leq 2|z_1 \overline{z_2}| \\
                    &= 2|z_1||\overline{z_2}| \\
                    &= 2|z_1||z_2|.
                \end{align*}

            \item
                Combining the results from parts (a) and (b),
                \begin{align*}
                    |z_1 + z_2|^2 &= z_1 \overline{z_1} + 2\Re(z_1 \overline{z_2}) + z_2 \overline{z_2} \\
                    &\leq |z_1| + 2|z_1||z_2| + |z_2| \\
                    &= (|z_1| + |z_2|)^2.
                \end{align*}
        \end{enumerate}
\end{enumerate}

\subsection*{1.9}
\begin{enumerate}
    \item[1.]
        \begin{enumerate}
            \item
                Simplifying,
                \begin{align*}
                    z &= \frac{-2(1 - \sqrt{3}i)}{(1 + \sqrt{3}i)(1 - \sqrt{3}i)} \\
                    &= \frac{-2 + 2\sqrt{3}i}{4} \\
                    &= -\frac{1}{2} + \frac{\sqrt{3}}{2}i \\
                    &= \cos\frac{2\pi}{3} + i\sin\frac{2\pi}{3}
                \end{align*}
                and so the principal argument is
                \begin{align*}
                    \text{Arg}z = \frac{2\pi}{3}.
                \end{align*}

            \item
                Simplifying,
                \begin{align*}
                    z &= \left( 2e^{i(-\pi/6 + 2k\pi)} \right)^6 \\
                    &= 2^6 e^{i(-\pi + 12k\pi)} \\
                    &= 2^6(\sin\pi + i\cos\pi)
                \end{align*}
                and so the principal argument is
                \begin{align*}
                    \text{Arg}z = \pi.
                \end{align*}
        \end{enumerate}

    \item[2.]
        \begin{enumerate}
            \item
                $|e^{i\theta}| = |\cos\theta + i\sin\theta| = \sqrt{\cos^2\theta + \sin^2\theta} = 1$

            \item
                $\overline{e^{i\theta}} = \overline{\cos\theta + i\sin\theta} = \overline{\cos\theta} + \overline{i\sin\theta} = \cos\theta - i\sin\theta = e^{-i\theta}$
        \end{enumerate}

    \item[9.]
        Since
        \begin{align*}
            1 + z + z^2 + \cdots + z^n &= \frac{(1 + z + z^2 + \cdots + z^n)(1 - z)}{1 - z} \\
            &= \frac{1 - z + z - z^2 + z^2 - z^3 + \cdots + z^n - z^{n + 1}}{1 - z} \\
            &= \frac{1 - z^{n + 1}}{1 - z}, \quad z \neq 1,
        \end{align*}
        if we let $z = e^{i\theta}$, we have
        \begin{align*}
            1 + e^{i\theta} + e^{i2\theta} + \cdots + e^{in\theta} &= \frac{1 - e^{i(n + 1)\theta}}{1 - e^{i\theta}} \cdot \frac{e^{-i\theta/2}}{e^{-i\theta/2}} \\
            &= \frac{e^{-i\theta/2} - e^{i(2n + 1)\theta/2}}{e^{-i\theta/2} - e^{i\theta/2}} \\
            &= \frac{\cos\frac{\theta}{2} - i\sin\frac{\theta}{2} - \cos\frac{(2n + 1)\theta}{2} - i\sin\frac{(2n + 1)\theta}{2}}{-2i\sin\frac{\theta}{2}} \cdot \frac{i}{i} \\
            &= \frac{\left[ \sin\frac{\theta}{2} + \sin\frac{(2n + 1)\theta}{2} \right] + i \left[ \cos\frac{\theta}{2} - \cos\frac{(2n + 1)\theta}{2} \right]}{2 \sin\frac{\theta}{2}}.
        \end{align*}
        Taking the real part of both sides yields Lagrange's trigonometric identity,
        \begin{align*}
            1 + \cos\theta + \cos2\theta + \cdots + \cos n\theta = \frac{1}{2} + \frac{\sin\frac{(2n + 1)\theta}{2}}{2\sin\frac{\theta}{2}}.
        \end{align*}

\end{enumerate}

\subsection*{1.11}
\begin{enumerate}
    \item[1.]
        \begin{enumerate}
            \item
                Since
                \begin{align*}
                    2i = 2e^{i(\pi/2 + 2k\pi)}, \quad k \in \mathbb{Z},
                \end{align*}
                the square roots are given by
                \begin{align*}
                    c_k = \sqrt{2} e^{i(\pi/4 + k\pi)}, \quad k \in \{ 0, 1 \},
                \end{align*}
                which become
                \begin{gather*}
                    c_0 = \sqrt{2} \left( \cos\frac{\pi}{4} + i\sin\frac{\pi}{4} \right) = \sqrt{2} \left( \frac{\sqrt{2}}{2} + \frac{\sqrt{2}}{2}i \right) = 1 + i \\
                    c_1 = \sqrt{2} \left( \cos\frac{5\pi}{4} + i\sin\frac{5\pi}{4} \right) = \sqrt{2} \left( -\frac{\sqrt{2}}{2} - \frac{\sqrt{2}}{2}i \right) = -1 - i.
                \end{gather*}

            \item
                Since
                \begin{align*}
                    1 - \sqrt{3} i = 2e^{i(-\pi/3 + 2k\pi)}, \quad k \in \mathbb{Z},
                \end{align*}
                the square roots are given by
                \begin{align*}
                    c_k = \sqrt{2} e^{i(-\pi/6 + k\pi)}, \quad k \in \{ 0, 1 \},
                \end{align*}
                which become
                \begin{gather*}
                    c_0 = \sqrt{2} \left[ \cos \left( -\frac{\pi}{6} \right) + i\sin \left( -\frac{\pi}{6} \right) \right] = \sqrt{2} \left( \frac{\sqrt{3}}{2} - \frac{1}{2}i \right) \\
                    c_1 = \sqrt{2} \left( \cos\frac{5\pi}{6} + i\sin\frac{5\pi}{6} \right) = \sqrt{2} \left( -\frac{\sqrt{3}}{2} + \frac{1}{2}i \right).
                \end{gather*}
        \end{enumerate}

    \item[6.]
        The zeros of $z^4 + 4$ occur where
        \begin{align*}
            z^4 = -4 = 4e^{i(\pi + 2k\pi)},
        \end{align*}
        or at
        \begin{align*}
            z_k = \sqrt{2} e^{i(\pi/4 + k\pi/2)}, \quad k \in \{ 0, 1, 2, 3 \},
        \end{align*}
        which become
        \begin{align*}
            z_0 &= \sqrt{2} e^{i \pi/4} = 1 + i \\
            z_1 &= \sqrt{2} e^{i 3\pi/4} = -1 + i \\
            z_2 &= \sqrt{2} e^{i 5\pi/4} = -1 - i \\
            z_3 &= \sqrt{2} e^{i 7\pi/4} = 1 - i.
        \end{align*}
        This means we can write the polynomial as
        \begin{align*}
            z^4 + 4 &= (z - z_0)(z - z_3)(z - z_1)(z - z_2) \\
            &= (z - 1 - i)(z - 1 + i)(z + 1 - i)(z + 1 + i) \\
            &= (z^2 - 2z + 2)(z^2 + 2z + 2).
        \end{align*}

    \item[7.]
        If $c^n = 1$ with $c \neq 1$, then from Section 1.9, Exercise 9,
        \begin{align*}
            1 + c + c^2 + \cdots + c^{n - 1} = \frac{1 - c^n}{1 - c} = \frac{1 - 1}{1 - c} = 0.
        \end{align*}
\end{enumerate}

\subsection*{1.12}
\begin{enumerate}
    \item[1.]
\end{enumerate}
\end{document}
