\documentclass[a4paper,12pt]{article}

\usepackage{amsfonts, amsmath, fancyhdr}
\usepackage[margin=3.5cm]{geometry}
\allowdisplaybreaks
\pagestyle{fancy}
\rhead{Erick Lin}

\begin{document}

\section*{MATH 4107 - HW3 Solutions}

\subsection*{2.5}
\begin{enumerate}
    \item[1.]
        \boldmath
        \textbf{Let $\varphi \colon G \to G'$ be a surjective homomorphism. Prove that if $G$ is cyclic, then $G'$ is cyclic, and if $G$ is abelian, then $G'$ is abelian.} \par
        \unboldmath
        If $G$ is cyclic, then $G = \langle a \rangle$ for some $a \in G$, and for all $x \in G$, $x = a^n$ for some $n \in \mathbb{Z}$, so $\varphi(x) = \varphi(a^n) = \varphi(a)^n$ because $\varphi$ is a homomorphism. This means that $\langle \varphi(a) \rangle \subset G'$. Furthermore, $\varphi$ is surjective, so for all $x' \in G'$, $x' = \varphi(x)$ for some $x \in G$; if $x = a^n$, then $x' = \varphi(a^n) = \varphi(a)^n$. In conclusion, $\langle \varphi(a) \rangle = G'$ and $G'$ is cyclic. \par
        Because $\varphi$ is surjective, for all $x', y' \in G'$, $x' = \varphi(x)$ and $y' = \varphi(y)$ for some $x, y \in G$. Since $\varphi$ is a homomorphism and $G$ is abelian, we have then that $x'y' = \varphi(x) \varphi(y) = \varphi(xy) = \varphi(yx) = \varphi(y) \varphi(x) = y'x'$ and $G'$ is abelian.

    \item[2.]
        \boldmath
        \textbf{Prove that the intersection $K \cap H$ of subgroups of a group $G$ is a subgroup of $H$, and that if $K$ is a normal subgroup of $G$, then $K \cap H$ is a normal subgroup of $H$.} \par
        \unboldmath
        Because a group $G$ and its subgroups share the same identity, $K$ and $H$ have the same identity as one another, so an identity exists in $K \cap H$. Also, since $K$ and $H$ are both groups, the inverse of any element in both $K$ and $H$ is also contained in both $K$ and $H$, and the product of any two elements contained in both $K$ and $H$ is also contained in both $K$ and $H$. \par
        For every $a \in K \cap H$ and every $h \in H \subset G$, $hah^{-1} \in K$ because $K$ is a normal subgroup of $G$. Furthermore, since $H$ is a group, $h^{-1} \in H$ and hence $hah^{-1} \in H$. This shows that $hah^{-1} \in K \cap H$, and in conclusion, $K \cap H$ is a normal subgroup of $H$.

    \item[4.]
        \boldmath
        \textbf{Let $f \colon \mathbb{R}^+ \to \mathbb{C}^\times$ be the map $f(x) = e^{ix}$. Prove that $f$ is a homomorphism, and determine its kernel and image.} \par
        \unboldmath
        Let $x, y \in \mathbb{R}^+$. Then $f(x + y) = e^{i(x + y)} = e^{ix} e^{iy} = f(x) f(y)$, and hence $f$ is a homomorphism. The kernel of $f$ is $\{ x \in \mathbb{R}^+ : e^{ix} = 1 \} = 2\pi \mathbb{Z}$, and the image of $f$ is $\{ z \in \mathbb{C}^\times | z = e^{ix} \text{ for some } x \in \mathbb{R}^+ \} = \{ z \in \mathbb{C}^\times : |z| = 1 \}$, the circle of radius $1$ around the origin in the complex plane.
\end{enumerate}

\subsection*{2.6}
\begin{enumerate}
    \item[2.]
        \boldmath
        \textbf{Describe all homomorphisms $\varphi : \mathbb{Z}^+ \to \mathbb{Z}^+$. Determine which are injective, which are surjective, and which are isomorphisms.} \par
        \unboldmath
        These are the functions which take the form $\varphi(x) = cx$ for some $c \in \mathbb{Z}$ and for all $x \in \mathbb{Z}^+$. $\varphi(x)$ is a homomorphism because $\varphi(x + y) = c(x + y) = cx + cy = \varphi(x) + \varphi(y)$; also, only linear functions that pass through the origin satify this property. Among these homomorphisms, all except $\varphi(x) = 0$ are injective, because $\ker(cx) = 0$ if $c \neq 0$, while $\ker(0) = \mathbb{Z}^+$. Only $\varphi(x) = x$ and $\varphi(x) = -x$ are surjective, while the other homomorphisms map to multiples of some number other than $1$. Because $\varphi(x) = x$ and $\varphi(x) = -x$ are both injective and surjective, they are the isomorphisms.

    \item[3.]
        \boldmath
        \textbf{Show that the functions $f = 1/x, g = (x - 1)/x$ generate a group of functions, the law of composition being composition of functions, that is isomorphic to the symmetric group $S_3$.} \par
        \unboldmath
        The group of functions generated by $f$ and $g$ is
        \begin{align*}
            G = \{ g \circ g \circ g = f \circ f = x, \quad g = \frac{x - 1}{x}, \quad g \circ g = \frac{-1}{x - 1}, \\
            f = \frac{1}{x}, \quad g \circ f = 1 - x, \quad g \circ g \circ f = \frac{x}{x - 1} \}.
        \end{align*}
        $G$ is a group because function composition is associative, and the identity element is $x$ under function composition; also, every element has an inverse ($g$ and $g \circ g$ are inverses, while each of the other elements is its own inverse). \par
        The map $\varphi(x) = 1$, $\varphi(g) = x$, $\varphi(g \circ g) = x^2$, $\varphi(f) = y$, $\varphi(g \circ f) = xy$, $\varphi(g \circ g \circ f) = x^2 y$, where the image is $S_3$, is a homomorphism by its definition along with the fact that the identities $g \circ g \circ g = f \circ f = x$, $f \circ g = g \circ g \circ f$, and $f \circ f \circ g = g \circ f$ hold, similar to $x^3 = y^2 = 1$, $yx = x^2y$, and $y^2x = xy$. Also, every element of the group of functions is mapped to a unique element of $S_3$, and the image is all of $S_3$, so the homomorphism is an isomorphism.

    \item[4.]
        \boldmath
        \textbf{Prove that in a group, the products $ab$ and $ba$ are conjugate elements.} \par
        \unboldmath
        Let $G$ denote the group. Because $b \in G$, $b^{-1} \in G$. Using left multiplication, we have that $ab = (b^{-1}b) (ab) = b^{-1} (ba) b$. From the definition, $ab$ is the conjugate of $ba$ by $b^{-1}$. Also, $ab = ab (aa^{-1}) = a (ba) a^{-1}$, so $ab$ is the conjugate of $ba$ by $a$. We can also rewrite the previous equations to show that $ba$ is the conjugate of $ab$ by $b$, and by $a^{-1}$.

    \item[6.]
        \boldmath
        \textbf{Are the matrices $
            \left[ \begin{array}{cc}
                1 & 1 \\
                  & 1
            \end{array}\right],
            \left[ \begin{array}{cc}
                1 & \\
                1 & 1
            \end{array}\right]
        $ conjugate elements of the group $GL_2(\mathbb{R})$? Are they conjugate elements of $SL_2(\mathbb{R})$?} \par
        \unboldmath
        To find all the elements in $GL_2(\mathbb{R})$ by which the matrices are conjugate, we write them in the form below:
        \begin{align*}
            \left[ \begin{array}{cc}
                    1 & 1 \\
                    0 & 1
            \end{array} \right]
            \left[ \begin{array}{cc}
                    a & b \\
                    c & d
            \end{array} \right]
            &= \left[ \begin{array}{cc}
                    a & b \\
                    c & d
            \end{array} \right]
            \left[ \begin{array}{cc}
                    1 & 0 \\
                    1 & 1
            \end{array} \right] \\
            \Rightarrow \left[ \begin{array}{cc}
                    a + c & b + d \\
                    c & d
            \end{array} \right]
            &= \left[ \begin{array}{cc}
                    a + b & b \\
                    c + d & d
            \end{array} \right] \\
            \left[ \begin{array}{cc}
                    1 & 0 \\
                    1 & 1
            \end{array} \right]
            \left[ \begin{array}{cc}
                    a & b \\
                    c & d
            \end{array} \right]
            &= \left[ \begin{array}{cc}
                    a & b \\
                    c & d
            \end{array} \right]
            \left[ \begin{array}{cc}
                    1 & 1 \\
                    0 & 1
            \end{array} \right] \\
            \Rightarrow \left[ \begin{array}{cc}
                    a & b \\
                    a + c & b + d
            \end{array} \right]
            &= \left[ \begin{array}{cc}
                    a & a + b \\
                    c & c + d
            \end{array} \right]
        \end{align*}
        These equations tell us that the matrix must have either $d = 0$ or $a = 0$, and have $b = c$ either way. In $GL_2(\mathbb{R})$, the two given matrices are conjugate by any matrix that fulfills these conditions. However, the determinant of any such matrix is $ad - bc = -b^2$, which is always nonpositive by the Trivial Inequality and thus cannot be $1$, so the two given matrices are not conjugate elements of $SL_2(\mathbb{R})$.

    \item[9.]
        \boldmath
        \textbf{Prove that a group $G$ and its opposite group $G^\circ$ are isomorphic.} \par
        \unboldmath
        The map $\varphi(x) = x^{-1}$ is a homomorphism, because $\varphi(ab) = (ab)^{-1} = b^{-1} a^{-1} = \varphi(b) \varphi(a) = \varphi(a) * \varphi(b)$. Furthermore, because $G$ and $G^\circ$ exhibit the same underlying set, the inverse mapping is a bijection, so $G$ and $G^\circ$ are isomorphic.

    \item[10.]
        \boldmath
        \textbf{Find all automorphisms of} \par
        \unboldmath
        \begin{enumerate}
            \item
                \boldmath
                \textbf{a cyclic group of order $10$.} \par
                \unboldmath
                Let $a$ be the generator of a cyclic group of order $10$. For a homomorphism $\varphi$, $\varphi(x^i) = \varphi(x)^i$, so $\varphi(x)$ for an arbitrary $x$ determines $\varphi$. Also, the order of $\varphi(x)$ must be 10, and the only elements of order $10$ are $x$, $x^3$, $x^7$, and $x^9$. The automorphisms are then $\varphi(x) = x$, $\varphi(x) = x^3$, $\varphi(x) = x^7$, and $\varphi(x) = x^9$.
                \iffalse
                    Its automorphisms are $\varphi(x) = x$ and $\varphi(x) = \begin{cases}
                        a^{10} / x &:x \neq e \\
                        e &:x = e
                    \end{cases}.
                    $ The latter is a homomorphism because for all $x = a^m, y = a^n$,
                    \begin{align*}
                        \varphi(xy) &= \frac{a^{10}}{xy} = \frac{a^{10}}{a^{(m + n) (\text{mod }10)}} = a^{10 - (m + n) (\text{mod }10)} \\
                        &= \varphi(a^m) \varphi(a^n) = \varphi(x) \varphi(y).
                    \end{align*}
                \fi

            \item
                \boldmath
                \textbf{the symmetric group $S_3$.} \par
                \unboldmath
                The automorphisms of $S_3$ are the conjugations by $g$, or $\varphi(z) = gzg^{-1}$, for all $g \in S_3$. There are only six automorphisms because each automorphism can permute the three elements of order $2$, or $y$, $xy$, and $x^2 y$, while the element of order $1$ and the elements of order $3$, which are inverses of one another, must be kept in place.
        \end{enumerate}
\end{enumerate}
\end{document}
