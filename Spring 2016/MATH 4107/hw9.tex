\documentclass[a4paper,12pt]{article}

\usepackage{amsfonts, amsmath, amsthm, enumitem, fancyhdr, tabularx}
\usepackage[margin=3.5cm]{geometry}
\allowdisplaybreaks
\pagestyle{fancy}
\rhead{Erick Lin}

\newcommand{\im}{\text{im}\,}

\begin{document}

\section*{MATH 4107 - HW9 Solutions}

\subsection*{7.5}
\begin{enumerate}
    \item[5.]
        \boldmath
        \textbf{Let $p$ and $q$ be permutations. Prove that the products $pq$ and $qp$ have cycles of equal sizes.} \par
        \unboldmath
        Since $pq = pq(pp^{-1}) = p(qp)p^{-1}$, $pq$ is the conjugate of $qp$ under $p$, and thus their cycle decompositions have the same orders.

    \item[7.]
        \boldmath
        \textbf{Prove that $A_n$ is the only subgroup of $S_n$ of index 2.} \par
        \unboldmath
        First, we prove that subgroups of index 2 are normal. For if $H \leq G$ is of index 2, then the cosets of $H$ are $H$ and $G \setminus H$. Because $H$ is a subgroup, for $x \in G \setminus H$, $xH \in G \setminus H$ and $Hx \in G \setminus H$. $xH$ and $Hx$ are the same coset, and hence $xH = Hx$. Because this is also true when $x \in H$, $H$ is normal. \par
        \iffalse
        Also, we know that the intersection of subgroups is a subgroup, and show that the same holds for normal subgroups. If $N_1$ and $N_2$ are normal, then the intersection $N = N_1 \cap N_2$ fulfills the property $gNg^{-1} \subseteq gN_1g^{-1}, gN_2g^{-1}$ for all $g \in G$, which means that $gNg^{-1}$ is contained in each of the normal subgroups and thus in $N$ as well. Because it is clear that $N \subseteq gNg^{-1}$, $N$ is normal. \par
        \fi
        \iffalse
        Next, we show that $A_n$ is the only nontrivial proper subgroup of $S_n$. \par
        Because $A_n$ is a subgroup of $S_n$ of index 2 by Lagrange's theorem and is the only proper subgroup (having an index that is not $1$ or $n$), it is also the only subgroup of index 2.
        \fi
        \iffalse
        Let $H \subset S_n$ be of index $2$. Since $S_n / H$ is isomorphic to $C_2 = \{ \pm 1 \}$, from the first isomorphism theorem, there is a surjective homomorphism $\varphi : S_n \to \{ \pm 1 \}$ with kernel $H$. Because all transpositions in $S_n$ are conjugate, for any transposition $(a\ b) \in S_n$, $\varphi((a\ b))$ is always the same element.
        \fi
        If $n = 2$, then $|S_2| = 2$, so any subgroup of $S_2$ of index $2$ has order $2/2 = 1$, which is the trivial group, or $A_2$. \par
        Otherwise, $n \geq 3$. We may let $H$ be a subgroup of $S_n$ of index $2$ (so that $H$ is normal), and let $(a\ b\ c) \notin H$, where $a, b, c \in [n]$. Then $(a\ b\ c)H$ and $H$ are different cosets of $H$ in $S_n$, or
        \begin{gather*}
            (a\ b\ c)H \neq H \\
            (a\ b\ c)(a\ b\ c)H \neq (a\ b\ c)H \\
            (a\ b\ c)^2 H \neq (a\ b\ c)H \\
            (a\ c\ b) H \neq (a\ b\ c)H.
        \end{gather*}
        Since $H$ has only $2$ different cosets, $(a\ c\ b) H = H$, and $(a\ c\ b)^2 \in H$. We know that all cycles with the same order are conjugate elements in $S_n$, so all $3$-cycles in $S_n$ are conjugate to $(a\ c\ b)$. Because $H$ is normal, $H$ must then contain all of these $3$-cycles. Because we know that $A_n$ is generated by $3$-cycles, $A_n \leq H$. Finally, because $A_n$ and $H$ are both of index $2$, $|A_n| = |H|$ and $A_n = H$.

    \item[10.]
        \boldmath
        \textbf{Verify the formulas
        \begin{gather*}
            24 = 1 + 3 + 6 + 6 + 8
        \end{gather*}
        and
        \begin{gather*}
            120 = 1 + 10 + 15 + 20 + 20 + 30 + 24
        \end{gather*}
        for the class equations of $S_4$ and $S_5$, and determine the centralizer of a representative element in each conjugacy class.} \par
        \unboldmath
        The conjugacy classes are permutations whose cycle decompositions have the same orders. The permutations in $S_4$ can have the cycle types $4$, $3 + 1$, $2 + 2$, $2 + 1 + 1$, and $1 + 1 + 1 + 1$; the number of possibilities are $4!/4 = 6$, $4!/3 = 8$, $[4!/(2! \times 2!)]/2 = 3$, $\binom{4}{2} = 6$, and $1$, respectively; these are the sizes of the conjugacy classes of $S_4$. The conjugacy class of size $1$ contains only the identity element, so its centralizer is all of $S_4$. Below, we list a representative of each of the other conjugacy classes and its centralizer.
        \begin{itemize}
            \item
                $(1\ 2\ 3\ 4)$: $\langle (1\ 2\ 3\ 4) \rangle = \{ (1\ 2\ 3\ 4), (1\ 3)(2\ 4), (1\ 4\ 3\ 2), () \}$
                
            \item
                $(1\ 2\ 3)$: $\langle (1\ 2\ 3) \rangle S_1 = \{ (1\ 2\ 3), (1\ 3\ 2), () \}$
                
            \item
                $(1\ 2)(3\ 4)$: $\{ (), (1\ 2)(3\ 4), (1\ 3\ 2\ 4), (1\ 4\ 2\ 3), (1\ 3)(2\ 4), (1\ 4)(2\ 3), \\
                ~~~~~~~~~~~~~~~~~(1\ 2), (3\ 4) \}$
                
            \item
                $(1\ 2)$: $\langle (1\ 2) \rangle S_2 = \langle (1\ 2) \rangle \{ (3\ 4), () \} = \{ (1\ 2)(3\ 4), (1\ 2), (3\ 4), () \}$
        \end{itemize}
        For $S_5$, the lengths of the cycles that compose permutations can be of the forms $5$, $4 + 1$, $3 + 2$, $3 + 1 + 1$, $2 + 2 + 1$, $2 + 1 + 1 + 1$, and $1 + 1 + 1 + 1 + 1$; the number of possibilities are $5!/5 = 24$, $5!/4 = 30$, $\binom{5}{3} \times 2 = 20$, $\binom{5}{3} (3 - 1)! = 20$, $\binom{5}{4} \binom{4}{2} / 2 = 15$, $\binom{5}{2} = 10$, and $1$, respectively; these are the sizes of the conjugacy classes of $S_5$. The conjugacy class of size $1$ contains only the identity element, so its centralizer is all of $S_5$. Below, we list a representative of each of the other conjugacy classes and its centralizer.
        \begin{itemize}
            \item
                $(1\ 2\ 3\ 4\ 5)$: $\langle (1\ 2\ 3\ 4\ 5) \rangle = \{ (1\ 2\ 3\ 4\ 5), (1\ 3\ 5\ 2\ 4), (1\ 4\ 2\ 5\ 3), \\
                ~~~~~~~~~~~~~~~~~(1\ 5\ 4\ 3\ 2), () \}$
                
            \item
                $(1\ 2\ 3\ 4)$: $\langle (1\ 2\ 3\ 4) \rangle S_1 = \{ (1\ 2\ 3\ 4), (1\ 3)(2\ 4), (1\ 4\ 3\ 2), () \}$
                
            \item
                $(1\ 2\ 3)(4\ 5)$: $\langle (1\ 2\ 3) \rangle \langle (4\ 5) \rangle = \{ (1\ 2\ 3)(4\ 5), (1\ 2\ 3), (1\ 3\ 2)(4\ 5), \\
                ~~~~~~~~~~~~~~~~~~(1\ 3\ 2), (4\ 5), () \}$
                
            \item
                $(1\ 2\ 3)$: $\langle (1\ 2\ 3) \rangle S_2 = \{ (1\ 2\ 3)(4\ 5), (1\ 2\ 3)(5\ 4), (1\ 3\ 2)(4\ 5), \\
                ~~~~~~~~~~~~(1\ 3\ 2)(5\ 4), (4\ 5), (5\ 4) \}$
                
            \item
                $(1\ 2)(3\ 4)$: $\{ (), (1\ 2)(3\ 4), (1\ 3\ 2\ 4), (1\ 4\ 2\ 3), (1\ 3)(2\ 4), (1\ 4)(2\ 3), \\
                ~~~~~~~~~~~~~~~~~(1\ 2), (3\ 4) \}$
                
            \item
                $(1\ 2)$: $\langle (1\ 2) \rangle S_3 \\
                ~~~~~~~~~~= \langle (1\ 2) \rangle \{ (3\ 4\ 5), (3\ 5\ 4), (4\ 3\ 5), (4\ 5\ 3), (5\ 3\ 4), (5\ 4\ 3) \} \\
                ~~~~~~~~~~ = \{ (1\ 2)(3\ 4\ 5), (1\ 2)(3\ 5\ 4), (1\ 2)(4\ 3\ 5), (1\ 2)(4\ 5\ 3), \\
                ~~~~~~~~~~~~~~~~(1\ 2)(5\ 3\ 4), (1\ 2)(5\ 4\ 3), (3\ 4\ 5), (3\ 5\ 4), (4\ 3\ 5), \\
                ~~~~~~~~~~~~~~~~(4\ 5\ 3), (5\ 3\ 4), (5\ 4\ 3) \}$
        \end{itemize}
\end{enumerate}

\subsection*{7.7}
\begin{enumerate}
    \item[4.]
        \begin{enumerate}[label=(\alph*)]
            \item
                \boldmath
                \textbf{Prove that no simple group has order $pq$, where $p$ and $q$ are prime.} \par
                \unboldmath
                Let $G$ denote a group of order $pq$. \par
                Without loss of generality, say $p < q$. Then by the third Sylow theorem, the number of $q$-Sylow subgroups divides $p$ and is congruent to $1$ modulo $q$. Since $kq + 1 > p$ for all positive $k$, $1$ is the only number that satisfies both conditions. Then the unique $q$-Sylow subgroup is normal in $G$ because it conjugates with itself from the second Sylow theorem, so $G$ is not simple. \par
                If $p = q$, then $|G| = p^2$. Because we know that $G \cong C_p \times C_p$ or $G \cong C_{p^2}$ from its order, $C_p \times \{ 1 \}$ or $C_p$ is a proper normal subgroup, so $G$ is not simple.

            \item
                \boldmath
                \textbf{Prove that no simple group has order $p^2 q$, where $p$ and $q$ are prime.} \par
                \unboldmath
                Let $G$ denote a group of order $pq$. \par
                If $p > q$, then by the third Sylow theorem, the number of $p$-Sylow subgroups divides $q$ and is congruent to $1$ modulo $p$. Since $kp + 1> q$ for all positive $k$, $1$ is the only number that satisfies both conditions. Then the unique $p$-Sylow subgroup is normal, and $G$ is not simple. \par
                If $p < q$, then the number of $q$-Sylow subgroups divides $p^2$ and is congruent to $1$ modulo $q$. If this number is not $1$, then it is given by $p^2 = kq + 1$ for some $k$, or $p^2 \cong 1$ (mod $q$). By arithmetic, we have that $(p - 1)(p + 1) \cong 0$ (mod $q$). Since $p - 1 < q$ and $p$ is prime, hence greater than $1$, $p - 1 \not\cong 0$ (mod $q$). Therefore, it is necessary that $p + 1 \cong 0$ (mod $q$). Since $q$ is not the smallest prime, $q$ is odd, so $p$ must be even. Thus, $p = 2$ and $q = 3$. The number of $3$-Sylow subgroups is $p^2 = 4$, and since they all contain the identity element but share no other element since they are of order $3$ (and hence contain some element and its inverse), there are $8$ unique non-identity elements contained in the union of the $3$-Sylow subgroups. Since this leaves $12 - 8 = 4$ remaining elements and any Sylow-2 subgroup of $G$ is of order $4$, these elements constitute exactly the unique Sylow-2 subgroup, so it is a nontrivial normal subgroup. \par
                If $p = q$, then $|G| = p^3$ and $G$ is a $p$-group. Since the center of a $p$-group is not trivial and automatically normal, $G$ is not simple.
        \end{enumerate}

    \item[5.]
        \boldmath
        \textbf{Find Sylow 2-subgroups of the following groups: (a) $D_{10}$, (b) $T$, (c) $O$, (d) $I$.} \par
        \unboldmath
        \begin{enumerate}
            \item
                Since $|D_{10}| = 20 = 4 \times 5$, the Sylow 2-subgroups are of order $2^2$. By the third Sylow theorem, the number of Sylow 2-subgroups divides $5$ and is congruent to $1$ modulo $2$, so the number is either $1$ or $5$. We know then that the number must be $5$, because each pair of opposite edges is stabilized by the group of order $4$ generated by the $180^\circ$ rotation action and the reflection across the axis cutting through the two edges.

            \item
                $T \cong S_4$, and since $|S_4| = 12 = 2^2 \times 3$, the Sylow 2-subgroups are of order $2^2 = 4$. By the third Sylow theorem, the number of Sylow 2-subgroups divides $3$ and is congruent to $1$ modulo $2$, so the number is either $1$ or $3$. The $180^\circ$ rotations around two edges generate a subgroup of order $4$, of which there is only $1$.

            \item
                Since $|O| = 24 = 2^3 \times 3$, the Sylow 2-subgroups are of order $2^3 = 8$. By the third Sylow theorem, the number of Sylow 2-subgroups divides $3$ and is congruent to $1$ modulo $2$, so the number is either $1$ or $3$. Then we know that the number is $3$, since the subgroups of order $8$ are given by the groups each generated by a rotation action around one of the three axes, and the $180^\circ$ rotation action across a perpendicular axis.

            \item
                $I \cong A_5$, and since $|I| = 60 = 2^2 \times 15$, the Sylow 2-subgroups are of order $2^2 = 4$. By the third Sylow theorem, the number of Sylow 2-subgroups divides $15$ and is congruent to $1$ modulo $2$, so the number is $1$, $3$, $5$, or $15$. Consider opposite pairs of edges, of which there are $15$. For each pair, then there are two different $180^\circ$ rotation actions that stabilize this pair of edges, and they generate a group of order $4$.
        \end{enumerate}

    \item[9.]
        \boldmath
        \textbf{Classify groups of order (a) 33, (b) 18, (c) 20, (d) 30.} \par
        \unboldmath
        In all cases, let $G$ denote the group.
        \begin{enumerate}
            \item
                From the third Sylow theorem, the number of $11$-Sylow subgroups divides $3$ and is congruent to $1$ modulo $11$, so there is only $1$ such subgroup, say $H$, and it is normal. Also, the number of $3$-Sylow subgroups divides $11$ and is congruent to $1$ modulo $11$, of which there is also only one, say $K$, which is also normal. $H$ and $K$ are of prime order and hence cyclic, so $H \cap K = \{ 1 \}$, and $HK = G$. Thus, $H \times K \cong G$ and all groups of order 33 belong in this conjugacy class.

            \item
                The factorization of $18$ is $3^2 \times 2$. The number of $3$-Sylow subgroups divides $2$ and is congruent to $1$ modulo $3$, so there is only $1$ such subgroup, say $H$, and it is normal. Also, the number of $2$-Sylow subgroups divides $3^2 = 9$ and is congruent to $1$ modulo $2$, it is $1$, $3$, or $9$. \par
                In the former case, if $K$ denotes the unique $2$-Sylow subgroup, then $G$ is isomorphic to $H \times K$, which is abelian, so all abelian groups of order $18$ are cyclic. \par
                Otherwise, $H \cong C_9$ or $H \cong C_3 \times C_3$. Since $x^2 = 1$ and $xyx^{-1} = y^a$, $a^2 \cong 1$ (mod $9$). Then $y = x^2 y x^{-2} = xy^a x^{-1} = (xyx^{-1})^a = (y^a)^a = y^{a^2}$, so $a^2 \equiv 1$ (mod $9$), so $a \equiv 1$ or $a \equiv -1$ (mod $9$). We end up getting three more groups from this, two being $C_3 \times C_3 \times C_2$ and $C_3 \times S_3$. 

            \item
                The factorization of $20$ is $2^2 \times 5$. The number of $2$-Sylow subgroups divides $5$ and is congruent to $1$ modulo $2$, so the number of such subgroups is either $1$ or $5$. Also, the number of $5$-Sylow subgroups divides $2$ and is congruent to $1$ modulo $5$, so there is only $1$ such subgroup, say $H$, which is normal. \par
                If the number of $2$-Sylow subgroups is $1$, then it is normal, and $G \cong C_2 \times C_2 \times C_5$ or $G \cong C_4 \times C_5$. \par
                Otherwise, let $x \in H$ so that $x^5 = 1$, and let $y \in G \setminus H$, $y \neq 1$, so that $y$ has order $4$. Assume that $y$ generates the cyclic subgroup $C_4$, so $y^4 xy^{-4} = x$. Also, since $H$, is normal, $yxy^{-1} = x^i$ for some $i$ in the range $0 \leq i < 5$. If $i = 1$, then $x$ commutes with $y$, so the group is isomorphic to $C_4 \times C_5$. Since $y = y^{-1}$ we have that $y^3 x y^{-3} = x^{a^3}$. Since $2^3 \equiv 3$ (mod $5$), the cases where $i = 2$ and $i = 3$ are the same, and the group is generated by the element $xy = y^2x$. Finally, if $i = 4$, then $G$ is generated by $xy = y^{-1}x$. \par
                If $H \cong C_2 \times C_2$, then let $H_0 = C_2 \times \{ 1 \}$. Then $H_0 K \leq G$. $|H_0 K| = 10$ so the index of $H_0 K$ in $G$ is 2. Therefore, $H_0 K$ is normal in $G$. If $y^{10} = 1$, then $xyx^{-1} = y^a$. $x^2 = 1$, so $a^2 \equiv 1$ (mod $10$), and $a \equiv 1$ or $a \equiv 9$ (mod $10$). If $a = 1$, then the group is ismorphic to $C_2 \times C_{10}$, and if $a = 9$, then $xy = y^{-1} x$ generates the dihedral group $D_{10}$.

            \item
                The factorization of $30$ is $2 \times 3 \times 5$. The number of $2$-Sylow subgroups divides $15$ and is congruent to $1$ modulo $2$, so the number of such subgroups is $1, 3, 5,$ or $15$. The number of $3$-Sylow subgroups divides $10$ and is congruent to $1$ modulo $3$, so there are $1$ or $10$ such subgroups. The number of $5$-Sylow subgroups divides $6$ and is congruent to $1$ modulo $5$, so there are either $1$ or $6$ such subgroups. \par
                If the $2$-Sylow, $3$-Sylow, and $5$-Sylow subgroups are all unique, then we may denote them by $H$, $K$, and $M$. Using similar reasoning as above, $G$ is isomorphic to $H \times K \times M$. \par
                The case where there are $3$ $2$-Sylow subgroups and the others are unique is impossible, because there are $3$ unique non-identity elements contained in the $2$-Sylow subgroups, leaving $30 - 3 - 1 = 26$ non-identity elements for the $3$-Sylow and $5$-Sylow subgroups, which is impossible. Similarly, it is impossible for there to be $5$ or $15$ $2$-Sylow subgroups, because they would leave $24$ and $14$ non-identity elements, respectively. \par
                If there are $3$ $2$-Sylow subgroups, $1$ $3$-Sylow subgroup, and $6$ $5$-Sylow subgroups, then there are $3$ unique non-identity elements in the union of the $2$-Sylows, $2$ in the $3$-Sylow, and $24$ in the $5$-Sylow subgroup. Including the identity, these add up to $30$ elements.
        \end{enumerate}
\end{enumerate}
\end{document}
