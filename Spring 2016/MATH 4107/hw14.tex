\documentclass[a4paper,12pt]{article}

\usepackage{amsfonts, amsmath, amssymb, amsthm, enumitem, fancyhdr, tabularx}
\usepackage[margin=3.5cm]{geometry}
\allowdisplaybreaks
\pagestyle{fancy}
\rhead{Erick Lin}

\newcommand{\im}{\text{im}\,}

\begin{document}

\section*{MATH 4107 - HW14 Solutions}

\subsection*{15.7}
\begin{enumerate}
    \item[1.]
        \boldmath
        \textbf{Identify the group $\mathbb{F}_4^+$.} \par
        \unboldmath
        The elements are the roots of the polynomial $x^4 - x$, and we know that two of the elements are $0$ and $1$, with the other two being the roots of $x^2 + x + 1$. The group cannot be cyclic, because otherwise $1$ would generate the entire group, contradicting the observation that $1$ does not generate a root of $x^2 + x + 1$. The only other group of order $4$ is the Klein four group $V_4$, so this identifies the group.

    \item[2.]
        \boldmath
        \textbf{Determine the irreducible polynomial of each of the elements of
            \begin{align*}
                \mathbb{F}_8 = \{ 0, 1, \beta, 1 + \beta, \beta^2, 1 + \beta^2, \beta + \beta^2, 1 + \beta + \beta^2 \},
            \end{align*}
        where $\beta$ is a root of one of the irreducible cubic factors.} \par
        \unboldmath
        We know that $x$ is the irreducible polynomial of $0$, and $x - 1$ is the irreducible polynomial of $1$. \par
        Say that $\beta$ is a root of $x^3 + x + 1$. Using the relations $1 + 1 = 0$ and $\beta^3 + \beta + 1 = 0$, we have that
        \begin{align*}
            [ x^3 + x + 1 ]_{x = \beta^2} &= \beta^6 + \beta^2 + 1 = (-\beta - 1)^2 + \beta^2 + 1 \\
            &= 2\beta^2 + 2\beta + 2 = 0 \\
            [ x^3 + x + 1 ]_{x = \beta + \beta^2} &= \beta^3 + 3\beta^2(\beta^2) + 3\beta(\beta^4) + \beta^6 + \beta + \beta^2 + 1 \\
            &= \beta^3 + \beta + 1 = 0,
        \end{align*}
        which shows that $\beta^2$ and $\beta + \beta^2$ are the other roots of $x^3 + x + 1$. \par
        This shows that the remaining elements $1 + \beta$, $1 + \beta^2$, and $1 + \beta + \beta^2$ must be the roots of the irreducible polynomial $x^3 + x^2 + 1$.

    \item[3.]
        \boldmath
        \textbf{Find a 13th root of 2 in the field $\mathbb{F}_{13}$.} \par
        \unboldmath
        \iffalse
        $\mathbb{F}_{13}$ is simply $\mathbb{Z}/13\mathbb{Z}$, because since the number of elements is prime, no two elements multiplied together give $0$ modulo 13. This also implies that every element generates $\mathbb{F}_{13}^\times$, and in particular, the only $x$ for which $x^{13} = 2$ is $x = 2$.
        \fi
        From the theorem, the elements of $\mathbb{F}_{13}$ are roots of the polynomial $x^{13} - x$, which means that $x^{13} = x$. Therefore, $2^{13} = 2$. \par
        Another way to look at it is that $\mathbb{F}_{13}^\times = \{ 1, \cdots, 12 \}$, a group of degree 12, so for any $x \in \mathbb{F}_{13}^\times$, $x^{12} = 1$, and thus $x^{13} = x$.

    \item[4.]
        \boldmath
        \textbf{Determine the number of irreducible polynomials of degree 3 over $\mathbb{F}_3$ and over $\mathbb{F}_5$.} \par
        \unboldmath
        For $\mathbb{F}_q$, the solution uses a combinatorial approach: enumerate the possible monic polynomials of degree $3$ (of which there are $q^3$, by counting possibilities for coefficients), and eliminate the ones that have any element of the field as a root. The latter includes products of $3$ factors of degree $1$ (thus having $3$ roots; this involves dividing up $3$ factors among $q$ choices, giving $\binom{q + 3 - 1}{3}$), as well as products of a factor of degree $1$ and an irreducible factor of degree $2$. Similarly as before, there are $\binom{q + 2 - 1}{2}$ reducible factors of degree $2$. Overall, the desired number is
        \begin{align*}
            q^3 - \binom{q + 3 - 1}{3} - q \left[ q^2 - \binom{q + 2 - 1}{2} \right]
        \end{align*}
        This results for $\mathbb{F}_3$ and $\mathbb{F}_5$ are $8$ and $40$, respectively.

    \item[5.]
        \boldmath
        \textbf{Factor $x^9 - x$ and $x^{27} - x$ in $\mathbb{F}_3$.} \par
        \unboldmath
        Since $9 = 3^2$, the factors in the former case are the irreducible polynomials of degree either $1$ or $2$. By enumerating the possible factors of degree $2$ and eliminating those that are reducible, we find that the irreducible factors of degree $2$ are $x^2 + 1$, $x^2 + x + 2$, and $x^2 + 2x + 2$. Thus,
        \begin{align*}
            x^9 - x = x(x - 1)(x - 2)(x^2 + 1)(x^2 + x + 2)(x^2 + 2x + 2).
        \end{align*}
        By similar reasoning, since $27 = 3^3$ (and with the aid of a computer),
        \begin{align*}
            x^{27} - x &= x(x - 1)(x - 2)(x^3 + 2x + 1)(x^3 + 2x + 2)(x^3 + x^2 + 2) \\
            &(x^3 + x^2 + x + 2)(x^3 + x^2 + 2x + 1)(x^3 + 2x^2 + 1) \\
            &(x^3 + 2x^2 + x + 1)(x^3 + 2x^2 + 2x + 2).
        \end{align*}

    \item[6.]
        \boldmath
        \textbf{Factor the polynomial $x^{16} - x$ over the fields $\mathbb{F}_4$ and $\mathbb{F}_8$.} \par
        \unboldmath
        The factorization over $\mathbb{F}_8$ is the same as that over $\mathbb{F}_2$, which is
        \begin{align*}
            x^{16} - x = x(x + 1)(x^2 + x + 1)(x^4 + x + 1)(x^4 + x^3 + 1)(x^4 + x^3 + x^2 + x + 1).
        \end{align*}
        $\mathbb{F}_4$ has the extra elements $\alpha, \alpha + 1$ where $\alpha$ is a root of $x^2 + x + 1$. $x^2 + x + 1$ then becomes $(x + \alpha)(x + (\alpha + 1))$, and all the degree-4 polynomials split into two degree-2 polynomials.

    \item[8.]
        \boldmath
        \textbf{The polynomials $f(x) = x^3 + x + 1$ and $g(x) = x^3 + x^2 + 1$ are irreducible over $\mathbb{F}_2$. Let $K$ be the field extension obtained by adjoining a root of $f$, and let $L$ be the extension obtained by adjoining a root of $g$. Describe explicitly an isomorphism from $K$ to $L$, and determine the number of such isomorphisms.} \par
        \unboldmath
        Let $\alpha, \beta$ be the roots described respectively. Then the isomorphism $\varphi : K \to L$ is given by $\varphi(\alpha) = \beta + 1$, because
        \begin{align*}
            f(\varphi(\alpha)) &= (\beta + 1)^3 + (\beta + 1) + 1 \\
            &= \beta^3 + 3\beta^2 + 4\beta + 3 \\
            &= \beta^3 + \beta^2 + 1 \\
            &= g(\beta)
        \end{align*}
\end{enumerate}
\end{document}
