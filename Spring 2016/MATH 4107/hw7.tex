\documentclass[a4paper,12pt]{article}

\usepackage{amsfonts, amsmath, amsthm, fancyhdr, tabularx}
\usepackage[margin=3.5cm]{geometry}
\allowdisplaybreaks
\pagestyle{fancy}
\rhead{Erick Lin}

\newcommand{\im}{\text{im}\,}

\begin{document}

\section*{MATH 4107 - HW7 Solutions}

\subsection*{6.9}
\begin{enumerate}
    \item[1.]
        \boldmath
        \textbf{Use the counting formula to determine the orders of the groups of rotational symmetries of a cube and a tetrahedron.} \par
        \unboldmath
        Of a cube, the stabilizer of any particular face is the cyclic subgroup of rotational symmetries that permute the four faces adjacent to this particular face, which is of size 4, and since any face can be rotated to any other face, the orbit of any particular face is of size 6; thus the order of the group of rotational symmetries is $4 \times 6 = 24$. \par
        Of a tetrahedron, the stabilizer of any particular face is the cyclic subgroup of rotational symmetries that permute the three faces adjacent to this particular face, which is of size 3, and since any face can be rotated to any other face, the orbit of any particular face is of size 4; thus the order of the group of rotational symmetries is $3 \times 4 = 12$.

    \item[2.]
        \boldmath
        \textbf{Let $G$ be the group of rotational symmetries of a cube, let $G_v$, $G_e$, $G_f$ be the stabilizers of a vertex $v$, an edge $e$, and a face $f$ of the cube, and let $V$, $E$, $F$ be the sets of vertices, edges, and faces, respectively. Determine the formulas that represent the decomposition of each of the three sets into orbits for each of the subgroups.} \par
        \unboldmath
        Let $A, B, C, D, H, I, J, K$ denote vertices such that $v = A$, $e = AB$, $f = ABCD$, and $HIJK$ is the face opposite $f$ (with the vertices in order). The table below gives the decompositions of $V$, $E$, and $F$ into orbits given each of the following stabilizers:
        \begin{table}[h]
            \begin{tabularx}{\textwidth}{|l|X|X|X|}
                \hline
                & $G_v$ & $G_e$ & $G_f$ \\ \hline
                $V$
                    & $\{ A \}, \{ B, D, H \}$, $\{ C, I, K \}$, $\{ J \}$
                    & $\{ A, B\}, \{ C, D \}, \{ H, I \}$, $\{ J, K \}$
                    & $\{ A, B, C, D \}$, $\{ H, I, J, K \}$
                \\ \hline
                $E$
                & $\{ AB, AD, AH \}$, $\{ BC, HI, DK \},$ $\{ BI, HK, CD \}$, $\{ JC, JI, JK \}$
                    & $\{ AB \}, \{ AD, BI \}$, $\{ AH, BC \}, \{ CD, HI \}$, $\{ DK, IJ \}, \{ CJ, HK \}$, $\{ JK \}$
                    & $\{ AB, BC, CD, AD \}$, $\{ AH, BI, CJ, DK \}$, $\{ HI, IJ, JK, HK \}$
                \\ \hline
                $F$
                    & $\{ ABCD, ABIH$, $ADKH \}$, $\{ BCJI, CDKJ$, $HKJI \}$
                    & $\{ ABCD, ABIH \}$, $\{ ADKH, BCJI \}$, $\{ CDKJ, HKJI \}$
                    & $\{ ABCD \}, \{ HKJI \}$, $\{ ABIH, BCJI$, $CDKJ, ADKH \}$
                \\ \hline
            \end{tabularx}
        \end{table}

    \item[3.]
        \boldmath
        \textbf{Determine the order of the group of symmetries of a dodecahedron, when orientation-reversing symmetries such as reflections in planes are allowed.} \par
        \unboldmath
        Since each face is a pentagon, it has 5 rotational and 5 reflectional symmetries, and the orbit of any face is all 12 faces. Then the order of the group of symmetries is $(5 + 5) * 12 = 120$.

    \item[4.]
        \boldmath
        \textbf{Identify the group $T'$ of all symmetries of a regular tetrahedron, including orientation-reversing symmetries.} \par
        \unboldmath
        Let $A, B, C, D$ denote the vertices of the tetrahedron. Without loss of generality, consider the symmetries that keep $A$ fixed. These include the rotations of $120^\circ, 240^\circ$, and $360^\circ$ around the axis of the tetrahedron that contains $A$. There are also three reflections by planes that pass through $A$ -- these are the planes that bisect $BC$, $CD$, and $DA$. We have enumerated 6 symmetries, and since $A$ was arbitrary, the number of symmetries considering all the vertices is $6 \times 4 = 24$. Because combinations of actions can permute the vertices in every possible way and $|S_4| = 24$, $T' \cong |S_4|$.
\end{enumerate}

\subsection*{6.10}
\begin{enumerate}
    \item[1.]
        \boldmath
        \textbf{Determine the orders of the orbits for left multiplication on the set of subsets of order 3 of $D_3$.} \par
        \unboldmath
        If $x$ denotes the reflection across some axis of the equilateral triangle and $y$ denotes its $120^\circ$ rotation clockwise, then $D_3 = \{ e, x, y, xy, y^2, xy^2 \}$. The number of subsets of order $3$ of $D_3$ is $\binom{6}{3} = 20$, and the orbits and their orders are given by
        \begin{gather*}
            \{ e, y, y^2 \}, \{ x, xy, xy^2 \} \text{: 2} \\
            \{ e, x, y \}, \{ e, x, xy \}, \{ y, xy, y^2 \}, \{ y, xy, xy^2 \}, \{ e, y^2, xy^2 \}, \{ x, y^2, xy^2 \} \text{: 6} \\
            \{ e, y, xy \}, \{ x, y, xy \}, \{ y, y^2, xy^2 \}, \{ xy, y^2, xy^2, \}, \{ e, y^2, xy^2 \}, \{ e, x, xy^2 \} \text{: 6} \\
            \{ e, y, xy^2 \}, \{ x, y^2, xy \}, \{ x, y, y^2 \}, \{ e, xy, xy^2 \}, \{ e, y^2, xy \}, \{ x, y, xy^2 \} \text{: 6}.
        \end{gather*}
\end{enumerate}

\subsection*{6.11}
\begin{enumerate}
    \item[1.]
        \boldmath
        \textbf{Describe all the ways in which $S_3$ can operate on a set of four elements.} \par
        \unboldmath
        If $S$ denotes the set of 4 elements, then a bijection exists between operations of $S_3$ on $S$ and permutation representations $\varphi : S_3 \to \text{Perm}(S) = S_4$. \par
        Because the $\varphi$ are homomorphisms and the generators of $S_3$ are $x = (1\ 2\ 3)$ and $y = (1\ 2)$, any $\varphi$ must have $\varphi(x)$ of order 1 or order 3, and $\varphi(y)$ of order 1 or order 2. The number of elements of $S_4$ with order 3, which are all cyclic permutations, is the number of ways to choose 3 elements multiplied by the number of cycles consisting of these three elements, or $\binom{4}{3} (3 - 1)! = 8$. The number of elements of $S_4$ with order 2, which include permutations that are cycles of length 2 and those that are products of two cycles of length 2, is $\binom{4}{2} + \binom{4}{2}/2 = 6 + 3 = 9$. There is only 1 element of order 1, the identity. \par
        If $\varphi(x)$ has order 1, then $\varphi(y)$ is unrestricted and has $9 + 1 = 10$ possibilities. Otherwise, if $\varphi(x)$ has order 3, then the only valid homomorphisms are those that map an element of order 2 in $S_3$ to an element of order 2 in $S_4$. The number of such homomorphisms is $\binom{|S_3|}{2} \times 8 = 24$.

    \item[4.]
        \boldmath
        \textbf{Let $G$ be the dihedral group $D_4$ of symmetries of a square. Is the action of $G$ on the vertices a faithful action? on the diagonals?} \par
        \unboldmath
        The action of $G$ on the vertices is faithful because any rotational or reflectional symmetry other than the identity either takes a vertex to a neighboring vertex or, in the case of a reflection about the diagonal, swaps two opposite vertices. The action of $G$ on the diagonals is not faithful, because composition of the two types of reflections and composition of two counterclockwise rotations by $90^\circ$ are actions that take a diagonal back to itself.
\end{enumerate}

\subsection*{Not in text:}
\begin{enumerate}
    \item[1.]
        \boldmath
        \textbf{Let $S_3$ act on the product set $S = \{ 1, 2, 3 \} \times \{ 1, 2, 3 \}$ diagonally: that is, $\sigma \cdot (i, j) := (\sigma(i), \sigma(j))$.} \par
        \unboldmath
        \begin{enumerate}
            \item
                \boldmath
                \textbf{Decompose $S$ into orbits, and find the stabilizer of each element of $S$.} \par
                \unboldmath
                $S$ has $3 \times 3 = 9$ elements, and
                \begin{align*}
                    S = \{ (1, 1), (2, 2), (3, 3) \} \cup \{ (1, 2), (2, 1), (1, 3), (3, 1), (2, 3), (3, 2) \}
                \end{align*}
                is the orbit decomposition because while $\sigma$ can take any element of $S_3$ to any other element, it must take two identical elements to two identical elements. All the elements of $S$ have the identity permutation in their stabilizers; in addition, the stabilizer of $(1, 1)$ contains the permutation $(2\ 3)$, the stabilizer of $(2, 2)$ contains $(1\ 3)$, and the stabilizer of $(3, 3)$ contains $(1\ 2)$. By the orbit-stabilizer theorem, these must describe the stabilizers entirely.

            \item
                \boldmath
                \textbf{Let $x = (1\ 2\ 3) \in S_3$ and $y = (1\ 2) \in S_3$. Determine the images of $x$ and $y$ under the permutation representation $\varphi : S_3 \to \text{Perm}(S) \cong S_9$.} \par
                \unboldmath
                Denoting the elements of $S$ as in (a),
                \begin{gather*}                
                    \varphi(x) = \bigl( (1, 1)\ (2, 2)\ (3, 3) \bigr) \bigl( (1, 2)\ (2, 3)\ (3, 1) \bigr) \bigl( (2, 1)\ (3, 2)\ (1, 3) \bigr) \\
                    \varphi(y) = \bigl( (1, 1)\ (2, 2) \bigr) \bigl( (3, 3) \bigr) \bigl( (1, 2)\ (2, 1) \bigr) \bigl( (1, 3)\ (2, 3) \bigr) \bigl( (3, 1)\ (3, 2) \bigr).
                \end{gather*}                
        \end{enumerate}

    \item[2.]
        \boldmath
        \textbf{Describe all ways in which $S_4$ acts on a set with two elements.} \par
        \unboldmath
        This is equivalent to describing the homomorphisms $\varphi$ from $S_4$ to $S_2$. Because the $\varphi$ are homomorphisms, any element of order $1$ or $3$ must map to an element of order $1$, and any element of order $2$ or $4$ must map to any element of order $1$ or $2$. \par
        Since $\varphi((1\ 2\ 4)) = \varphi((1\ 2\ 3)) = 1$, $(1\ 2\ 4)(1\ 2\ 3) = (1\ 4)(2\ 3)$ implies that $1 = \varphi((1\ 2\ 4)(1\ 2\ 3)) = \varphi((1\ 4)(2\ 3)) = \varphi((1\ 4)) \varphi((2\ 3))$, the $2$-cycles $(1\ 4)$ and $(1\ 3)$ both map to either $(1\ 2)$ or $()$. A similar result occurs for the $2$-cycles $(1\ 4)$ and $(2\ 4)$, so we can conclude that all $2$-cycles map to the same of the two elements. This gives $2$ possible cases. \par
        Regardless of whether the $2$-cycles all map to $()$ to $(1\ 2)$, all products of $2$-cycles map to $()$. Since $(1\ 3\ 2\ 4)(1\ 2\ 3\ 4) = (1\ 4\ 3)$, $\varphi((1\ 3\ 2\ 4)) \varphi((1\ 2\ 3\ 4)) = \varphi((1\ 3\ 2\ 4)(1\ 2\ 3\ 4)) = \varphi(1\ 4\ 3) = 1$, which means that $(1\ 3\ 2\ 4)$ and $(1\ 2\ 3\ 4)$ both map to either $()$ and $(1\ 2)$, and by extending this argument, we have that all $4$-cycles map to the same element in $S_2$, so each case described above has $2$ possible homomorphisms. Thus, the total number of homomorphisms is $2 \times 2 = 4$.
        \iffalse
            First, it can be seen that the mapping of any 2-cycle determines the mappings of all the 2-cycles in $S_4$. \par
            If the 2-cycles all map to the identity permutation in $S_2$, then it can be directly shown that the 4-cycles in $S_4$ must all map to the same element in $S_2$, of which there are 2 choices. Otherwise, if the 2-cycles all map to the permutation $(1\ 2)$, then any product of two 2-cycles maps to $(1\ 2)(1\ 2) = ()$, the identity permutation, and the 4-cycles must all again map to the same element in $S_2$. Thus, the total number of possibilities for homomorphisms is $2 + 2 = 4$.
        \fi
        \iffalse
            $(124)(123) = (14)(23) \Rightarrow \varphi((124)(123)) = \varphi((14)(23)) = \varphi((14)(23)) = \varphi((14))\varphi((23)) = 1$ since $\varphi((124)) = \varphi((123)) = 1$. So either $\varphi((14))$ and $\varphi((13))$ both equal $(12)$ or both equal $(1)$. Similarly, both $\varphi((14))$ and $\varphi((24))$ map to $(1)$ or to $(12)$. So the mapping of $(ab)$ determines the mapping of the other 2-cycles. \par
            Case 1: $\varphi(ab) = (1)$ for all 2-cycles ab. Then $\varphi((ab)(cd)) = \varphi((ab))\varphi((cd)) = 1$ so $ab$ products of 2-cycles $((ab)(cd))$ map to 1. Note that $(1324)(1234) = (143)$ so $\varphi((1314))\varphi((1234)) = \varphi((1324))\varphi((1234)) = 1$ so both $(1324)$ and $(1234)$ map to $(1)$ or to $(12)$. There are no further restrictions, so this describes 2 possible homomorphisms. \par
            Case 2: $\varphi(ab) = (12)$ for all 2-cycles $ab$. By the same arguments as above, all products of 2-cycles map to (1) and all 4-cycles map to (1) or to (12), giving 2 possible homomorphisms.
        \fi

    \item[3.]
        \boldmath
        \textbf{Let $n \geq 5$. It is a fact that the alternating group $A_n$ is \textit{simple}: that is, it has no proper normal subgroups. Describe all ways in which $A_n$ acts on a set $S$ with $m < n$ elements.} \par
        \unboldmath
        This is equivalent to describing the homomorphisms from $A_n$ to $S_m$. If $K$ is the kernel of some homomorphism $\varphi$, then $K$ is normal in $A_n$, and since $A_n$ is simple, we have that either $K = \{ () \}$ or $K = A_n$. We can rule out the former case because that would make $\varphi$ injective and hence $|S_m| = m! \geq |A_n| = n!/2$, or $m \geq n$ (since $n \geq 5$). Otherwise, the kernel is the entire set, so $\varphi$ is the trivial homomorphism.
\end{enumerate}
\end{document}
