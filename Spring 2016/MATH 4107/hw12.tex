\documentclass[a4paper,12pt]{article}

\usepackage{amsfonts, amsmath, amssymb, amsthm, enumitem, fancyhdr, tabularx}
\usepackage[margin=3.5cm]{geometry}
\allowdisplaybreaks
\pagestyle{fancy}
\rhead{Erick Lin}

\newcommand{\im}{\text{im}\,}

\begin{document}

\section*{MATH 4107 - HW12 Solutions}

I was aided by Saurabh Kumar on this assignment.
\subsection*{11.4}
\begin{enumerate}
    \item[1.]
        \boldmath
        \textbf{Consider the homomorphism $\mathbb{Z}[x] \to \mathbb{Z}$ that sends $x \rightsquigarrow 1$. Explain what the Correspondence Theorem, when applied to this map, says about ideals of $\mathbb{Z}[x]$.} \par
        \unboldmath
        Let $\varphi$ denote this homomorphism, which is surjective. Its kernel $K$ is the set of all polynomials with integer coefficients that have $1$ as a root, or in other words, the set of polynomials divisible by $x - 1$, so it is the principal ideal of $\mathbb{Z}[x]$ generated by $x - 1$. \par
        The Correspondence Theorem relates ideals $I$ of $\mathbb{Z}[x]$ that contain $x - 1$ to ideals $J$ of $\mathbb{Z}$, by $J = \varphi(I)$ and $I = \varphi^{-1}(J)$. Here $J$ is a principal ideal, generated by an integer $n$. Let $I_1$ denote the ideal of $\mathbb{Z}[x]$ generated by $x - 1$ and $n$. Then $I_1$ contains $K$, and its image is equal to $J$, so from the Correspondence Theorem, $I_1 = I$. In conclusion, every ideal of $\mathbb{Z}[x]$ that contains $x - 1$ has the form $I = (x - 1, n)$ for some integer $n$.

    \item[4.]
        \boldmath
        \textbf{Are the rings $\mathbb{Z}[x]/(x^2 + 7)$ and $\mathbb{Z}[x]/(2x^2 + 7)$ isomorphic?} \par
        \unboldmath
        Assume for the purpose of contradiction that these rings are isomorphic, so that there exists a surjective homomorphism $\pi : \mathbb{Z}[x] \to \mathbb{Z}[x]/(x^2 + 7)$ with kernel $(2x^2 + 7)$. We can write $\pi(x) = a_0 + a_1 \alpha$ where $a_0, a_1 \in \mathbb{Z}[x]$ and $\alpha$ satisfies $\alpha^2 + 7 = 1$, or $\alpha^2 = -7$. Since $2x^2 + 7$ is in the kernel,
        \begin{gather*}
            0 = \pi(2x^2 + 7) = 2(a_0 + a_1 \alpha)^2 + 7 = 2(a_0^2 + 2a_0 a_1 \alpha + \alpha^2) + 7 \\
            \Leftrightarrow 7 - 2a_0^2 = 4a_0 a_1 \alpha;
        \end{gather*}
        this is a contradiction because $\alpha$ is not an integer. Thus, the two rings are not isomorphic.
\end{enumerate}

\subsection*{11.5}
\begin{enumerate}
    \item[1.]
        \boldmath
        \textbf{Let $f = x^4 + x^3 + x^2 + x + 1$ and let $\alpha$ denote the residue of $x$ in the ring $R = \mathbb{Z}[x]/(f)$. Express $(\alpha^3 + \alpha^2 + \alpha)(\alpha^5 + 1)$ in terms of the basis $(1, \alpha, \alpha^2, \alpha^3)$ of $R$.} \par
        \unboldmath
        Since $0 = f(\alpha) = \alpha^4 + \alpha^3 + \alpha^2 + \alpha + 1$, we have that
        \begin{align*}
            (\alpha^3 + \alpha^2 + \alpha)(\alpha^5 + 1) &= (\alpha^3 + \alpha^2 + \alpha)(\alpha^5 + 1 - \alpha f(\alpha) + f(\alpha)) \\
            &= (\alpha^3 + \alpha^2 + \alpha)(2) = 2\alpha^3 + 2\alpha^2 + 2\alpha.
        \end{align*}

    \item[2.]
        \boldmath
        \textbf{Let $a$ be an element of a ring $R$. If we adjoin an element $\alpha$ with the relation $\alpha = a$, we expect to get a ring isomorphic to $R$. Prove that this is true.} \par
        \unboldmath
        We want to show that $R$ is isomorphic to $R[x]/(x - a)$. Let $\pi : R[x] \to R[\alpha]$ denote the canonical map which is a homomorphism, and let $\varphi : R \to R[\alpha]$ be $\pi$ restricted to the constant polynomials, so it is also a homomorphism. Then $\ker \varphi = R \cap (x - a) = \{ 0 \}$, and $\varphi$ is injective. Also, since $\alpha = a \in R$ and any element of $R[\alpha]$ may be written as a linear combination of elements of the basis of $(1)$, $\varphi$ is surjective.

    \item[3.]
        \boldmath
        \textbf{Describe the ring obtained from $\mathbb{Z}/12\mathbb{Z}$ by adjoining an inverse of $2$.} \par
        \unboldmath
        This is the ring $(\mathbb{Z}/12\mathbb{Z})[x]/(2x - 1)$, and since $(\mathbb{Z}/12\mathbb{Z})[x] = \mathbb{Z}[x]/(12)$, the ring can be further simplified to $\mathbb{Z}[x]/(2x - 1, 12)$. This means we have the identities $2x - 1 = 0$, $12 = 0$. As a result, $12x = 6 \Leftrightarrow 0 = 6$, and $6x = 3 \Leftrightarrow 0 = 3$, so now we have the field $(\mathbb{Z}/3\mathbb{Z})[x] / (2x - 1)$. Because $2x - 1$ can be written as a linear combination of $x - 2$ and $3$ ($2x - 1 = 2(x - 2) + 3$), and $x - 2 = 2(2x - 1)$, this field can also be written as $(\mathbb{Z}/3\mathbb{Z})[x] / (x - 2)$, which is $\mathbb{Z}/3\mathbb{Z}$ by substituting $x = 2$.

    \item[5.]
        \boldmath
        \textbf{Are there fields $F$ such that the rings $F[x]/(x^2)$ and $F[x]/(x^2 - 1)$ are isomorphic?} \par
        \unboldmath
        Yes, if $F$ is the field $\mathbb{Z}/2\mathbb{Z}$. For if this is the case and we let $y = x - 1$, then $y^2 = (x - 1)^2 = x^2 - 2x + 1 = x^2 - 1 = 0$ in $F[x]/(x^2 - 1)$, and $y^2 - 1 = (x - 1)^2 - 1 = x^2 - 2x = x^2 = 0$ in $F[x]/(x^2)$.

    \item[6.]
        \boldmath
        \textbf{Let $a$ be an element of a ring $R$, and let $R'$ be the ring $R[x]/(ax - 1)$ obtained by adjoining an inverse of $a$ to $R$. Let $\alpha$ denote the residue of $x$ (the inverse of $a$ in $R'$).} \par
        \unboldmath
        \begin{enumerate}
            \item
                \boldmath
                \textbf{Show that every element $\beta$ of $R'$ can be written in the form $\beta = \alpha^k b$, with $b$ in $R$.} \par
                \unboldmath
                Since $\beta \in R'$, we can write
                \begin{align*}
                    \beta = a_k \alpha^k + a_{k - 1} \alpha^{k - 1} + \cdots + a_1 \alpha + a_0
                \end{align*}
                where $a_0, a_1, \cdots, a_k \in r$ and $k > 0$. Multiplying both sides by $a^k$ gives
                \begin{align*}
                    a^k \beta = a_k + a_{k - 1} a + \cdots + a_1 a^{k - 1} + a_0 a^k.
                \end{align*}
                Since the right-hand side is the polynomial
                \begin{align*}
                    p(x) = a_k + a_{k - 1} x + \cdots + a_1 x^{k - 1} + a_0 x^k.
                \end{align*}
                evaluated at $a$, we may use polynomial division by $x - a$, which gives
                \begin{align*}
                    p(x) = (x - a) q(x) + r(x)
                \end{align*}
                where the degree of $r(x)$ is zero, so $r(x) = b$ for some $b \in R$. Re-evaluating this polynomial at $a$, we have
                \begin{align*}
                    p(a) = (a - a) q(a) + b = b,
                \end{align*}
                and thus
                \begin{align*}
                    a^k \beta &= b \\
                    \Rightarrow \beta &= \alpha^k b.
                \end{align*}

            \item
                \boldmath
                \textbf{Prove that the kernel of the map $R \to R'$ is the set of elements $b$ of $R$ such that $a^n b = 0$ for some $n > 0$.} \par
                \unboldmath
                Let $b$ be an element of the kernel of the map $R \to R'$. Then $b \in (ax - 1)$ necessarily, and so $b = (ax - 1) p(x)$ for some $p(x) \in R[x]$. We may write
                \begin{align*}
                    p(x) = a_k x^k + a_{k - 1} x^{k - 1} + \cdots + a_1 x + a_0
                \end{align*}
                where $a_0, a_1, \cdots, a_k \in r$ and $k > 0$. Since $b \in R$, it is the constant term of $(ax - 1) p(x)$ (where $ax - 1$ denotes the polynomial rather than ideal), which implies that $a a_{i - 1} - a_i = 0$ for all $1 \leq i \leq k - 1$, and $a a_k = 0$. From the former, $a_i = a a_{i - 1}$. Also $b = -a_0$. Thus by induction,
                \begin{align*}
                    a_i = a^i a_0 = -a^i b,
                \end{align*}
                and in particular for $i = k$,
                \begin{align*}
                    a_k = -a^k b.
                \end{align*}
                Multiplying both sides by $a$,
                \begin{align*}
                    0 = a a_k = -a^{k + 1} b
                \end{align*}
                and so $a^{k + 1} b = 0$. $k + 1$ is the desired $n$.

            \item
                \boldmath
                \textbf{Prove that $R'$ is the zero ring if and only if $a$ is nilpotent.} \par
                \unboldmath
                ($\Leftarrow$) This means that $a^n = 0$ for some $n > 0$. Multiplying both sides by $\alpha^n$, we have that
                \begin{align*}
                    0 = a^n = \alpha^n a^n = 1^n = 1,
                \end{align*}
                and we know that this is true only for the zero ring. \par
                ($\Rightarrow$) $R'$ is necessarily of the form $R[x]/R[x]$, which means that $a$ fulfills $(ax - 1) = R[x]$ from the definition. Since $R'$ is the zero ring, the kernel of the map $R \to R'$ is all of $R$, and from part (b), for all $b$ in $R$, there exists some $n > 0$ such that $a^n b = 0$. In particular, this is true for $b = 1$, and hence $a^n = 0$ for some $n > 0$.
        \end{enumerate}
\end{enumerate}
\end{document}
