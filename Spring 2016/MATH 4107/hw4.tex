\documentclass[a4paper,12pt]{article}

\usepackage{amsfonts, amsmath, amsthm, fancyhdr}
\usepackage[margin=3.5cm]{geometry}
\allowdisplaybreaks
\pagestyle{fancy}
\rhead{Erick Lin}

\newcommand{\im}{\text{im}\,}

\begin{document}

\section*{MATH 4107 - HW4 Solutions}

\subsection*{2.7}
\begin{enumerate}
    \item[1.]
        \boldmath
        \textbf{Let $G$ be a group. Prove that the relation $a \sim b$ if $b = gag^{-1}$ for some $g$ in $G$ is an equivalence relation on $G$.} \par
        \unboldmath
        \textit{Transitivity}: If $a \sim b$ and $b \sim c$, then $b = gag^{-1}$ and $c = hbh^{-1}$ for some $g$, $h$ in $G$. It follows from substitution that $c = h gag^{-1} h^{-1} = (hg) a (hg)^{-1}$. Since $G$ is a group, $hg$ is in $G$, and hence $a \sim c$. \par
        \textit{Symmetry}: If $a \sim b$, then $b = gag^{-1}$. Using left and right multiplication, $a = g^{-1} b g$. Since $G$ is a group, $g^{-1}$ is in $G$, and hence $b \sim a$. \par
        \textit{Reflexivity}: For all $a$, $a = eae^{-1}$, and $e$ is in $G$; thus $a \sim a$.
\end{enumerate}

\subsection*{2.8}
\begin{enumerate}
    \item[1.]
        \boldmath
        \textbf{Let $H$ be the cyclic subgroup of the alternating group $A_4$ generated by the permutation $(1\ 2\ 3)$. Exhibit the left and the right cosets of $H$ explicitly.} \par
        \unboldmath
        Written explicitly,
        \begin{align*}
            A_4 = \{ (), (1\ 2)(3\ 4), (1\ 3)(2\ 4), (1\ 4)(2\ 3), (1\ 2\ 3), (1\ 3\ 2), \\
            (1\ 3\ 4), (1\ 4\ 3), (1\ 2\ 4), (1\ 4\ 2), (2\ 3\ 4), (2\ 4\ 3) \}.
        \end{align*}
        Since $(1\ 2\ 3)^2 = (1\ 3\ 2)$ and $(1\ 2\ 3)^3 = ()$, $H$ is given by
        \begin{align*}
            H = \{ (1\ 2\ 3), (1\ 3\ 2), () \}.
        \end{align*}
        The left cosets of $H$ are given by
        \begin{gather*}
            ()H = (1\ 2\ 3)H = (1\ 3\ 2)H = H \\
            (1\ 2)(3\ 4)H = (1\ 4\ 3)H = (2\ 4\ 3)H = \{ (2\ 4\ 3), (1\ 4\ 3), (1\ 2)(3\ 4) \} \\
            (1\ 3)(2\ 4)H = (1\ 4\ 2)H = (2\ 3\ 4)H = \{ (1\ 4\ 2), (2\ 3\ 4), (1\ 3)(2\ 4) \} \\
            (1\ 4)(2\ 3)H = (1\ 3\ 4)H = (1\ 2\ 4)H = \{ (1\ 3\ 4), (1\ 2\ 4), (1\ 4)(2\ 3) \},
        \end{gather*}
        while the right cosets of $H$ are given by
        \begin{gather*}
            H() = H(1\ 2\ 3) = H(1\ 3\ 2) = H \\
            H(1\ 2)(3\ 4) = H(1\ 3\ 4) = H(2\ 3\ 4) = \{ (1\ 3\ 4), (2\ 3\ 4), (1\ 2)(3\ 4) \} \\
            H(1\ 3)(2\ 4) = H(1\ 2\ 4) = H(2\ 4\ 3) = \{ (2\ 4\ 3), (1\ 2\ 4), (1\ 3)(2\ 4) \} \\
            H(1\ 4)(2\ 3) = H(1\ 4\ 3) = H(1\ 4\ 2) = \{ (1\ 4\ 2), (1\ 4\ 3), (1\ 4)(2\ 3) \}.
        \end{gather*}

    \item[2.]
        \boldmath
        \textbf{In the additive group $\mathbb{R}^m$ of vectors, let $W$ be the set of solutions of a system of homogeneous linear equations $AX = 0$. Show that the set of solutions of an inhomogeneous system $AX = B$ is either empty, or else it is an (additive) coset of $W$.} \par
        \unboldmath
        Since
        \begin{align*}
            W = \{ \vec{v} \in \mathbb{R}^m : A\vec{v} = \vec{0} \},
        \end{align*}
        $W = \ker(A)$ by the definition of a kernel. Then we know that the left cosets of $W$ are the nonempty fibers of $A$. Since the set of solutions of an inhomogeneous system for some column vector $\vec{w}$ of $B$,
        \begin{align*}
            S = \{ \vec{v} \in \mathbb{R}^m : A\vec{v} = \vec{w} \},
        \end{align*}
        is precisely a fiber of $A$, we know that $S$ is either a left coset of $W$ or empty. \par
        We also know this to be true from linear algebra, for an inhomogeneous system may have either no solutions or a particular solution $\vec{v}_1$ which leads to the set of solutions $S = \vec{v}_1 + W$.

    \item[3.]
        \boldmath
        \textbf{Does every group whose order is a power of a prime $p$ contain an element of order $p$?} \par
        \unboldmath
        Yes, if $G$ is a group whose order is of power $p^d$ for some $d \geq 2$, then by Lagrange's theorem, the order of any element $x \in G$ divides $p^d$. We may write $x^{p^k} = 1$ for some $0 \leq k \leq d$, and thus $1 = x^{p^{k - 1} p} = ( x^{p^{k - 1}} )^p$ for some $x \in G$. Then this means that $x^{p^{k - 1}}$, which is also in $G$, has order $p$.

    \item[6.]
        \boldmath
        \textbf{Let $\varphi : G \to G'$ be a group homomorphism. Suppose that $|G| = 18$, $|G'| = 15$, and that $\varphi$ is not the trivial homomorphism. What is the order of the kernel?} \par
        \unboldmath
        Since we know that $|\im \varphi|$ divides both $|G|$ and $|G'|$ and that $\im \varphi$ contains elements other than the identity element, $|\im \varphi|$ must be $3$. Also, we know that $|G| = |\ker \varphi| \cdot |\im \varphi|$, so $|\ker \varphi| = 18/3 = 6$.

    \item[9.]
        \boldmath
        \textbf{Let $G$ be a finite group. Under what circumstances is the map $\varphi : G \to G$ defined by $\varphi(x) = x^2$ an automorphism of $G$?} \par
        \unboldmath
        To satisfy the homomorphism property $\varphi(xy) = (xy)^2 = x^2 y^2 = \varphi(x) \varphi(y)$, it is necessary that $x(yx)y = x(xy)y$, or $yx = xy$ using left and right cancellation. This means that $G$ must be abelian. \par
        Additionally, injectivity implies surjectivity and vice versa for a function mapping from a set to itself. if $|G|$ is even, then by Lagrange's theorem there is an element $x \in G$ of order 2, and hence $\varphi(x) = x^2 = e$; however, $\varphi(e) = e^2 = e$ already, which means that $\varphi$ would not be injective; thus $|G|$ must be odd.

    \item[10.]
        \boldmath
        \textbf{Prove that every subgroup of index 2 is a normal subgroup, and show by example that a subgroup of index 3 need not be normal.} \par
        \unboldmath
        Let $G$ be a group and $H$ be a subgroup of index 2. This means that $H$ has 2 left cosets, which partition $G$, and 2 right cosets, which also partition $G$. Let $g \in G$. \par
        If $g \in H$, then $gH = H = Hg$, and hence $gHg^{-1} = H$. \par
        If $g \notin H$, then $gH = G \setminus H$, the other left coset which is not $H$. Also, $Hg = G \setminus H$, the other right coset which is not $H$. Hence, $gH = Hg$ so $gHg^{-1} = H$. \par
        In either case $gHg^{-1} = H$, which shows that $H$ is normal. \qed \par
        In the symmetric group $S_3$, the cyclic subgroup $\{ 1, y \}$ has order 3; the left cosets are
        \begin{gather*}
            1\{ 1, y \} = y\{ 1, y \} = \{ 1, y \} \\
            x\{ 1, y \} = xy\{ 1, y \} = \{ x, xy \} \\
            x^2\{ 1, y \} = x^2y\{ 1, y \} = \{ x^2, x^2y \}
        \end{gather*}
        and the right cosets are
        \begin{gather*}
            \{ 1, y \}1 = \{ 1, y \}y = \{ 1, y \} \\
            \{ 1, y \}x = \{ 1, y \}x^2y = \{ x, x^2y \} \\
            \{ 1, y \}x^2 = \{ 1, y \}xy = \{ x^2, xy \}.
        \end{gather*}
        It can be seen that $g \{ 1, y \} \neq \{ 1, y \} g$ when $g = xy$ or $g = x^2y$, so $g \{ 1, y \} g^{-1} \neq \{ 1, y \}$, and $\{ 1, y \}$ is not normal.

    \item[11.]
        \boldmath
        \textbf{Let $G$ and $H$ be the following subgroups of $GL_2(\mathbb{R})$:
        \begin{align*}
            G = \left\{ \left[ \begin{array}{cc}
                    x & y \\
                    0 & 1
            \end{array} \right] \right\},
            H = \left\{ \left[ \begin{array}{cc}
                    x & 0 \\
                    0 & 1
            \end{array} \right] \right\},
        \end{align*}
        with $x$ and $y$ real and $x > 0$. An element of $G$ can be represented by a point in the right half plane. Make sketches showing the partitions of the half plane into left cosets and into right cosets of $H$.} \par
        \unboldmath
        Since $G \subset H$ and $H$ is a group, $G \leq H$. The left cosets of $H$ are
        \begin{align*}
            \left[ \begin{array}{cc}
                    x & y \\
                    0 & 1
            \end{array} \right]
            \left[ \begin{array}{cc}
                    x' & 0 \\
                    0 & 1
            \end{array} \right]
            = \left[ \begin{array}{cc}
                    x'x & y \\
                    0 & 1
            \end{array} \right]
            \hspace{4cm}
        \end{align*}
        and the right cosets of $H$ are
        \begin{align*}
            \left[ \begin{array}{cc}
                    x' & 0 \\
                    0 & 1
            \end{array} \right]
            \left[ \begin{array}{cc}
                    x & y \\
                    0 & 1
            \end{array} \right]
            = \left[ \begin{array}{cc}
                    x'x & x'y \\
                    0 & 1
            \end{array} \right].
            \hspace{4cm}
        \end{align*}
\end{enumerate}

\subsection*{2.9}
\begin{enumerate}
    \item[3.]
        \boldmath
        \textbf{Prove that every integer $a$ is congruent to the sum of its decimal digits modulo 9.} \par
        \unboldmath
        Let $a$ be written in the decimal representation $a_{n - 1} 10^{n - 1} + a_{n - 2} 10^{n - 2} + \cdots + a_0 10^0$, where $n$ is the number of digits. For any integer $x$, denote $x$ modulo $9$ by $\overline{x}$. Then $\overline{10^k} = \overline{1} + \overline{9} * (\overline{10^{k - 1}} + \overline{10^{k - 2}} + \cdots + \overline{10^0}) = \overline{1}$ for any integer $k > 0$, where addition and multiplication are done modulo 9. This means that $\overline{a} = \overline{a_{n - 1}} \overline{10^{n - 1}} + \overline{a_{n - 2}} \overline{10^{n - 2}} + \cdots + \overline{a_0} \overline{10^0} = \overline{a_{n - 1}} + \overline{a_{n - 2}} + \cdots + \overline{a_0}$.
\end{enumerate}

\subsection*{Not in text:}
\begin{enumerate}
    \item
        \boldmath
        \textbf{Let $G$ be a group and let $H \leq G$ be a subgroup. Find a bijection
        \begin{gather*}
            \{ \text{left cosets of $H$ in $G$} \} \longleftrightarrow \{ \text{right cosets of $H$ in $G$} \}.
        \end{gather*}
        You may not assume that $G$, $H$, or $[G:H]$ is finite.} \par
        \unboldmath
        First, we can observe that if $ab^{-1}$ is an element of $H$, then $a = hb$ for some $h \in H$, and thus $Ha = Hb$. The converse is also true. \par
        Now let $\varphi$ be the mapping from the set of left cosets to the set of right cosets defined as
        \begin{align*}
            \varphi(xH) = Hx^{-1}
        \end{align*}
        for all $x \in G$. $\varphi$ is well-defined because if $x \in H$ and $xH = yH$, then $x^{-1} y \in H$. Since $x^{-1} y = x^{-1} \left( y^{-1} \right)^{-1}$, we have from our first observation that $Hx^{-1} = Hy^{-1}$, and thus $\varphi(xH) = \varphi(yH)$. \par
        To show that $\varphi$ is injective, let $x, y \in G$ such that $\varphi(xH) = \varphi(yH)$, so that $Hx^{-1} = Hy^{-1}$. Then we know that $x^{-1}y \in H$. Because $H$ is a group, $(x^{-1}y)^{-1} = y^{-1}x \in H$ as well. This shows that $yH = xH$, and it follows that $x = y$. \par
        Finally, to show that $\varphi$ is surjective, let $Hx$ be any right coset of $H$ in $G$. Because $G$ is a group, $x^{-1} \in G$, and hence $Hx = \varphi(x^{-1}H)$.
\end{enumerate}
\end{document}
