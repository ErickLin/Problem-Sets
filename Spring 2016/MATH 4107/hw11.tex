\documentclass[a4paper,12pt]{article}

\usepackage{amsfonts, amsmath, amssymb, amsthm, enumitem, fancyhdr, tabularx}
\usepackage[margin=3.5cm]{geometry}
\allowdisplaybreaks
\pagestyle{fancy}
\rhead{Erick Lin}

\newcommand{\im}{\text{im}\,}

\begin{document}

\section*{MATH 4107 - HW11 Solutions}

\subsection*{11.2}
\begin{enumerate}
    \item[1.]
        \boldmath
        \textbf{For which positive integers $n$ does $x^2 + x + 1$ divide $x^4 + 3x^3 + x^2 + 7x + 5$ in $[\mathbb{Z}/(n)][x]$?} \par
        \unboldmath
        If $f(x) = x^2 + x + 1$ and $g(x) = x^4 + 3x^3 + x^2 + 7x + 5$, then for any $n$, there are uniquely determined polynomials $q$ and $r$ in $[\mathbb{Z}/(n)][x]$ such that $g(x) = f(x) q(x) + r(x)$, and $r$ has degree less than that of $f$. The only $n$ for which $f$ divides $g$ in $[\mathbb{Z}/(n)][x]$ are those for which $r(x) = 0$. \par
        Using the polynomial division algorithm, we have that $q(x) = x^2 + 2x - 2$ and $r(x) = 7x + 7$. $r(x) = 0$ if and only if all its coefficients are $0$. Since both of these coefficients are $7$, whose factors are $1$ and $7$, the values of $n$ for which $r(x) = 0$ in $[\mathbb{Z}/(n)][x]$, and hence $f$ divides $g$, are $n = 1$ and $n = 7$.
\end{enumerate}

\subsection*{11.3}
\begin{enumerate}
    \item[1.]
        \boldmath
        \textbf{Prove that an ideal of a ring $R$ is a subgroup of the additive group $R^+$.} \par
        \unboldmath
        By definition, an ideal $I$ of $R$ is a nonempty subset of $R$ that is closed under addition. Also, if $s \in I$ and $r \in R$, then $rs \in I$; to see that $I$ contains the identity element, since $0 \in R$, $0s = 0 \in I$. Finally, since $R$ contains $1$ and $R^+$ is a group, $R$ contains the additive inverse $-1$; thus for any $s \in I$, $(-1)s = -s \in I$, so $I$ contains the additive inverse of $S$.

    \item[3.]
        \boldmath
        \textbf{Find generators for the kernels of the following maps:}
        \unboldmath
        \begin{enumerate}
            \item
                \boldmath
                \textbf{$\mathbb{R}[x, y] \to \mathbb{R}$ defined by $f(x, y) \rightsquigarrow f(0, 0)$.} \par
                \unboldmath
                The kernel is the set of polynomials divisible by either $x - 0 = x$ and $y - 0 = y$, and the generators are $x$ and $y$.

            \item
                \boldmath
                \textbf{$\mathbb{R}[x] \to \mathbb{C}$ defined by $f(x) \rightsquigarrow f(2 + i)$.} \par
                \unboldmath
                A polynomial in $\mathbb{R}[x]$ that has $2 + i$ as a root must also have $2 - i$ as a root by the complex conjugate root theorem. Then the kernel is the set of polynomials divisible by both $x - (2 + i)$ and $x - (2 - i)$, or
                \begin{align*}
                    (x - 2 - i)(x - 2 + i) = (x - 2)^2 - i^2 = x^2 - 4x + 5,
                \end{align*}
                and is the principal ideal generated by $x^2 - 4x + 5$.

            \item
                \boldmath
                \textbf{$\mathbb{Z}[x] \to \mathbb{R}$ defined by $f(x) \rightsquigarrow f(1 + \sqrt{2})$.} \par
                \unboldmath
                A polynomial in $\mathbb{Z}[x]$ that has $1 + \sqrt{2}$ as a root must also have $1 - \sqrt{2}$ as a root by the conjugate root theorem. Then the kernel is the set of polynomials divisible by both $x - (1 + \sqrt{2})$ and $x - (1 - \sqrt{2})$, or
                \begin{align*}
                    (x - 1 - \sqrt{2})(x - 1 + \sqrt{2}) = (x - 1)^2 - 2 = x^2 - 2x - 1,
                \end{align*}
                and is the principal ideal generated by $x^2 - 2x - 1$.

              \item
                \boldmath
                \textbf{$\mathbb{Z}[x] \to \mathbb{C}$ defined by $x \rightsquigarrow \sqrt{2} + \sqrt{3}$.} \par
                \unboldmath
                A polynomial in $\mathbb{Z}[x]$ that has $\sqrt{2} + \sqrt{3}$ as a root must also have $\sqrt{2} - \sqrt{3}$, $-\sqrt{2} + \sqrt{3}$, and $-\sqrt{2} - \sqrt{3}$ as roots by the conjugate root theorem. Then the kernel is the set of polynomials divisible by the product of the monomials $x - (\pm \sqrt{2} \pm \sqrt{3})$, or
                \begin{align*}
                    [(x - \sqrt{3})^2 - 2] [(x + \sqrt{3})^2 - 2] &= (x^2 - 2\sqrt{3}x + 1)(x^2 + 2\sqrt{3}x + 1) \\
                    &= (x^2 + 1)^2 - 12x^2 \\
                    &= x^4 - 10x^2 + 1,
                \end{align*}
                and is the principal ideal generated by $x^4 - 10x^2 + 1$.

              \item
                \boldmath
                \textbf{$\mathbb{C}[x, y, z] \to \mathbb{C}[t]$ defined by $x \rightsquigarrow t$, $y \rightsquigarrow t^2$, $z \rightsquigarrow t^3$.} \par
                \unboldmath
                This mapping sends $g(x, y, z) \rightsquigarrow g(t, t^2, t^3)$, and its kernel contains the polynomials $f_1(x, y, z) = x^3 - z$ and $f_2(x, y, z) = f_2(x, y) = x^2 - y$. We'll show that the kernel is the ideal generated by $f_1$ and $f_2$, that is, if $g(x, y, z)$ is a polynomial and $g(t, t^2, t^3) = 0$, then the linear combinations of $f_1$ and $f_2$ form $g$. \par
                \iffalse
                    First, we regard $f_1$ as a polynomial in $x$ whose coefficients are polynomials in $z$. $f_1$ is monic in $x$, and polynomial division with remainder gives $g = f_1 q_1 + r_1$, where $q_1$ and $r_1$ are polynomials and $r_1$, if nonzero, has degree at most $2$ in $x$. Hence, $r_1$ may be written $r_1(x, y, z) = \rho_5(y, z) x^2 + \rho_4(y, z) x + \rho_3(y, z)$. \par
                    If we then regard $f_2$ as a polynomial in $x$ whose coefficients are polynomials in $y$, then $f_2$ is monic in $x$, and polynomial division with remainder gives $r_1 = f_2 q_2 + r_2$, where $q_2$ and $r_2$ are polynomials and $r_2$, if nonzero, has degree at most $1$ in $x$. Hence, $r_2$ may be written $r_2(x, y, z) = \rho_1(y, z) x + \rho_0(y, z)$. \par
                If $g(t, t^2, t^3) = 0$, then both $g$ and $f_1 q_1$ are in the kernel, so $r_1$ is too, and by a similar argument $r_2$ is in the kernel as well: $r_2(t, t^2, t^3) = \rho_1(t^2, t^3) t + \rho_0(t^2, t^3) = 0$. The monomials that appear in $\rho_1(t^2, t^3) x$ must have odd degree, while those that appear in $\rho_0(t^2, t^3)$ must have even degree, so for the previous condition to be true, $\rho_0(y, z)$ and $\rho_1(y, z)$ must both be zero. Since the remainder $r_2$ is zero, $f_2$ divides $r_1$.
                \fi
                Using the isomorphisms $R[x, y, z] \cong (R[x, y])[z]$, $R[x, y] \cong (R[x])[y]$ and the division algorithm, we have
                \begin{align*}
                    g(x, y, z) = f_1(x, y, z) q_1(x, y, z) + f_2(x, y) q_2(x, y) + r(x),
                \end{align*}
                where $q_1$, $q_2$, and $r$ are polynomials.
                If $g(t, t^2, t^3) = 0$, then $g$, $f_1 q_1$, and $f_2 q_2$ are in the kernel, so $r$ is too: $r(t) = 0$. In order for this condition to hold, $r(x) = 0$, and as a result, $g$ is a linear combination of $f_1$ and $f_2$.
        \end{enumerate}

    \item[5.]
        \boldmath
        \textbf{The derivative of a polynomial $f$ with coefficients in a field $F$ is defined by the calculus formula $(a_n x^n + \cdots + a_1 x + a_0)' = na_n x^{n - 1} + \cdots + 1a_1$. The integer coefficients are interpreted in $F$ using the unique homomorphism $\mathbb{Z} \to F$.}
        \unboldmath
        \begin{enumerate}
            \item
                \boldmath
                \textbf{Prove the product rule $(fg)' = f'g + fg'$ and the chain rule $(f \circ g)' = (f' \circ g)g'$.} \par
                \unboldmath
                First we prove the product rule in the special case where $f = x^m$ and $g = x^n$ for some $m, n \geq 0$. We have that
                \begin{align*}
                    (fg)' &= (x^{m + n})' = (m + n) x^{m + n - 1} \\
                    &= (mx^{m - 1}) x^n + (nx^{n - 1}) x^m \\
                    &= f'g + fg'.
                \end{align*}
                In particular, the formula holds for $m = 0$ or $n = 0$. For $f$ and $g$ in general, since $F$ satisfies the group properties under addition and the ring properties, the derivative satisfies $(f + g)' = f' + g'$, $(af)' = af'$ for all $a$, and distributivity, which implies that the above holds for all polynomials as well. \qed \par
                In the special case that $f$ or $g$ is constant, $f \circ g$ is also constant, so $(f \circ g)' = 0$. If $f$ is constant, then $f' = f' \circ g = 0$, while if $g$ is constant, then $g' = 0$, so in summary $(f' \circ g)g' = 0$ as well. \par
                Now, if $f = x^n$ for some $n > 0$ and $g$ is an arbitrary polynomial, then we prove the chain rule by induction on $n$. For $n = 1$, $f = x$ and $f' = 1$ so $(f \circ g)' = g' = 1g' = (1 \circ g)g' = (f' \circ g)g'$. If we assume that the chain rule holds for $n = k$, then if $n = k + 1$,
                \begin{align*}
                    (f \circ g)' &= (x^{k + 1} \circ g)' = (g^{k + 1})' = (gg^k)' = g'(g^k) + g(g^k)' \\
                    &= g'(g^k) + g(x^k \circ g)' = g'(g^k) + g(kx^{k - 1} \circ g)g'\\
                    &= g'(g^k) + gkg^{k - 1}g' = (k + 1)g^k g' = [(k + 1)x^k \circ g] g' \\
                    &= [(x^{k + 1})' \circ g] g' = (f' \circ g) g'
                \end{align*}
                with the help of the product rule. Finally, due to the distributivity property, the chain rule hence holds for all polynomials $f$ and $g$.

            \item
                \boldmath
                \textbf{Let $\alpha$ be an element of $F$. Prove that $\alpha$ is a multiple root of a polynomial $f$ if and only if it is a common root of $f$ and of its derivative $f'$.} \par
                \unboldmath
                ($\Rightarrow$) If $f$ has a multiple root $\alpha$, then we may write $f = (x - \alpha)^2 g$ for some $g$. From the product rule, $f' = 2(x - \alpha)g + (x - \alpha)^2 g' = (x - \alpha)[2g + (x - \alpha)g']$, so $\alpha$ is a root of $f'$ as well. \par
                ($\Leftarrow$) Assume that $\alpha$ is a common root of $f$ and $f'$, and write $f = (x - \alpha) g$. From the product rule, $f' = 1g + (x - \alpha)g'$. Since it is given that $f'(\alpha) = 0$, solving the expression for $g$ and substituting gives $g(\alpha) = f'(\alpha) + (\alpha - \alpha)g' = 0$. $\alpha$ is a root of $g$ as well, and thus it is at least a double root of $f$.
        \end{enumerate}

    \item[8.]
        \boldmath
        \textbf{Let $R$ be a ring of prime characteristic $p$. Prove that the \textit{Frobenius map} $R \to R$ defined by $x \rightsquigarrow x^p$ is a ring homomorphism.} \par
        \unboldmath
        Using the binomial theorem,
        \begin{align*}
            (x + y)^p = \sum_{k = 0}^p \binom{n}{k} x^k y^{p - k} = x^p + y^p + \sum_{k = 1}^{p - 1} \binom{p}{k} x^k y^{p - k}
        \end{align*}
        and since the numerator of $\binom{p}{k}$ is divisible by $p$ but the denominator is not, $p | \binom{p}{k}$, and the above becomes
        \begin{align*}
            (x + y)^p = x^p + y^p.
        \end{align*}
        We also have that
        \begin{align*}
            (xy)^p = x^p y^p
        \end{align*}
        from the fact that multiplication in a ring is commutative and associative. \par
        Finally, from the definition of a multiplicative identity,
        \begin{align*}
            1^p = 1.
        \end{align*}

    \item[9.]
        \begin{enumerate}
            \item
                \boldmath
                \textbf{An element $x$ of a ring $R$ is called \textit{nilpotent} if some power is zero. Prove that if $x$ is nilpotent, then $1 + x$ is a unit.} \par
                \unboldmath
                In other words, if $x$ is nilpotent, then $x^n = 0$ for some $n > 0$. Then
                \begin{align*}
                    (1 + x)(1 - x + x^2 + \cdots + (-1)^{n - 1} x^{n - 1}) &= 1 + (-1)^{n - 1} x^n \\
                    &= 1 + 0 = 1,
                \end{align*}
                which shows that $1 - x + x^2 + \cdots + (-1)^{n - 1} x^{n - 1}$ is the multiplicative inverse of $1 + x$, and hence $1 + x$ is a unit.

            \item
                \boldmath
                \textbf{Suppose that $R$ has prime characteristic $p \neq 0$. Prove that if $a$ is nilpotent then $1 + a$ is \textit{unipotent}, that is, some power of $1 + a$ is equal to $1$.} \par
                \unboldmath
                Let $a^m = 0$ for some $m > 0$. If $m = p$, then from Exercise 8, $(1 + a)^p = 1^p + a^p = 1 + 0 = 1$. If we assume otherwise, then
                \begin{align*}
                    (1 + a)^{mp} &= \sum_{k = 0}^{mp} \binom{mp}{k} a^k = a^{mp} + \sum_{i = 0}^{m - 1} \sum_{j = 0}^{p - 1} \binom{mp}{ip + j} a^{ip + j} \\
                    &= a^{mp} + \sum_{i = 0}^{m - 1} \left[ \binom{mp}{ip} a^{ip} + \sum_{j = 1}^{p - 1} \binom{mp}{ip + j} a^{ip + j} \right] \\
                    &= 0 + \sum_{i = 0}^{m - 1} \left[ \begin{cases}
                        1,\ i = 0 \\
                        0,\ i \neq 0
                    \end{cases} + \sum_{j = 1}^{p - 1} \binom{mp}{ip + j} a^{ip + j} \right] \\
                    &= 1 + \sum_{i = 0}^{m - 1} \sum_{j = 1}^{p - 1} \binom{mp}{ip + j} a^{ip + j}.
                \end{align*}
                If $j \nmid p$, then the number of $p$'s in the numerator of $\binom{mp}{ip + j} = \frac{(mp)!}{(ip + j)!(mp - ip - j)!}$ is $m$, while the number of $p$'s in the denominator is only $i + (m - i - 1) = m - 1$, which means that $p | \binom{mp}{k}$. Therefore, the above becomes
                \begin{align*}
                    (1 + a)^{mp} = 1.
                \end{align*}
        \end{enumerate}

    \item[12.]
        \boldmath
        \textbf{Let $I$ and $J$ be ideals of a ring $R$. Prove that the set $I + J$ of elements of the form $x + y$, with $x$ in $I$ and $y$ in $J$, is an ideal. This ideal is called the \textit{sum} of the ideals $I$ and $J$.} \par
        \unboldmath
        For any elements $x_1 + y_1$, $x_2 + y_2$ in $I + J$ where $x_1, x_2 \in I$ and $y_1, y_2 \in J$,
        \begin{align*}
            (x_1 + y_1) + (x_2 + y_2) = (x_1 + x_2) + (y_1 + y_2)
        \end{align*}
        by the group properties of $R$ under addition, and since $I$ and $J$ are closed under addition by the definition of an ideal, $x_1 + x_2 \in I$ and $y_1 + y_2 \in J$. Thus $I + J$ is closed under addition. \par
        Now, let $x + y \in I + J$ where $x \in I$, $y \in J$. If $r \in R$, then we know that $rx \in I$ and $ry \in J$, so $rx + ry \in I + J$. By the distributive property of rings, $rx + ry = r(x + y)$, so $r(x + y) \in I + J$.

    \item[13.]
        \boldmath
        \textbf{Let $I$ and $J$ be ideals of a ring $R$. Prove that the intersection $I \cap J$ is an ideal. Show by example that the set of products $\{ xy \mid x \in I, y \in J \}$ need not be an ideal, but that the set $IJ$ of finite sums $\sum x_v y_v$ of products of elements of $I$ and $J$ is an ideal. This ideal is called the \textit{product ideal} of $I$ and $J$. Is there a relation between $IJ$ and $I \cap J$?} \par
        \unboldmath
        For any elements $s_1, s_2 \in I \cap J$, $s_1 + s_2 \in I$ because $I$ is closed under addition, and $s_1 + s_2 \in J$ because $J$ is closed under addition. Therefore, $s_1 + s_2 \in I \cap J$, and $I \cap J$ is closed under addition. Also, if $s \in I \cap J$ and $r \in R$, then by the definition of an ideal, $rs \in I$ and $rs \in J$, hence $rs \in I \cap J$. This shows that $I \cap J$ is an ideal. \par
        In the polynomial ring $R[a, b, c, d]$, if we let $I = (a, b)$ and $J = (c, d)$, then $ac$ and $bd$ are both contained in $\{ xy \mid x \in I, y \in J \}$, but $ac + bd$ are not. This means that the product set is not closed under addition and is not an ideal. \par
        The sum of any two finite sums of products of elements of $I$ and $J$ is also a finite sum of products of elements of $I$ and $J$, so $IJ$ is closed under addition. If $\sum x_v y_v \in IJ$ and $r \in R$, then $r \sum x_v y_v = \sum (rx_v) y_v \in IJ$ because $rx_v \in I$. \par
        If $x \in I$ and $y \in J$, then $xy \in I$ since $y \in R$, and $xy \in J$ since $x \in R$, by the definition of an ideal; thus $xy \in I \cap J$. Because $I \cap J$ is an ideal, it is closed under addition, so finite sums of products of elements of $I$ and $J$ are also contained in $I \cap J$. This shows that $IJ \subset I \cap J$.
\end{enumerate}
\end{document}
