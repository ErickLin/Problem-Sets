\documentclass[a4paper,12pt]{article}

\usepackage{amsmath}
\usepackage{amssymb}
\usepackage{bm}

\begin{document}
	
	\section*{CS 2051 - HW3 Solutions}
	
	\begin{enumerate}
		
		\item By Stirling's approximation (for large $N$),
		\[ \dbinom{N}{\alpha N} \sim e^{-N[\alpha\ln\alpha + (1 - \alpha)\ln(1 - \alpha)]} \]
		and by the Extreme Value Theorem, any potential maximum for a continuous function is located at
		\begin{gather*}
		\frac{\partial}{\partial\alpha}\dbinom{N}{\alpha N} \sim e^{-N[\alpha\ln\alpha + (1 - \alpha)\ln(1 - \alpha)]} \frac{\partial}{\partial\alpha}\left(-N[\alpha\ln\alpha + (1 - \alpha)\ln(1 - \alpha)]\right) = 0 \\
		e^{-N[\alpha\ln\alpha + (1 - \alpha)\ln(1 - \alpha)]}(-N)\left[\frac{\alpha}{\alpha} + \ln\alpha - \frac{1 - \alpha}{1 - \alpha} - \ln(1 - \alpha)\right] = 0 \\
		1 + \ln\alpha - 1 - \ln(1 - \alpha) = 0 \\
		\ln\left(\frac{\alpha}{1 - \alpha}\right) = 0 \\
		\frac{\alpha}{1 - \alpha} = 1 \\
		\alpha = \frac{1}{2}
		\end{gather*}
		Since $\frac{\partial}{\partial\alpha}\binom{N}{\alpha N} < 0$ for $\alpha > \frac{1}{2}$ and $\frac{\partial}{\partial\alpha}\binom{N}{\alpha N} > 0$ for $\alpha < \frac{1}{2}$, $\alpha = \frac{1}{2}$ has a local maximum. Evaluating the maximum for large $N$,
		\begin{align*}
		\frac{\partial}{\partial\alpha}\dbinom{N}{N / 2} &\sim e^{-N[\ln(1/2) / 2 + (1 - 1/2)\ln(1 - 1/2)]} \\
		&= e^{-N\ln(1/2)} \\
		&= \left(\frac{1}{2}\right)^{-N} \\
		&= 2^N.
		\end{align*}
		
		\item (a) Each term in the expansion of the left side can be seen as the product of $n$ groups, with each containing either $s$ or $t$. The term can have a minimum of $0$ $s$'s (if $t$ is chosen every time) and a maximum of $n$ $s$'s (if $s$ is chosen every time). Then the sum of all the terms in the expansion equals the sum of the number of possible ways to choose $i$ $s$'s leaving $n - 1$ $t$'s, for all possible $i$. \par
		(b) Through inspection, it can be seen that it is only possible to obtain the $x^{23}$ term by choosing $2$ $x^9$'s and $1$ $x^5$, leaving $97$ $1$'s. The coefficient of the $x^{23}$ is the number of ways to do this, which is
		\[ \dbinom{100}{2}\dbinom{(100 - 2)}{1}\dbinom{(100 - 2 - 1)}{97} = 485100. \]
		
		\item In the set, $500$ of the elements are congruent to $0$ $(\mbox{mod } 2)$, and $500$ of the elements are congruent to $1$ $(\mbox{mod } 2)$. In choosing $3$ elements of the set, either $1$ or $3$ of them must be congruent to $1$ $(\mbox{mod } 2)$ for their sum to be congruent to $1$ $(\mbox{mod } 2)$. Then the number of chosen elements that are congruent to $0$ $(\mbox{mod } 2)$ must be $2$ or $0$, respectively. The number of ways to do this is
		\[ \dbinom{500}{1}\dbinom{500}{2} + \dbinom{500}{3}\dbinom{500}{0} = 83083500. \]
		
		\item The prime factorization of $4200$ is $2^3 \cdot 3^1 \cdot 5^2 \cdot 7^1$. The number of divisors is the number of ways to choose a variable power between $0$ and the multiplicity for each distinct prime factor, which is
		\[ (3 + 1)(1 + 1)(2 + 1)(1 + 1) = 48. \]
		The number of divisors shared between $4200$ and $462$ is the number of divisors of their GCD. Since the prime factorization of $462$ is $2^1 \cdot 3^1 \cdot 7^1 \cdot 11^1$, the GCD is $2^1 \cdot 3^1 \cdot 7^1$, and its number of divisors is
		\[ (1 + 1)(1 + 1)(1 + 1) = 8. \]
		Then the number of divisors of $4200$ that do not divide $462$ is
		\[ 48 - 8 = 40. \]
		
		\item (a) First, consider the binomial expansion of the derivative of $(1 + x)^n$, which is
		\[ n(1 + x)^{n - 1} = n\sum_{i = 0}^{n - 1}\left[\dbinom{n - 1}{i} 1^i x^{n - i}\right]. \]
		For $x = 1$ and $k = i + 1$, the quantity becomes
		\begin{align*}
		n(2)^{n - 1} &= n\sum_{k = 1}^{n}\dbinom{n - 1}{k - 1}. \\
		&= \sum_{k = 1}^{n}\frac{n(n - 1)!}{(k - 1)!(n - k)!} \\
		&= \sum_{k = 1}^{n}\left[k\frac{n!}{k!(n - k)!}\right] \\
		&= \sum_{k = 1}^{n}\left[k\dbinom{n}{k}\right]
		\end{align*}
		(b) Consider the number of ways to choose $n$ elements from a set of $2n$ elements. This can also be formulated as the number of ways to choose $k <= n$ elements from the first $n$ elements of the set and $n - k$ elements from the last $n$ elements of the set over all possible $k$. Simplifying the equality:
		\begin{align*}
		\dbinom{2n}{n} &= \sum_{k = 0}^{n}\left[\dbinom{n}{k}\dbinom{n}{n - k}\right]  \\
		&= \sum_{k = 0}^{n}\dbinom{n}{k}^2.
		\end{align*}
		
		\item This is equivalent to choosing $k$ objects from the distinguishable set over all possible $k$. Then the expression is
		\[ \sum_{k = 0}^{n}C(n, k) \]
		
	\end{enumerate}
	
\end{document}