First, consider the case in which the added line is parallel to one or more existing lines. These parallel lines divide the plane into a number of "super-regions." On one side of the added line, imagine "flipping" every super-region, or flipping the color of all the regions contained inside the super-region. This allows the statement to hold. \par
Now, consider the remaining case in which the added line intersects every existing line, and imagine following the added line in a predetermined direction from one end to the other. Consider the first existing line that it intersects - the added line separates the two regions on either side of the existing line into four regions. Then one region on either side of the existing line can be flipped to the opposite color, as long as the two flipped regions are adjacent to one another. \par
If the added line is parallel to any of the existing lines, then the two regions that are chosen to be flipped have an additional constraint - they must be on the side of the added line that is closest to the nearest existing parallel line. \par
Now consider the additional line in relation to each remaining existing line that it intersects. If the region on the opposite side of the added line from the previously flipped region is flipped at every new intersection, then the statement is guaranteed to hold once the line has passed through all the intersections. This completes the proof.