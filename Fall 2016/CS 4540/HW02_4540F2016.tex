\documentclass[a4paper,11pt]{article}
\usepackage{amsfonts, amsmath, enumitem}

\textheight=9.5in
\textwidth=6.5in
\topmargin=-.8in
\headsep=0pt
\oddsidemargin=0truecm
\evensidemargin=0truecm
\footskip=20pt
%\footheight=20pt
\pretolerance=1600
\tolerance=1600
\hbadness=1600

\newtheorem{theorem}{Theorem}[section]
\newtheorem{corollary}[theorem]{Corollary}
\newtheorem{lemma}[theorem]{Lemma}
\newtheorem{proposition}[theorem]{Proposition}
\newtheorem{conjecture}[theorem]{Conjecture}
\newtheorem{problem}[theorem]{Problem}

\DeclareMathOperator*{\argmax}{argmax}

\begin{document}

\title{
}

\date{}

%%%%%%%%%%%%%%%%%%%%%%%%%%%%%%%%%%%%%%%%%%%%%%%%%%%%%%%%
% Author's definitions

\newcommand{\DEF}[1]{{\em #1\/}}

\newcommand\chic{\chi_c}
\newcommand\C{\hbox{${\cal C}$}}
\newcommand{\RR}{\mbox{$\mathbb R$}}
\newcommand{\NN}{\mbox{$\mathbb N$}}
\newcommand{\ZZ}{\mbox{$\mathbb Z$}}
\newcommand{\eopf}{\raisebox{0.8ex}{\framebox{}}}
\newcommand{\dist}{\hbox{\rm d}}
\renewcommand\a{\alpha}
\renewcommand\b{\beta}
\renewcommand\c{\gamma}
\renewcommand\d{\delta}
\newcommand\D{\Delta}
\newcommand{\directedchi}{\mbox{$\vec{\chi}$}}
\newcommand{\directedE}{\mbox{$\vec{E}$}}
\newcommand{\directedG}{\mbox{$\vec{G}$}}
\newcommand{\directedK}{\mbox{$\vec{K}$}}

\newenvironment{proof}%
{\noindent{\bf Proof.}\ }%
{\hfill\eopf\par\bigskip}%

%%%%%%%%%%%%%%%%%%%%%%%%%%%%%%%%%%%%%%%%%%%%%%%%%%%%%%%%

%\maketitle

\noindent{\bf  Last Name:} $Lin \quad$
\noindent{\bf First Name:} $Erick \quad$
\noindent{\bf Email:}          $elin42@gatech.edu$\\
\noindent{\bf CS 4540, Fall 2016, Homework 2, 9/15/16 Due 9/22/16 in class $~~~~~~~$ Page 1/5}

\bigskip


\noindent{\bf Problem 1:  NP-complete special cases (20 points)}.  \\
In class we say a polynomial time dynamic programming algorithm for finding a maximum weight independent set
for an undirected graph $G(V,E)$ with positive weights on its vertices $c(v) > 0$, for all $v\in V$, for the case where 
$G$ is a tree (also in DPV textbook.) \\
(a) Give a polynomial time algorithm for finding a maximum weight independent set
for an undirected graph $G(V,E)$ with positive weights on its vertices $c(v) > 0$, for all $v\in V$, for the case where 
$G$ is a cycle. That is: $V= \{ v_1 , \ldots v_n \}$ and $E = \{  \{ v_1 , v_2  \} , \{  v_2 , v_3 \} , \ldots \{  v_{n-1} , v_n\}  \}\cup \{ \{ v_n , v_1\} \}$. \\
(b) Give a polynomial time algorithm for finding a maximum weight independent set
for an undirected graph $G(V,E)$ with positive weights on its  for the case where 
$G$ is a Theta. That is: 
$V= \{ x_1 , \ldots x_n \} \cup  \{ y_1 , \ldots y_n \} \cup \{ a , b\}$ 
and $E = \{  \{ x_1 , x_2  \} , \{  x_2 , x_3 \} , \ldots \{  x_{n-1} , x_n\}     \}
\cup \{  \{ y_1 , y_2  \} , \{  y_2 , y_3 \} , \ldots \{  y_{n-1} , y_n\}     \} 
\cup \{ \{ a, x_1 \} \}  \cup \{ \{ a, y_1 \} \}  \cup \{ \{ b, x_n\} \}  \cup \{ \{ b , y_n\} \} 
$. \\

\noindent
{\bf Answer:} 
\begin{enumerate}[label=(\alph*)]
    \item
        Let
        \begin{itemize}
            \item
                $G_i(\{ v_1, \cdots, v_i \}, \{ \{ v_1, v_2 \}, \cdots, \{ v_{i - 1}, v_i \} \})$ be the subgraph of $G(V, E)$,
            \item
                $opt(i)$ be the maximum-weight independent set on $G_i$,
            \item
                $opt(n + 1)$ be the maximum-weight independent set on $G$, the object of output.
        \end{itemize}
        $opt(1)$ is trivially $v_1$, and $opt(2)$ is $\argmax_{i \in [1, 2]} c(v_i)$. Then the recurrence relation for $3 \leq i \leq n + 1$ is given by
        \begin{align*}
            opt(i) = \argmax_{S \in \{ opt(i - 1), opt(i - 2) \cup v_i \}} \sum_{v_j \in S} c(v_j)
        \end{align*}
        because if the maximum-weight independent set for $G_i$ does not contain $v_{i - 1}$, which is the only vertex adjacent to $v_i$, then it must contain $v_i$ because $c(v_i)$ contributes a positive value to the weight. Additionally, if $opt(i - 1)$ does not contain $v_{i - 1}$, then it is the same as $opt(i - 2)$, and $opt(i - 2) \cup v_i$ would be the argument chosen for $S$. \par
        Computing $opt(i)$ for all $i$ in order while keeping a running sum of the weights of the vertices in $opt(i)$ takes $O(n)$ time, because only one addition and one comparison is made at each iteration.

    \item
        The algorithm from part (a) also gives the maximum-weight independent set on any linear graph $G_n$ by $opt(n)$. \par
        Thus, we have a linear-time subroutine for finding the maximum-weight independent sets (which we denote by $opt(G')$ for any graph $G'$ hereafter) on the subgraphs induced by $\{ x_1, \cdots, x_n \}$, $\{ y_1, \cdots, y_n \}$, $\{ x_2, \cdots, x_n \}$, $\{ y_2, \cdots, y_n \}$, $\{ x_1, \cdots, x_{n - 1} \}$, and $\{ y_1, \cdots, y_{n - 1} \}$. If the maximum-weight independent set on $G$
        \begin{itemize}
            \item
                contains neither $a$ nor $b$, then it is given by $opt(\{ x_1, \cdots, x_n \}) \cup opt(\{ y_1, \cdots, y_n \})$.
            \item
                contains $a$ and necessarily not $b$, then it is given by $a \cup opt(\{ x_2, \cdots, x_n \}) \cup opt(\{ y_2, \cdots, y_n \})$.
            \item
                contains $b$ and necessarily not $a$, then it is given by $opt(\{ x_1, \cdots, x_{n - 1} \}) \cup opt(\{ y_1, \cdots, y_{n - 1} \}) \cup b$.
        \end{itemize}
        Out of these three, we take the one with the maximum sum of weights. The algorithm consists of a constant number of $O(n)$-time subroutines, so it runs in $O(n)$ time.
\end{enumerate}


\pagebreak 

\noindent{\bf  Last Name:} $Lin \quad$
\noindent{\bf First Name:} $Erick \quad$
\noindent{\bf Email:}          $elin42@gatech.edu$\\
\noindent{\bf CS 4540, Fall 2016, Homework 2, 9/15/16 Due 9/22/16 in class $~~~~~~~$ Page 2/5}

\bigskip


\noindent{\bf Problem 2:  Knapsack Algorithm ? (20 points)}.  \\
Consider the following greedy algorithm for the knapsack problem (without repetition.)
Sort the items by decreasing ratio of profit to size,
and then greedily pick items in this order, while capacity is not violated. 
Show that this algorithm can be made to perform arbitrarily bad.\\

\noindent
{\bf Answer:}
\iffalse
    Imagine a scenario with a knapsack of capacity $c$, $n$ items of size $s$ and profit $p$ where $c = ns$ and $s \mid p$, and $1$ item of size $c - s + 1$ and profit $(c - s + 1)p/s$. \par
    Then the last item has the highest profit-to-size ratio, so it would be chosen first, leaving no room in the knapsack for other items to be added -- the total profit of the knapsack would then be $(c - s + 1)p/s = (c + 1)p/s - 1$. \par
    However, the maximum total profit possible is $np = cp/s$, which is made by adding the $n$ items of size $s$.
\fi
Imagine a scenario with a knapsack of capacity $c$ and two items, one of size $c$ and profit $c$ and the other of size $1$ and profit $2$. \par
The second item has the highest profit-to-size ratio, so it would be chosen first, leaving no room in the knapsack for other items to be added -- the total profit of the knapsack would then be $2$. \par
However, the maximum total profit possible is $c$, which is made by adding the first item. As $c$ gets arbitrarily large, the error $c - 2$ also increases arbitrarily.


\pagebreak 

\noindent{\bf  Last Name:} $Lin \quad$
\noindent{\bf First Name:} $Erick \quad$
\noindent{\bf Email:}          $elin42@gatech.edu$\\
\noindent{\bf CS 4540, Fall 2016, Homework 2, 9/15/16 Due 9/22/16 in class $~~~~~~~$ Page 3/5}

\bigskip


\noindent{\bf Problem 3:  Knapsack 2 Approximation  (20 points)}.  \\
Consider the following modification to the algorithm given in Problem 2. 
Let the sorted order of items be $a_1 , a_2 , \ldots , a_n$. 
Find the lowest $k$ such that the size of the first $k$ items exceeds the capacity. 
Now pick the more profitable of the two sets
$\{ a_1 , \ldots, a_{k-1} \}$ and $\{ a_k \}$ (we have assumed that each item's size is smaller than the total capacity.) 
Show that, if the profit of the optimal solution is OPT, then the profit of the picked set is at least OPT/2. \\

\noindent
{\bf Answer:}
Assume otherwise, that the profit of the chosen set is less than OPT/2. The other set is even less profitable, so the combined profit of $\{ a_1, \cdots, a_{k - 1} \}$ and $\{ a_k \}$ is less than OPT/2 + OPT/2 = OPT, and by definition the combined size of items in this union exceeds the capacity. Some other set whose combined size of items is at most the capacity must give a profit of OPT, but then this set has a higher profit-to-size ratio, contradicting the fact that the items are indexed in decreasing order of their profit-to-size ratios.


\pagebreak 

\noindent{\bf  Last Name:} $Lin \quad$
\noindent{\bf First Name:} $Erick \quad$
\noindent{\bf Email:}          $elin42@gatech.edu$\\
\noindent{\bf CS 4540, Fall 2016, Homework 2, 9/15/16 Due 9/22/16 in class $~~~~~~~$ Page 4/5}

\bigskip
\noindent{\bf Problem 4:  2Partition,  Pseudopolynomial (20 points)}.  \\
2Partition is the following problem: Input $n$ distinct positive integers $a_1, \ldots , a_n$ (written in binary.)
Is there a subset $S \subset \{ 1 , 2 , \ldots , n \}$ such that $\sum_{i\in S} a_i  = \sum_{i \not\in S} a_i$? \\
This is a well known NP-complete problem. \\
Give a pseudopolynomial algorithm to solve 2Partition; i.e., your algorithm should run in time polynomial 
in $n$ and $A$, where $A=\sum_{i=1}^{n} a_i$.\\

\noindent
{\bf Answer:}
The approach is to mark all the numbers from $0$ to $A$ that are the sum of any subset of the given integers by going through the integers $a_i$ one at a time. \par
We know that $0$ is trivally the sum of the empty subset, so it is marked initially. Then for all integers $a_i$ in order, we iterate over all numbers $m$, from $A - a_i$ to $0$, in decreasing order. If $m$ is marked, then we know that it is the sum of some subset of $\{ a_1, \cdots, a_{i - 1} \}$, and the sum of the union of this subset with $\{ a_i \}$ is $m + a_i$, so we can mark $m + a_i$. \par
At the end of this procedure, all possible sums that can be obtained from some subset are marked. In particular, we know that there is a 2Partition if $A/2$ is marked (this means $A$ must be even). \par
We can see that the procedure runs in $O(\sum_{i = 1}^n (A - a_i)) = O(An - A) = O(An)$ time.


\pagebreak

\noindent{\bf  Last Name:} $Lin \quad$
\noindent{\bf First Name:} $Erick \quad$
\noindent{\bf Email:}          $elin42@gatech.edu$\\
\noindent{\bf CS 4540, Fall 2016, Homework 2, 9/15/16 Due 9/22/16 in class $~~~~~~~$ Page 5/5}

\bigskip
\noindent{\bf Problem 5:  Knapsack with Repetition, Fully Polynomial Approximation (20 points)}.  \\
Consider the knapsack problem for $n$ items, with capacity $W$ and maximum item profit $P$.\\
For knapsack without repetition, we saw an algorithm that produces a solution with profit $(1- \epsilon )$OPT, 
and runs in time polynomial in $n$, $\log P$, $\log W$ and $\frac{1}{\epsilon}$, 
where OPT is the profit of an optimal solution. \\
Consider knapsack with repetition; i.e., each item can be picked multiple times, 
but the capacity of valid solutions is still bounded by $W$. 
Give an algorithm that produces a solution with profit $(1- \epsilon )$OPT, 
and runs in time polynomial in $n$, $\log P$, $\log W$ and $\frac{1}{\epsilon}$, 
where OPT is the profit of an optimal solution. \\

\noindent
{\bf Answer:}
\iffalse
    One algorithm for knapsack with repetition is given by computing the maximum profit for each size. Each item can potentially be added up to $W$ times (depending on the size of the item) instead of just once, so the running time is $O(nW^2)$. \par
    We now present a fully polynomial approximation scheme for knapsack with repetition:
    \begin{enumerate}[itemsep=-0.5ex]
        \item
            Given $\epsilon > 0$, let $K = \frac{\epsilon W}{n}$.
        \item
            Define $W' = \left\lfloor \frac{W}{K} \right\rfloor$, and for each item $a$, define $\text{size}'(a) = \left\lfloor \frac{\text{size}(a)}{K} \right\rfloor$.
            %For each item $a$, define $\text{profit}'(a) = \left\lfloor \frac{\text{profit}(a)}{K} \right\rfloor$.
        \item
            With these as sizes of items, use the dynamic programming algorithm given above to find the most profitable set, say $S'$.
        \item
            Output $S'$.
    \end{enumerate}
    We show that $S'$ satisfies $\text{profit}(S') \geq (1 - \epsilon) \cdot \text{OPT}$ as follows. Let $O$ denote the optimal set. For any item $a$, $0 \leq \text{size}(a) - K \cdot \text{size}'(a) \leq K$, so
    \begin{align*}
        0 \leq \text{size}(O) - K \cdot \text{size}'(O) \leq nK.
    \end{align*}
    The dynamic programming subroutine must return a set at least as good as $O$ under $\text{size}'$. Thus,
    \begin{align*}
        \text{profit}(S') &\geq K \cdot \text{profit}'(S') \geq K \cdot \text{profit}'(O) \\
        &\geq \text{profit}(O) - nK = \text{OPT} - \epsilon P \\
        &\geq (1 - \epsilon) \cdot \text{OPT}
    \end{align*}
    where the last step comes from the fact that $P$ is a lower bound on $\text{OPT}$.
    The running time of the algorithm is $O(n \lfloor \frac{W}{K} \rfloor^2) = $.
\fi
One observation that can be made is that each item will appear in the knapsack at most $W$ times, since the size of each item is at least $1$. If we replace each item $a$ with $\log_2 W$ items $a_1, a_2, \cdots, a_{\log_2 W}$ where the size and profit of $a_i$ are respectively $2^i$ times the size and profit of $a$, then any number of repetitions of $a$ in the knapsack can be simulated by some combination of the $a_i$ because any integer can be represented as the sum of different powers of $2$. \par
Using the $n \log_2 W$ new items, we can then reduce this problem to the knapsack problem without repetition in which the capacity is $W$ and the maximum item profit is $PW$. \par
Substituting these values into the known time complexity of the fully polynomial approximation scheme for knapsack without repetition gives a time complexity of $O((n \log_2 W)^2 \lfloor n \log_2 W/\epsilon \rfloor) = O((n \log W)^2 \lfloor n \log W/\epsilon \rfloor)$, which is polynomial in the quantities required.


\end{document}
