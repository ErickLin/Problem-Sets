\documentclass[a4paper,12pt]{article}

\usepackage{amsfonts, amsmath, amssymb, amsthm, enumitem, fancyhdr, mathtools}
\usepackage[margin=3.5cm]{geometry}
\allowdisplaybreaks
\pagestyle{fancy}
\rhead{Erick Lin}

\renewcommand{\thesubsection}{\arabic{subsection}}
\DeclarePairedDelimiterX{\norm}[1]{\lVert}{\rVert}{#1}
\newcommand{\bs}{\boldsymbol}
\newtheorem{theorem}{Theorem}
\newtheorem{lemma}[theorem]{Lemma}
\theoremstyle{remark}
\newtheorem{remark}{Remark}

\begin{document}

\section*{MATH 4441 -- HW3 Solutions}
\begin{enumerate}
    \item[2.]
        \iffalse
            We know that the formula for a curve in any number of dimensions is given by
            \begin{align*}
                \kappa(t) &= \norm*{\left( \frac{\bs{\alpha}'(t)}{\norm{\bs{\alpha}'(t)}} \right)' \frac{1}{\norm{\bs{\alpha}'(t)}}} \\
                &= \norm*{\left( \frac{\norm{\bs{\alpha}'(t)} \bs{\alpha}''(t) - \bs{\alpha}'(t) \norm{\bs{\alpha}'(t)}'}{\norm{\bs{\alpha}'(t)}^2} \right) \frac{1}{\norm{\bs{\alpha}'(t)}}} \\
                &= \frac{\norm*{\norm{\bs{\alpha}'(t)} \bs{\alpha}''(t) - \bs{\alpha}'(t) \norm{\bs{\alpha}'(t)}'}}{\norm{\bs{\alpha}'(t)}^3}
            \end{align*}
            What remains to be shown is that the numerator becomes $\norm{\bs{\alpha}'(t) \times \bs{\alpha}''(t)}$ in 3 dimensions:
            \begin{align*}
                %\norm*{\norm{\bs{\alpha}'(t)} \bs{\alpha}''(t) - \bs{\alpha}'(t) \norm{\bs{\alpha}'(t)}'}
            \end{align*}
        \fi
        First, we have
        \begin{align*}
            \bs{\alpha}'(s) = \bs{T}(s) = \frac{\bs{\alpha}'(t)}{\norm{\bs{\alpha}'(t)}}
        \end{align*}
        and using the chain rule,
        \begin{align*}
            \bs{\alpha}''(s) = \frac{1}{\norm{\bs{\alpha}'(t)}} \left[ \frac{\norm{\bs{\alpha}'(t)} \bs{\alpha}''(t) + \frac{\bs{\alpha}'(t)}{\norm{\bs{\alpha}'(t)}^2} \left( \frac{\bs{\alpha}'(t) \cdot \bs{\alpha}''(t)}{\norm{\bs{\alpha}'(t)}} \right)}{\norm{\bs{\alpha}'(t)}} \right].
        \end{align*}
        The chain rule also gives
        \begin{align*}
            \bs{\alpha}'(t) \times \bs{\alpha}''(s) = \frac{\bs{\alpha}'(t) \times \bs{\alpha}''(t)}{\norm{\bs{\alpha}'(t)}^2}.
        \end{align*}
        Since $\bs{\alpha}''(s) = \kappa(s) \bs{N}(s)$ and $\norm{\bs{\alpha}'(t) \times \bs{N}(s)} = \norm{\bs{\alpha}'(t)}$ (since $\bs{\alpha}'(t)$ is in the same direction as $\bs{\alpha}'(s) = \bs{T}(s)$, which is perpendicular to $\bs{N}(s)$), taking the norm of and dividing both sides gives
        \begin{align*}
            \kappa(s) = \frac{\norm{\bs{\alpha}'(t) \times \bs{\alpha}''(t)}}{\norm{\bs{\alpha}'(t)}^3}.
        \end{align*}

    \item[4.]
        We parameterize the curve by arc length, first by computing
        \begin{align*}
            \bs{\alpha}'(t) &= (e^t \cos t - e^t \sin t, e^t \sin t + e^t \cos t, e^t) \\
            &= e^t(\cos t - \sin t, \sin t + \cos t, 1) \\
            \norm{\bs{\alpha}'(t)} &= e^t \sqrt{(\cos t - \sin t)^2 + (\sin t + \cos t)^2 + 1} \\
            &= \sqrt{3}e^t.
            %\bs{\alpha}''(t) &= (-2e^t \sin t, 2e^t \cos t, e^t) \\
            %&= e^t(-2\sin t, 2\cos t, 1) \\
            %\norm{\bs{\alpha}''(t)} &= e^t \sqrt{4\sin^2 t + 4\cos^2 t + 1} \\
            %&= \sqrt{5}e^t.
        \end{align*}
        Then if $f(t) = \int_{-\infty}^t \norm{\bs{\alpha}'(x)} dx = \sqrt{3} e^t$ and $g(s) = f^{-1}(s) = \ln \frac{s}{\sqrt{3}}$ (we have the substitutions $s = f(t)$ and $t = g(s)$),
        \begin{align*}
            \bs{\beta}(s) = \bs{\alpha}(g(s)) = \frac{s}{\sqrt{3}} (\cos\ln\frac{s}{\sqrt{3}}, \sin\ln\frac{s}{\sqrt{3}}, 1)
        \end{align*}
        is a reparameterization of $\bs{\alpha}$ by arc length. \par
        Of the curve, the unit tangent vector is given by
        \begin{align*}
            \bs{T}(t) = \frac{\bs{\alpha}'(t)}{\norm{\bs{\alpha}'(t)}} = \frac{1}{\sqrt{3}} (\cos t - \sin t, \sin t + \cos t, 1)
        \end{align*}
        or
        \begin{align*}
            \bs{T}(s) = \bs{\beta}'(s) = \frac{1}{\sqrt{3}} (\cos\ln\frac{s}{\sqrt{3}} - \sin\ln\frac{s}{\sqrt{3}}, \sin\ln\frac{s}{\sqrt{3}} + \cos\ln\frac{s}{\sqrt{3}}, 1),
        \end{align*}
        the unit normal vector is given by
        \begin{align*}
            \bs{N}(s) = \frac{\bs{T}'(s)}{\norm{\bs{T}'(s)}} = \frac{1}{\sqrt{3}s} (-\sin\ln\frac{s}{\sqrt{3}} - \cos\ln\frac{s}{\sqrt{3}}, \cos\ln\frac{s}{\sqrt{3}} - \sin\ln\frac{s}{\sqrt{3}}, 0),
        \end{align*}
        the binormal vector is given by
        \begin{align*}
            \bs{B}(s) &= \bs{T}(s) \times \bs{N}(s) \\
            &= \frac{1}{3s}(\sin\ln\frac{s}{\sqrt{3}} - \cos\ln\frac{s}{\sqrt{3}}, \sin\ln\frac{s}{\sqrt{3}} + \cos\ln\frac{s}{\sqrt{3}}, 2),
        \end{align*}
        the curvature is given by
        \begin{align*}
            \kappa(s) = \norm{\bs{T}'(s)} = 1,
        \end{align*}
        and the torsion is given by
        \begin{align*}
            \tau(s) &= -\bs{B}'(s) \cdot \bs{N}(s) \\
            &= -\frac{2}{3s^2} ( \cos\ln\frac{s}{\sqrt{3}} + \sin\ln\frac{s}{\sqrt{3}}, \cos\ln\frac{s}{\sqrt{3}} + \sin\ln\frac{s}{\sqrt{3}}, -1 ) \cdot \bs{N}(s) \\
            &= \frac{2}{3\sqrt{3}s^3} (1 + 2\sin\ln\frac{s}{\sqrt{3}}\cos\ln\frac{s}{\sqrt{3}}, 1 - 2\sin\ln\frac{s}{\sqrt{3}}\cos\ln\frac{s}{\sqrt{3}}, 0).
        \end{align*}

    \item[5.]
        Letting $\bs{\omega}(s) = a(s)\bs{T}(s) + b(s)\bs{N}(s) + c(s)\bs{B}(s)$ for some $a$, $b$, $c$, taking the cross products yields
        \begin{align*}
            \bs{\omega}(s) \times \bs{T}(s) &= c(s)\bs{N}(s) - b(s)\bs{B}(s) \\
            \bs{\omega}(s) \times \bs{N}(s) &= -c(s)\bs{T}(s) + a(s)\bs{B}(s) \\
            \bs{\omega}(s) \times \bs{B}(s) &= b(s)\bs{T}(s) - a(s)\bs{N}(s).
        \end{align*}
        The Frenet formulas simultaneously require that
        \begin{align*}
            \bs{T}'(s) &= \kappa(s)\bs{N}(s) \\
            \bs{N}'(s) &= -\kappa(s)\bs{T}(s) + \tau(s)\bs{B}(s) \\
            \bs{B}'(s) &= -\tau(s)\bs{N}(s).
        \end{align*}
        If we take $a = \tau$, $b = 0$, and $c = \kappa$, then we have
        \begin{align*}
            \bs{T}'(s) &= \bs{\omega}(s) \times \bs{T}(s) \\
            \bs{N}'(s) &= \bs{\omega}(s) \times \bs{N}(s) \\
            \bs{B}'(s) &= \bs{\omega}(s) \times \bs{B}(s)
        \end{align*}
        as desired.

    \item[8.]
        A parameterization $\bs{\alpha}$ for a sphere of center $\bs{p}$ and radius $R$ is given by
        \begin{align*}
            \norm{\bs{\alpha}(s) - \bs{p}} = R \Leftrightarrow (\bs{\alpha}(s) - \bs{p}) \cdot (\bs{\alpha}(s) - \bs{p}) = R.
        \end{align*}
        Differentiating both sides, we have, using the product rule, that
        \begin{align}
            \bs{\alpha}'(s) \cdot (\bs{\alpha}(s) - \bs{p}) = 0 \Leftrightarrow \bs{T}(s) \cdot (\bs{\alpha}(s) - \bs{p}) = 0. \label{eq:Tdot}
        \end{align}
        Differentiating again, we obtain
        \begin{gather}
            \bs{\alpha}''(s) \cdot (\bs{\alpha}(s) - \bs{p}) + \norm{\bs{\alpha}'(s)}^2 = 0 \nonumber \\
            \Leftrightarrow \bs{T}'(s) \cdot (\bs{\alpha}(s) - \bs{p}) + 1 = 0 \nonumber \\
            \Leftrightarrow \kappa(s) \bs{N}(s) \cdot (\bs{\alpha}(s) - \bs{p}) = -1, \label{eq:Ndot}
        \end{gather}
        and differentiating a third time along with using (\ref{eq:Tdot}) from above gives
        \begin{gather}
            %[\kappa'(s) \bs{N}(s) + \kappa(s) \bs{N}'(s)] \cdot (\bs{\alpha}(s) - \bs{p}) + \bs{T}'(s) \cdot \bs{\alpha}'(s) = 0 \\
            %\Leftrightarrow [\kappa'(s) \bs{N}(s) + \kappa(s) (-\kappa(s)\bs{T}(s) + \tau(s)\bs{B}(s))] \cdot (\bs{\alpha}(s) - \bs{p}) = 0
            \kappa'(s)\bs{N}(s) \cdot (\bs{\alpha}(s) - \bs{p}) + \kappa(s)[\bs{N}'(s) \cdot (\bs{\alpha}(s) - \bs{p}) + \bs{N}(s) \cdot \bs{T}(s)] = 0 \nonumber \\
            \Leftrightarrow \kappa'(s)\bs{N}(s) \cdot (\bs{\alpha}(s) - \bs{p}) + [-\kappa^2(s)\bs{T}(s) + \kappa(s)\tau(s)\bs{B}(s)] \cdot (\bs{\alpha}(s) - \bs{p}) = 0 \nonumber \\
            \Leftrightarrow \kappa'(s)\bs{N}(s) \cdot (\bs{\alpha}(s) - \bs{p}) + \kappa(s)\tau(s)\bs{B}(s) \cdot (\bs{\alpha}(s) - \bs{p}) = 0 \nonumber \\
            \Leftrightarrow -\frac{\kappa'(s)}{\kappa(s)} + \kappa(s)\tau(s)\bs{B}(s) \cdot (\bs{\alpha}(s) - \bs{p}) = 0 \nonumber \\
            \Leftrightarrow \bs{B}(s) \cdot (\bs{\alpha}(s) - \bs{p}) = \frac{\kappa'(s)}{\kappa^2(s)\tau(s)}. \label{eq:Bdot}
        \end{gather}
        Differentiating a last time along with using (\ref{eq:Ndot}) gives the result
        \begin{gather*}
            \bs{B}'(s) \cdot (\bs{\alpha}(s) - \bs{p}) + \bs{B}(s) \cdot \bs{T}(s) = \left( \frac{\kappa'(s)}{\kappa^2(s)\tau(s)} \right)' \\
            -\tau(s)\bs{N}(s) \cdot (\bs{\alpha}(s) - \bs{p}) = \left( \frac{\kappa'(s)}{\kappa^2(s)\tau(s)} \right)' \\
            \frac{\tau(s)}{\kappa(s)} = \left( \frac{\kappa'(s)}{\kappa^2(s)\tau(s)} \right)'.
        \end{gather*}

    \item[10.]
        \begin{enumerate}
            \item
                $D(0) = \norm{\bs{T}_{\bs{\alpha}}(0)}^2 + \norm{\bs{N}_{\bs{\alpha}}(0)}^2 + \norm{\bs{B}_{\bs{\alpha}}(0)}^2 = 1 + 1 + 1 = 3$.
            \item
                Without loss of generality, for normal vectors $\bs{T}_{\bs{\alpha}}(s)$ and $\bs{T}_{\bs{\beta}}(s)$, $\bs{T}_{\bs{\alpha}}(s) \cdot \bs{T}_{\bs{\beta}}(s)$ takes on its maximum value $1$ if and only if $\bs{T}_{\bs{\alpha}}(s) = \bs{T}_{\bs{\beta}}(s)$. Then if the maximum sum is $3$, the same must be true for all three corresponding pairs of Frenet frame vectors.
            \item
                Using the product rule and substituting the Frenet formulas for all the derivatives of the normal vectors,
                \begin{align*}
                    D'(s) &= \bs{T}_{\bs{\alpha}}'(s) \bs{T}_{\bs{\beta}}(s) + \bs{T}_{\bs{\alpha}}(s) \bs{T}_{\bs{\beta}}'(s) + \bs{N}_{\bs{\alpha}}'(s) \bs{N}_{\bs{\beta}}(s) + \bs{N}_{\bs{\alpha}}(s) \bs{N}_{\bs{\beta}}'(s) \\
                    &\qquad+ \bs{B}_{\bs{\alpha}}'(s) \bs{B}_{\bs{\beta}}(s) + \bs{B}_{\bs{\alpha}}(s) \bs{B}_{\bs{\beta}}'(s) \\
                    &= 0.
                \end{align*}
            \item
                Parts (c) and (b) combined inform that the Frenet frames of $\bs{\alpha}$ and $\bs{\alpha}$ agree at $s$. Because any three-dimensional curve is specified uniquely by its tangent, normal, and binormal vectors at each $s$, $\bs{\alpha} = \bs{\beta}$.
        \end{enumerate}

    \item[11.]
        \begin{enumerate}
            \item
                One parameterization $\gamma(t) : \mathbb{R} \to \mathbb{R}^3$ is given by $(1 - t)(x, y, 0) + t(0, 0, 1)$.
            \item
                The line $\gamma(t) = ((1 - t)x, (1 - t)y, t)$ intersects $S^2$ wherever it is at a distance of $1$ from the origin, i.e., where $[(1 - t)x]^2 + [(1 - t)y]^2 + t^2 = 1$, which may be simplified to
                \begin{align*}
                    (1 - t)^2 (x^2 + y^2) + t^2 = 1,
                \end{align*}
                or further, if we denote the constant $x^2 + y^2$ by $r$, to
                \begin{align*}
                    (r + 1)t^2 - 2rt + (r - 1) = 0.
                \end{align*}
                The discriminant of this quadratic equation is $4r^2 - 4(r + 1)(r - 1) = 4 > 0$, which means the equation has two distinct real roots, and hence $\gamma$ intersects $S^2$ at the two points corresponding to these roots. By substituting, it can be seen that one of these roots is $t = 1$, which corresponds to the point $(0, 0, 1)$.
            \item
                This point lies on the line since evaluating $\gamma(t)$ at $t = \frac{u^2 + v^2 - 1}{u^2 + v^2 + 1}$ (and letting $x = u, y = v$) gives the same expression, and the expression also satisfies the distance requirement for points in $S_1$:
                \begin{align*}
                    \norm{\bs{f}(u, v)}^2 &= \left( \frac{2u}{u^2 + v^2 + 1} \right)^2 + \left( \frac{2v}{u^2 + v^2 + 1} \right)^2 + \left( \frac{u^2 + v^2 - 1}{u^2 + v^2 + 1} \right)^2 \\
                    &= \frac{u^4 + v^4 + 2u^2v^2 + 2u^2 + 2v^2 + 1}{(u^2 + v^2 + 1)^2} \\
                    &= 1.
                \end{align*}
            \item
                We know that $S^2$ is a 2-manifold of $\mathbb{R}^3$. If $(x, y, \sqrt{1 - x^2 - y^2})$ is a point on $S^2 \setminus (0, 0, 1)$, then solving
                \begin{align*}
                    \frac{2u}{x} = \frac{2v}{y} = \frac{u^2 + v^2 - 1}{\sqrt{1 - x^2 - y^2}}
                \end{align*}
                for $(u, v)$ gives $\bs{f}^{-1}(x, y, z)$, which %so we can see that the inverse of $\bs{f} : \mathbb{R}^2 \to S^2 \setminus (0, 0, 1)$
                exists everywhere in $S^2 \setminus (0, 0, 1)$. % and is hence bijective.
                Having continuous components, $\bs{f}^{-1}(x, y, z)$ is continuous. Then $f$ is a local parameterization around $(u, v)$ because
                \begin{align*}
                    D\bs{f}_{(u, v)} = \frac{2}{(u^2 + v^2 + 1)^2} \left[ \begin{array}{cc}
                            -u^2 + v^2 + 1 & 0 \\
                            0 & u^2 - v^2 + 1 \\
                            2u & 2v
                    \end{array} \right]
                \end{align*}
                is of rank $2$.
        \end{enumerate}
\end{enumerate}

\end{document}
