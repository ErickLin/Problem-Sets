\documentclass[a4paper,12pt]{article}

\usepackage{amsmath, fancyhdr, siunitx}
\pagestyle{fancy}
\rhead{Erick Lin}

\begin{document}

\section*{PHYS 2213B - Quiz 4 Corrections}

\begin{enumerate}
    \setcounter{enumi}{2}
    \item
        Because electrons in a conductor follow the Fermi-Dirac distribution, we have
        \begin{align}
            n(E) = g(E) F_{\text{FD}},
        \end{align}
        and in the limit as $T \to 0$,
        \begin{align}
            F_{\text{FD}} = \begin{cases}
                1 & \text{for } E < E_\text{F} \\
                0 & \text{for } E > E_\text{F}
            \end{cases}.
        \end{align}
        Then we have in the limit as $T \to 0$ that
        \begin{align*}
            \overline{E} &= \frac{1}{N} \int_0^\infty E n(E) dE \\
            &= \frac{1}{N} \int_0^{E_\text{F}} E g(E) dE \\
            &= \frac{1}{N} \int_0^{E_\text{F}} \frac{3N}{2} E_\text{F}^{-3/2} E^{3/2} dE \\
            &= \frac{3}{2} E_\text{F}^{-3/2} \int_0^{E_\text{F}} E^{3/2} dE \\
            &= \frac{3}{2} E_\text{F}^{-3/2} \left[ \frac{2}{5} E^{5/2} \right]_0^{E_\text{F}} \\
            &= \frac{3}{5} E_\text{F}.
        \end{align*}
        The fraction of electrons with energy $E < \overline{E}$ is given by
        \begin{align*}
            \frac{\int_0^{\overline{E}} n(E) dE}{\int_0^\infty n(E) dE} &= \frac{\int_0^{3E_\text{F}/5} g(E) dE}{\int_0^{E_\text{F}} g(E) dE} \\
            &= \frac{\frac{3N}{2} E_\text{F}^{-3/2} \int_0^{3E_\text{F}/5} E^{1/2} dE}{\frac{3N}{2} E_\text{F}^{-3/2} \int_0^{E_\text{F}} E^{1/2} dE} \\
            &= \frac{\left[ \frac{2}{3} E^{3/2} \right]_0^{3E_\text{F}/5}}{\left[ \frac{2}{3} E^{3/2} \right]_0^{E_\text{F}}} \\
            &= \frac{\left( \frac{3}{5} E_\text{F} \right)^{3/2}}{\left( E_\text{F} \right)^{3/2}} \\
            &= \left( \frac{3}{5} \right)^{3/2}.
        \end{align*}

\end{enumerate}

\end{document}
