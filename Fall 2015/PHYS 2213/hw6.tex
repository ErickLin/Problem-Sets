\documentclass[a4paper,12pt]{article}

\usepackage{amsfonts, amsmath, enumitem, fancyhdr, siunitx}
\usepackage[margin=1in]{geometry}
\pagestyle{fancy}
\rhead{Erick Lin}
\allowdisplaybreaks

\begin{document}

\section*{PHYS 2213B - HW6}

\begin{enumerate}
    \item
        Rb: $1s^2 2s^2 2p^6 3s^2 3p^6 4s^2 3d^{10} 4p^6 5s^1$ \\
        Ag: $1s^2 2s^2 2p^6 3s^2 3p^6 4s^2 3d^{10} 4p^6 4d^{10} 5s^1$ \\
        Hf: $1s^2 2s^2 2p^6 3s^2 3p^6 4s^2 3d^{10} 4p^6 4d^{10} 5s^2 5p^6 4f^{14} 5d^2 6s^2$ \\
        Sb: $1s^2 2s^2 2p^6 3s^2 3p^6 4s^2 3d^{10} 4p^6 4d^{10} 5s^2 5p^3$

    \item
        \begin{enumerate}
            \item
                The Biot-Savart law is given by
                \begin{align*}
                    \vec{B} = \frac{\mu_0}{4\pi} \int_C \frac{I d\vec{l} \times \vec{r}}{r^3}
                \end{align*}
                where $C$ is the path given by the orbit of the electron, $I$ is the current flowing along $C$ due to the movement of the electron, $d\vec{l}$ is a vector whose magnitude is the length of a differential element of the orbit in the direction of the current, and $r = |\vec{r}|$. Because $I$ is the time derivative of the charge $Ze$ (the nucleus is moving with respect to the electron) and $\mu_0 = 1/(\epsilon_0 c^2)$, the above equation becomes
                \begin{align*}
                    \vec{B} &= \frac{1}{4\pi \epsilon_0 c^2} \int_C \frac{\frac{d(Ze)}{dt} d\vec{l} \times \vec{r}}{r^3} \\
                    &= \frac{1}{4\pi \epsilon_0 c^2} \int_C \frac{d(Ze) \frac{d\vec{l}}{dt} \times \vec{r}}{r^3} \\
                    &= \frac{1}{4\pi \epsilon_0 c^2} \int_C \frac{d(Ze) \vec{v} \times \vec{r}}{r^3} \\
                    &= \frac{Ze}{4\pi \epsilon_0 c^2} \frac{m\vec{v} \times \vec{r}}{mr^3} \\
                    &= \frac{Ze \vec{L}}{4\pi \epsilon_0 m c^2 r^3}.
                \end{align*}

            \item
                In spin-orbit coupling, the dipole potential energy is given by
                \begin{align*}
                    V_{sl} = -\vec{\mu}_S \cdot \vec{B}
                \end{align*}
                where $\vec{\mu}_S$ is the spin angular momentum, which for an electron is given by
                \begin{align*}
                    \vec{\mu}_S = g \frac{e}{2m} \vec{S} = -2.0023 \frac{e}{2m} \vec{S} \approx -\frac{e}{m} \vec{S}.
                \end{align*}
                Then we have
                \begin{align*}
                    V_{sl} = \frac{Ze^2 \vec{S} \cdot \vec{L}}{4\pi \epsilon_0 m^2 c^2 r^3}.
                \end{align*}
        \end{enumerate}

    \item
        \begin{enumerate}
            \item
                The combined rotational inertia $I_x$ for the two particles, where $m$ is the mass of each particle and $R$ is the distance from each particle to the center of mass, is
                \begin{align*}
                    I = 2(mR^2) = 2(\SI{16}{\amu}) \left( \frac{\SI{121e-12}{\m}}{2} \right)^2 = \SI{1.17128e-19}{\amu\m\squared}.
                \end{align*}

            \item
                The combined rotational inertia $I_z$ for the two particles is
                \begin{align*}
                    I = 2 \left( \frac{2mr^2}{5} \right) = \frac{4}{5} (\SI{16}{\amu}) (\SI{3.0e-15}{m})^2 = \SI{1.152e-28}{\amu\m\squared}.
                \end{align*}

            \item
                Using this classical model, it appears that rotations around the $z$ axis should add a third rotational degree of freedom. However, quantum theory predicts that in a rigid rotator, the allowed energy levels are
                \begin{align*}
                    E = \frac{L^2}{2I} = \frac{\hbar^2 \ell(\ell + 1)}{2I}
                \end{align*}
                where $I$ is the total rotational inertia and $\ell \in \mathbb{N}$. Because $I_z \ll I_x$, rotation about the $z$ axis would lead to a high rotational energy. Otherwise, when the rotational energy is relatively low (which is required for small quantum numbers like that of the oxygen atom), only rotations about the $x$ and $y$ axes are permitted. This justifies the proposition that diatomic molecules have two rotational degrees of freedom.
        \end{enumerate}

    \item
        \begin{enumerate}
            \item
                \iffalse
                We will make use of the following results:
                \begin{gather}
                    \lambda = \frac{h}{m v_{\text{rms}}} \\
                    \frac{N}{V} = \frac{1}{d^3} \\
                    \overline{K.E.} = \overline{\frac{1}{2}mv^2} = \frac{1}{2} m \overline{v^2} = \frac{3}{2}kT \\
                    \Rightarrow v_{\text{rms}} = \sqrt{\overline{v^2}} = \sqrt{ \frac{3kT}{m} }.
                \end{gather}
                The conclusion is that $\lambda \ll d$ implies
                \begin{gather*}
                    \frac{h}{m v_{\text{rms}}} \ll \left( \frac{V}{N} \right)^{1/3} \\
                    \frac{h^3}{m^3 v_{\text{rms}}^3} \ll \frac{V}{N} \\
                    \frac{N}{V} \frac{h^3}{(3mkT)^{3/2}} \ll 1.
                \end{gather*}
                \fi
                Since $\overline{K} = \frac{3}{2} kT$ in the Maxwell velocity distribution, we have that
                \begin{align*}
                    \lambda = \frac{h}{p} = \frac{h}{\sqrt{2m\overline{K}}} = \frac{h}{\sqrt{3mkT}}.
                \end{align*}
                Then if $\lambda \ll d$,
                \begin{gather*}
                    \frac{h}{\sqrt{3mkT}} \ll \left( \frac{V}{N} \right)^{1/3} \\
                    \frac{h^3}{(3mkT)^{3/2}} \ll \frac{V}{N} \\
                    \frac{N}{V} \frac{h^3}{(3mkT)^{3/2}} \ll 1.
                \end{gather*}

            \item
                Using the equation of state for an ideal gas $N / V = P / (RT)$ and substituting in the given values as well as the Planck and Boltzmann constants, the mass of an Ar atom, and Avogadro's number,
                \begin{align*}
                    \frac{\lambda}{d} &\approx \frac{\SI{1}{\text{atm}} \left( \SI{6.022e23}{\per\mol} \right)}{\left( \SI{8.206e-5}{\m\cubed\text{atm}\per\K\per\mol} \right) \left( \SI{293}{\K} \right)} \times \\
                    &\qquad \frac{(\SI{6.626e-34}{\m\squared\kg\per\s)^3}}{\left[ 3(\SI{6.63e-26}{\kg})(\SI{1.38e-23}{\J\per\K})(\SI{293}{\K}) \right]^{3/2}} \\
                    &\approx 3.19 \times 10^{-7}
                \end{align*}
                which is orders of magnitude smaller than $1$ so Maxwell-Boltzmann statistics should be an accurate approximator for argon gas. \par
                On the other hand, substituting in the given values as well as the Planck and Boltzmann constants and the mass of an electron,
                \begin{align*}
                    \frac{\lambda}{d} &\approx \SI{5.86e28}{\per\m\cubed} \frac{(\SI{6.626e-34}{\m\squared\kg\per\s)^3}}{\left[ 3(\SI{9.109e-31}{\kg})(\SI{1.38e-23}{\J\per\K})(\SI{293}{\K}) \right]^{3/2}} \\
                    &\approx 1.47 \times 10^{4}
                \end{align*}
                which is greater than $1$ so in the case of conduction electrons in silver Maxwell-Boltzmann statistics are no longer valid.
        \end{enumerate}

    \item
        The ratio of hydrogen atoms by state is given by
        \begin{align}
            \frac{n(E_2)}{n(E_1)} = \frac{g(E_2)}{g(E_1)} e^{(E_1 - E_2) / (kT)}
        \end{align}
        where $g(E_i)$ is the number of possible configurations for the sole electron when the atom is in the state corresponding to energy level $i$. Substituting the proper values (including $g(E_1) = 2$, $g(E_2) = 8$, and $E_1 - E_2 = \SI{-10.2}{\eV}$) for the hydrogen atom gives
        \begin{align}
            \frac{1}{9999999} = 4e^{\SI{-10.2}{\eV} / (kT)}
        \end{align}
        which becomes, when rearranged,
        \begin{align*}
            T = \frac{\SI{-10.2}{\eV}}{k \ln^{1/(4 \times 9999999)}} \approx \SI{6762}{\K}.
        \end{align*}
\end{enumerate}
\begin{enumerate}[label=\textbf{Q\arabic*.}]
    \item
        (b)

    \item
        (d)

    \item
        (d)

    \item
        (d)

    \item
        (b)

\end{enumerate}
\end{document}
