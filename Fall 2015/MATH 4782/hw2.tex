\documentclass[a4paper,12pt]{article}

\usepackage[margin=3.5cm]{geometry}
\usepackage{amsfonts, amsmath, physics}
\allowdisplaybreaks

\begin{document}
	
\section*{MATH/PHYS 4782 - HW2 Solutions}

\begin{enumerate}

	\item[4.2.] Using Taylor expansions and splitting the summation,
        \begin{align*}
            \mbox{exp}(iAx) &= \sum_{k = 0}^{\infty} \frac{(iAx)^k}{k!} \\
            &= \sum_{k = 0}^{\infty} \frac{(iAx)^{2k}}{(2k)!} + \sum_{k = 0}^{\infty} \frac{(iAx)^{2k + 1}}{(2k + 1)!} \\
            &= \sum_{k = 0}^{\infty} \frac{(A^2)^k (-1)^k x^{2k}}{(2k)!} + \sum_{k = 0}^{\infty} \frac{iA(A^2)^k (-1)^k x^{2k + 1}}{(2k + 1)!} \\
            &= I \sum_{k = 0}^{\infty} \frac{(-1)^k x^{2k}}{(2k)!} + iA \sum_{k = 0}^{\infty} \frac{(-1)^k x^{2k + 1}}{(2k + 1)!} \\
            &= \cos(x) I + i\sin(x)A.
        \end{align*}

    \item[4.4.] Assume $H$ takes the form $\frac{1}{-e^{i\varphi}} R_x(\varphi) R_y(\varphi) R_z(\varphi)$ with $\varphi = -\frac{\pi}{2}$. Verifying that the quantities are indeed equal (using equations (4.4) and (4.6), as well as the properties of the Pauli matrices $X^2 = I$, $XZ + ZX = 0 \Rightarrow XZ = -ZX \Rightarrow XZX = -ZX^2 = -Z$),
        \begin{align*}
            &\frac{R_x(-\frac{\pi}{2}) R_y(-\frac{\pi}{2}) R_z(-\frac{\pi}{2})}{-e^{-i\pi/2}} \\
            &= \frac{e^{i\pi X/4} e^{i\pi Z / 4} e^{i\pi X /4}}{-e^{-i\pi/2}} \\
            &= \frac{\left[ \cos{\left( \frac{\pi}{4} \right)} I + i\sin{\left( \frac{\pi}{4} \right)} X \right] \left[ \cos{\left( \frac{\pi}{4} \right)} I + i\sin{\left( \frac{\pi}{4} \right)} Z \right] \left[ \cos{\left( \frac{\pi}{4} \right)} I + i\sin{\left( \frac{\pi}{4} \right)} X \right]}{-e^{-i\pi/2}} \\
            &= \frac{(I + iX)(I + iZ)(I + iX)}{-2\sqrt{2}e^{-i\pi/2}} \\
            &= \frac{I + iX + iZ - ZX + iX - X^2 - XZ - iXZX}{-2\sqrt{2}e^{-i\pi/2}} \\
            &= \frac{i(X + Z)}{-\sqrt{2}e^{-i\pi/2}} \\
            &= \frac{i}{-e^{-i\pi/2}} \left( \frac{1}{\sqrt{2}} \right) \left(
            \left[ \begin{array}{cc}
                0 & 1 \\
                1 & 0
            \end{array} \right]
            + \left[ \begin{array}{cc}
                1 & 0 \\
                0 & -1
            \end{array} \right] \right) \\
            &= \frac{1}{\sqrt{2}}
            \left[ \begin{array}{cc}
                1 & 1 \\
                1 & -1
            \end{array} \right] \\
            &= H.
        \end{align*}

    \item[4.5.] Using the property of the Pauli matrices that $\sigma_i \sigma_j = -\sigma_j \sigma_i$ when $i \neq j$ and $\sigma_i \sigma_j = I$ when $i = j$,
        \begin{align*}
            (\hat{n} \cdot \vec{\sigma})^2 &= \left( \sum_{i = x}^{z} n_i \sigma_i \right)^2 \\
            &= \sum_{i = x}^{z} \sum_{j = x}^{z} n_i n_j \sigma_i \sigma_j \\
            &= \sum_{i = x}^{z} n_i^2 I \\
            &= \norm{\hat{n}} I \\
            &= I.
        \end{align*}
        Now, writing the Taylor expansion for $R_{\hat{n}}(\theta)$ and splitting the summation,
        \begin{align*}
            e^{-i\theta \hat{n} \cdot \vec{\sigma} / 2} &= \sum_{k = 0}^{\infty} \frac{(-i\theta \hat{n} \cdot \vec{\sigma} / 2)^k}{k!} \\
            &= \sum_{k = 0}^{\infty} \frac{(-i\theta \hat{n} \cdot \vec{\sigma} / 2)^{2k}}{(2k)!} + \sum_{k = 0}^{\infty} \frac{(-i\theta \hat{n} \cdot \vec{\sigma} / 2)^{2k+ 1}}{(2k + 1)!} \\
            &= \sum_{k = 0}^{\infty} \frac{(-1)^{2k} (\hat{n} \cdot \vec{\sigma})^{2k} (-1)^k (\theta / 2)^{2k}}{(2k)!} \\
            &\qquad+ \sum_{k = 0}^{\infty} \frac{(-1)^{2k + 1} i (\hat{n} \cdot \vec{\sigma})^{2k + 1} (-1)^k (\theta / 2)^{2k + 1}}{(2k + 1)!} \\
            &= \cos{\left(\frac{\theta}{2}\right)} I - i(\hat{n} \cdot \vec{\sigma}) \sin{\left(\frac{\theta}{2}\right)} \\
            &= \cos{\left(\frac{\theta}{2}\right)} I - i\sin{\left(\frac{\theta}{2}\right)} (n_x X + n_y Y + n_z Z).
        \end{align*}

    \item[4.7.]
        \begin{align*}
            XYX &=
            \left[ \begin{array}{cc}
                0 & 1 \\
                1 & 0
            \end{array} \right]
            \left[ \begin{array}{cc}
                0 & -i \\
                i & 0
            \end{array} \right]
            \left[ \begin{array}{cc}
                0 & 1 \\
                1 & 0
            \end{array} \right] \\
            &=
            \left[ \begin{array}{cc}
                i & 0 \\
                0 & -i 
            \end{array} \right]
            \left[ \begin{array}{cc}
                0 & 1 \\
                1 & 0
            \end{array} \right] \\
            &=
            \left[ \begin{array}{cc}
                0 & i \\
                -i & 0
            \end{array} \right] \\
            &= -Y
        \end{align*}
        Using equation (4.5),
        \begin{align*}
            XR_y(\theta)X &= X \left(I \cos{\frac{\theta}{2}} - iY \sin{\frac{\theta}{2}} \right) X \\
            &= X^2 \cos{\frac{\theta}{2}} - iXYX \sin{\frac{\theta}{2}} \\
            &= I\cos{\frac{\theta}{2}} + iY \sin{\frac{\theta}{2}} \\
            &= e^{i\theta Y / 2} \\
            &= R_y(-\theta).
        \end{align*}

    \item[4.8.]
        \begin{enumerate}

            \item As a unitary matrix, $U^\dagger U = I$ and $U = 
                \left[ \begin{array}{cc}
                        a & b \\
                        c & d
                \end{array} \right]
                $
                can be written as a linear combination of the Pauli matrices
                \[
                    U = c_0 I + c_1 X + c_2 Y + c_3 Z,
                \]
                with each coefficient nonzero. Then using the properties of the Pauli matrices $X^2 = I$, $Y^2 = I$, $Z^2 = I$, $XY = -YX = iZ$, $XZ = -ZX = -iY$, and $YZ = -ZY = iX$,
                \begin{align*}
                    U^\dagger U = I &= (c_0^* I + c_1^* X + c_2^* Y + c_3^* Z)(c_0 I + c_1 X + c_2 Y + c_3 Z) \\
                    &= (|c_0|^2 + |c_1|^2 + |c_2|^2 + |c_3|^2)I + (c_0^* c_1 + c_1^* c_0 + i c_2^* c_3 - i c_3^* c_2)X \\
                    &+ (c_0^* c_2 - i c_1^* c_3 + c_2^* c_0 + i c_3^* c_1)Y + (c_0^* c_3 + i c_1^* c_2 - i c_2^* c_1 + c_3^* c_0)Z
                \end{align*}
                which implies that
                \begin{align}
                    |c_0|^2 + |c_1|^2 + |c_2|^2 + |c_3|^2 &= 1 \\
                    c_0^* c_1 + c_1^* c_0 + i c_2^* c_3 - i c_3^* c_2 &= 0 \\
                    c_0^* c_2 + c_2^* c_0 + i c_3^* c_1 - i c_1^* c_3 &= 0 \\
                    c_0^* c_3 + c_3^* c_0 + i c_1^* c_2 - i c_2^* c_1 &= 0.
                \end{align}
                Now define $\vartheta, \alpha$ such that $\cos{\frac{\vartheta}{2}} = |c_0|$ (which is possible because $|c_0|^2 < 1$) and $\mbox{exp}(i \alpha) = c_0 / |c_0|$ so that
                \[
                    c_0 = \mbox{exp}(i \alpha) \cos{\frac{\vartheta}{2}}.
                \]
                From (1),
                \begin{align*}
                    |c_1|^2 + |c_2|^2 + |c_3|^2 = 1 - \cos^2{\frac{\vartheta}{2}} = \sin^2{\frac{\vartheta}{2}}.
                \end{align*}
                Normalizing the right hand side,
                \begin{align*}
                    \left| \frac{c_1}{\sin{\frac{\vartheta}{2}}} \right|^2 + \left| \frac{c_2}{\sin{\frac{\vartheta}{2}}} \right|^2 + \left| \frac{c_3}{\sin{\frac{\vartheta}{2}}} \right|^2 = 1
                \end{align*}
                so we can define $n_x = |c_1 / \sin{\frac{\vartheta}{2}}|$, $n_y = |c_2 / \sin{\frac{\vartheta}{2}}|$, and $n_z = |c_3 / \sin{\frac{\vartheta}{2}}|$. \par
                The last element required is the phases of $c_1$, $c_2$, and $c_3$, which will be denoted by $\alpha_1$, $\alpha_2$, and $\alpha_3$. Substituting into (2) above,
                \begin{align*}
                    0 &= c_0^* c_1 + c_1^* c_0 + i c_2^* c_3 - i c_3^* c_2 \\
                    &= \mbox{exp}(-i \alpha) \cos{\frac{\vartheta}{2}} \mbox{exp}(i \alpha_1) n_x \sin{\frac{\vartheta}{2}} + \mbox{exp}(-i \alpha_1) n_x \sin{\frac{\vartheta}{2}} \mbox{exp}(i \alpha) \cos{\frac{\vartheta}{2}} \\
                    &\qquad+ i\mbox{exp}(-i \alpha_2) n_y \sin{\frac{\vartheta}{2}} \mbox{exp}(i \alpha_3) n_z \sin{\frac{\vartheta}{2}} \\
                    &\qquad- i\mbox{exp}(-i \alpha_3) n_z \sin{\frac{\vartheta}{2}} \mbox{exp}(i \alpha_2) n_y \sin{\frac{\vartheta}{2}} \\
                    &= 2 \cos(\alpha - \alpha_1) n_x \cos{\frac{\vartheta}{2}} \sin{\frac{\vartheta}{2}} + 2i \sin(\alpha_3 - \alpha_2) n_y n_z \sin^2{\frac{\vartheta}{2}},
                \end{align*}
                which shows that $\cos(\alpha - \alpha_1) = 0$ and $\sin(\alpha_3 - \alpha_2) = 0$, since the other factors are nonzero. Then $\alpha_1 = \alpha + \frac{\pi}{2} + k\pi \; \forall \; k \in \mathbb{N}$, and $\exp(i \alpha_1) = \exp(i \alpha) \exp(i \frac{\pi}{2}) \exp(ik\pi) = \pm i \exp(i \alpha)$. Due to symmetry, (3) and (4) produce the same result for $\alpha_2$ and $\alpha_3$. Now we have
                \begin{align*}
                    c_1 &= \pm i \mbox{exp}(i \alpha) \sin\frac{\vartheta}{2} n_x \\
                    c_2 &= \pm i \mbox{exp}(i \alpha) \sin\frac{\vartheta}{2} n_y \\
                    c_3 &= \pm i \mbox{exp}(i \alpha) \sin\frac{\vartheta}{2} n_z
                \end{align*}
                so
                \begin{align*}
                    U = \mbox{exp}(i \alpha) \left[ \cos\frac{\vartheta}{2} I \pm i \sin\frac{\vartheta}{2} (n_x X + n_y Y + n_z Z) \right] = \mbox{exp}(i \alpha) R_{\hat{n}}(\theta)
                \end{align*}
                where $\theta = -\vartheta$ if the $\pm$ turns out to be a $+$, and $\vartheta$ otherwise.

            \item In terms of the Pauli matrices
                \[
                    H = \frac{X + Z}{\sqrt{2}},
                \]
                so following the formula, the proper choices are
                \begin{align*}
                    \alpha &= \frac{\pi}{2} \; (\mbox{to cancel out the } {-i} \; \mbox{in the second term}) \\
                    \theta &= \pi \\
                    \hat{n} &= \frac{(1, 0, 1)}{\sqrt{2}}.
                \end{align*}

            \item In terms of the Pauli matrices
                \[
                    S = \frac{1 + i}{2} I + \frac{1 - i}{2} Z
                \]
                so the proper choices are
                \begin{align*}
                    \alpha &= \frac{\pi}{4} \; (\mbox{to get the } 1 \pm i \; \mbox{in the coefficients}) \\
                    \theta &= \frac{\pi}{2} \\
                    \hat{n} &= (0, 0, 1).
                \end{align*}

            \iffalse
            \item Let $\alpha = -\frac{\pi}{2}$, $\theta = \frac{\pi}{2}$, and $\hat{n} = \frac{(1, 0, 1)}{\sqrt{2}}$. Then
                \begin{align*}
                    \mbox{exp} \left( -\frac{i\pi}{2} \right) R_{\hat{n}} \left( \frac{\pi}{2} \right) &= \left( \cos{\frac{\pi}{2}} - i\sin{\frac{\pi}{2}} \right) \left[ \cos{\left( \frac{\pi}{2} \right)} I - i\sin{\left( \frac{\pi}{2} \right)} (\hat{n} \cdot \vec{\sigma}) \right] \\
                    &= (0 - i)[0 - i \left( \frac{X + Z}{\sqrt{2}} \right)] \\
                    &= H
                \end{align*}

            \item Let $\alpha = \frac{\pi}{4}$, $\theta = -\frac{\pi}{4}$, and $\hat{n} = (1, 0, 1)$. Then
                \begin{align*}
                    \mbox{exp} \left( \frac{i\pi}{4} \right) R_{\hat{n}} \left( -\frac{\pi}{4} \right) &= \left( \cos{\frac{\pi}{4}} + i\sin{\frac{\pi}{4}} \right) \left[ \cos{\left( -\frac{\pi}{4} \right)} I + i\sin{\left( -\frac{\pi}{4} \right)} (\hat{n} \cdot \vec{\sigma}) \right] \\
                    &= \left( \frac{\sqrt{2}}{2} + i\frac{\sqrt{2}}{2} \right) \left[\frac{\sqrt{2}}{2}I - i\frac{\sqrt{2}}{2} Z \right] \\
                    &= \frac{1}{2}I - \frac{1}{2}iZ + \frac{1}{2}iI + \frac{1}{2}Z \\
                    &=
                    \left[ \begin{array}{cc}
                        1 & 0 \\
                        0 & i
                    \end{array} \right]
                \end{align*}
            \fi
        \end{enumerate}

    \item[4.9.] $U$ can be written as
        $\left[ \begin{array}{cc}
                a & b \\
                c & d
        \end{array} \right],$
        which has the properties
        \begin{align*}
            |a|^2 + |c|^2 &= |b|^2 + |d|^2 = 1 \\
            ab^* + cd^* &= 0.
        \end{align*}
        The first equality shows that if any element $e$ in the matrix equals $0$, then the other element in the same column must have magnitude $1$. In the second equality, the term not containing $e$ must have the element diagonally opposite $e$ in the matrix equal $0$, because the only other factor of that term is in the same column as $e$, in order for the equality to hold. Then the remaining element (the one in the same row as $e$) must also be of magnitude $1$ by the first equality. This shows that exactly two elements have magnitude $1$, and they are diagonally opposite from one another. If they lie on the standard diagonal, $\gamma$ in equation (4.12) can be taken to be $0$, and if they lie on the other diagonal, then $\gamma$ can be taken to be $\pi$. The different coefficients satisfy all possible combinations of phases. \par
        The other case is when all of the elements are nonzero. First, some trigonometric properties can be used. Due to the first equality, $\gamma$, $\gamma'$ can be defined such that $|a| = \cos(\gamma / 2)$, $|c| = \sin(\gamma / 2)$, $|b| = \cos(\gamma' / 2)$, $|d| = \sin(\gamma' / 2)$. Due to the second equality,
        \begin{align*}
            0 &= ab^* + cd^* \\
            -ab^* &= cd^* \\
            |{-ab}^*| &= |cd^*| \\
            |{-a}||b^*| &= |c||d^*| \\
            |a||b| &= |c||d| \\
            \cos\frac{\gamma}{2} \cos\frac{\gamma'}{2} &= \sin\frac{\gamma}{2} \sin\frac{\gamma'}{2} \\
            \cos(\frac{\gamma + \gamma'}{2}) &= 0
        \end{align*}
        This shows that $\gamma' = \gamma - \pi$, so $|b|$ can be rewritten as $\sin(\gamma / 2)$ and $|d|$ as $\cos(\gamma / 2)$. \par
        Now define $\alpha_1$, $\alpha_2$, $\alpha_3$, and $\alpha_4$ as the phases of $a$, $-b$, $c$, and $d$ respectively. Again applying the second inequality,
        \begin{gather*}
            -\cos\frac{\gamma}{2} \sin\frac{\gamma}{2} e^{i(\alpha_1 - \alpha_2)} + \sin\frac{\gamma}{2} \cos\frac{\gamma}{2} e^{i(\alpha_3 - \alpha_4)} = 0 \\
            e^{i(\alpha_1 - \alpha_2)} = e^{i(\alpha_3 - \alpha_4)} \\
            \alpha_1 - \alpha_2 = \alpha_3 - \alpha_4 \\
            \alpha_1 + \alpha_4 = \alpha_2 + \alpha_3
        \end{gather*}
        If we let $\alpha = (\alpha_1 + \alpha_4) / 2 = (\alpha_2 + \alpha_3) / 2$, then we can define $\beta$, $\delta$ such that
        \begin{align*}
            \alpha_1 &= \alpha - \beta / 2 - \delta / 2 \\
            \alpha_3 &= \alpha + \beta / 2 - \delta / 2,
        \end{align*}
        and then
        \begin{align*}
            \alpha_2 &= 2\alpha - \alpha_3 = \alpha - \beta / 2 + \delta / 2 \\
            \alpha_4 &= 2\alpha - \alpha_1 = \alpha + \beta / 2 + \delta / 2.
        \end{align*}
        Putting it all together,
        \begin{align*}
            a &= e^{i(\alpha - \beta / 2 - \delta / 2)} \cos\frac{\gamma}{2} \\
            b &= -e^{i(\alpha - \beta / 2 + \delta / 2)} \sin\frac{\gamma}{2} \\
            c &= e^{i(\alpha + \beta / 2 - \delta / 2)} \sin\frac{\gamma}{2} \\
            d &= e^{i(\alpha + \beta / 2 + \delta / 2)} \cos\frac{\gamma}{2}
        \end{align*}
        which matches the form (4.12).

    \item[4.13.]
        \begin{align*}
            HXH &= \frac{1}{2}
            \left[ \begin{array}{cc}
                1 & 1 \\
                1 & -1
            \end{array} \right]
            \left[ \begin{array}{cc}
                0 & 1 \\
                1 & 0
            \end{array} \right]
            \left[ \begin{array}{cc}
                1 & 1 \\
                1 & -1
            \end{array} \right] \\
            &= \frac{1}{2}
            \left[ \begin{array}{cc}
                1 & 1 \\
                -1 & 1
            \end{array} \right]
            \left[ \begin{array}{cc}
                1 & 1 \\
                1 & -1
            \end{array} \right] \\
            &= \frac{1}{2}
            \left[ \begin{array}{cc}
                2 & 0 \\
                0 & -2
            \end{array} \right] \\
            &= Z
        \end{align*}
        \begin{align*}
            HYH &= \frac{1}{2}
            \left[ \begin{array}{cc}
                1 & 1 \\
                1 & -1
            \end{array} \right]
            \left[ \begin{array}{cc}
                0 & -i \\
                i & 0
            \end{array} \right]
            \left[ \begin{array}{cc}
                1 & 1 \\
                1 & -1
            \end{array} \right] \\
            &= \frac{1}{2}
            \left[ \begin{array}{cc}
                i & -i \\
                -i & -i
            \end{array} \right]
            \left[ \begin{array}{cc}
                1 & 1 \\
                1 & -1
            \end{array} \right] \\
            &= \frac{1}{2}
            \left[ \begin{array}{cc}
                0 & 2i \\
                -2i & 0
            \end{array} \right] \\
            &= -Y
        \end{align*}
        \begin{align*}
            HZH &= \frac{1}{2}
            \left[ \begin{array}{cc}
                1 & 1 \\
                1 & -1
            \end{array} \right]
            \left[ \begin{array}{cc}
                1 & 0 \\
                0 & -1
            \end{array} \right]
            \left[ \begin{array}{cc}
                1 & 1 \\
                1 & -1
            \end{array} \right] \\
            &= \frac{1}{2}
            \left[ \begin{array}{cc}
                1 & -1 \\
                1 & 1
            \end{array} \right]
            \left[ \begin{array}{cc}
                1 & 1 \\
                1 & -1
            \end{array} \right] \\
            &= \frac{1}{2}
            \left[ \begin{array}{cc}
                0 & 2 \\
                2 & 0
            \end{array} \right] \\
            &= X
        \end{align*}

    \item[4.17.]
        \iffalse
        The matrix representing the action of the Hadamard matrix in the computational basis, with the top line as the control qubit, is
        \begin{align*}
            H_c =
            \left[ \begin{array}{cc}
                I & O \\
                O & H
            \end{array} \right] =
            \left[ \begin{array}{cccc}
                1 & 0 & 0 & 0 \\
                0 & 1 & 0 & 0 \\
                0 & 0 & 1/\sqrt{2} & 1/\sqrt{2} \\
                0 & 0 & 1/\sqrt{2} & -1/\sqrt{2}
            \end{array} \right]
        \end{align*}
        \fi
        The action of the CNOT gate is given by (using an identity from the previous exercise) $\ket{c} \otimes \ket{t} \rightarrow \ket{c} \otimes X^c \ket{t} = \ket{c} \otimes (HZH)^c \ket{t} = \ket{c} \otimes H^c Z^c H^c \ket{t}$. Because $H^2 = I$, this can be further simplified to
        \[
            \ket{c} \otimes H Z^c H \ket{t}.
        \] The circuit representation for this gate is shown below. \\[1.5in]

    \item[4.21.] Let the starting state be $\ket{\psi} = \ket{\psi_1 \psi_2 \psi_3} = \ket{\psi_1} \otimes \ket{\psi_2} \otimes \ket{\psi_3}$. Passing through the five controlled gates in order (and using the identity $(X^c)^2 = I$),
        \begin{align*}
            \ket{\psi} &\rightarrow \ket{\psi_1} \otimes (\ket{\psi_2} \otimes V^{\psi_2} \ket{\psi_3}) \\
            &\rightarrow (\ket{\psi_1} \otimes X^{\psi_1} \ket{\psi_2}) \otimes V^{\psi_2} \ket{\psi_3} \\
            &\rightarrow \ket{\psi_1} \otimes (X^{\psi_1} \ket{\psi_2} \otimes (V^\dagger)^{\psi_1 \oplus \psi_2} V^{\psi_2} \ket{\psi_3}) \\
            &\rightarrow (\ket{\psi_1} \otimes \ket{\psi_2}) \otimes (V^\dagger)^{\psi_1 \oplus \psi_2} V^{\psi_2} \ket{\psi_3} \\
            &\rightarrow \ket{\psi_1} \otimes \ket{\psi_2} \otimes V^{\psi_1} (V^\dagger)^{\psi_1 \oplus \psi_2} V^{\psi_2} \ket{\psi_3}
        \end{align*}
        If $\psi_1 = \psi_2 = 1$, then the result is $\ket{\psi_1} \otimes \ket{\psi_2} \otimes V^2 \ket{\psi_3} = \ket{\psi_1 \psi_2} U \ket{\psi_3}$. \\
        If $\psi_1 = 0, \psi_2 = 1$, the result is $\ket{\psi_1} \otimes \ket{\psi_2} \otimes V^\dagger V \ket{\psi_3}$, and if $\psi_1 = 1, \psi_2 = 0$, the result is $\ket{\psi_1} \otimes \ket{\psi_2} \otimes V V^\dagger \ket{\psi_3}$. Both cases evaluate to $\ket{\psi_1 \psi_2} I \ket{\psi_3}$ because $V = V^{-1}$ as an unitary matrix. \\
        Finally, if $\psi_1 = \psi_2 = 0$, then the result is also $\ket{\psi_1 \psi_2} I \ket{\psi_3}$. \par
        Since the result is $\ket{\psi_1 \psi_2} U^{\psi_1 \psi_2} \ket{\psi_3} = C^2(U) \ket{\psi_1 \psi_2} \ket{\psi_3}$ in all cases, we have that the circuit implements $C^2(U)$ when $\ket{\psi_1}$, $\ket{\psi_2}$ are taken to be the control qubits and $\ket{\psi_3}$ is taken to be the target qubit.
    \item[4.23.] Being controlled-$U$ operations, the gates match the form of the circuit shown in Figure 4.6. First, $XZX = -Z$ for one qubit, which is proved in a similar way to $XYX = -Y$ from Exercise 4.7; also, $HZH = X$ from Exercise 4.13. For $U = R_x(\theta)$, if we let $\alpha = 0$, $A = H$, $B = \mbox{exp}(-i \theta Z / 2)$, and $C = \mbox{exp}(i \theta Z / 2) H$, then
        \begin{align*}
            ABC &= H \mbox{exp}(-i \theta Z / 2) \mbox{exp}(i \theta Z / 2) H \\
            &= H^2 \\
            &= I \\
            \mbox{exp}(i \alpha) AXBXC &= H \mbox{exp}[X(-i \theta Z / 2)X] \mbox{exp}(i \theta Z / 2) H \\
            &= H \mbox{exp}(i \theta Z / 2) \mbox{exp}(i \theta Z / 2) H \\
            &= \mbox{exp}[H(i \theta Z)H] \\
            &= \mbox{exp}(i \theta X) \\
            &= R_x(\theta).
        \end{align*}
        For $U = R_y(\theta)$, if we let $\alpha = 0$, $A = I$, $B = \mbox{exp}(-i \theta Y / 2)$, and $C = \mbox{exp}(i \theta Y / 2)$, then
        \begin{align*}
            ABC &= \mbox{exp}(-i \theta Y / 2) \mbox{exp}(i \theta Y / 2) \\
            &= I \\
            \mbox{exp}(i \alpha) AXBXC &= I \mbox{exp}[X(-i \theta Y / 2)X] \mbox{exp}(i \theta Y / 2) \\
            &= \mbox{exp}(i \theta Y / 2) \mbox{exp}(i \theta Y / 2) \\
            &= \mbox{exp}(i \theta Y) \\
            &= R_x(\theta).
        \end{align*}
        In this circuit, $A$ can be removed to reduce the number of gates needed.

    \item[4.24.] Let the starting state be $\ket{\psi} = \ket{\psi_1 \psi_2 \psi_3}$. Examining the transformations for each qubit,
        \begin{align*}
            \ket{\psi_1} &\rightarrow T \ket{\psi_1} \\
            \ket{\psi_2} &\rightarrow S X^{\psi_1} T^\dagger X^{\psi_1} T^\dagger \ket{\psi_2} \\
            &= S (X^{\psi_1} T^\dagger)^2 \ket{\psi_2} \\
            \ket{\psi_3} &\rightarrow H T X^{\psi_1} T^\dagger X^{\psi_2} T X^{\psi_1} T^\dagger X^{\psi_2} H \ket{\psi_3} \\
            &= H (T X^{\psi_1} T^\dagger X^{\psi_2})^2 H \ket{\psi_3}.
        \end{align*}
        \iffalse
        \begin{align*}
            \ket{\psi} &\rightarrow \ket{\psi_1} \otimes \ket{\psi_2} \otimes H \ket{\psi_3} \\
            &\rightarrow \ket{\psi_1} \otimes \ket{\psi_2} \otimes X^{\psi_2} H \ket{\psi_3} \\
            &\rightarrow \ket{\psi_1} \otimes \ket{\psi_2} \otimes T^\dagger X^{\psi_2} H \ket{\psi_3} \\
            &\rightarrow \ket{\psi_1} \otimes \ket{\psi_2} \otimes X^{\psi_1} T^\dagger X^{\psi_2} H \ket{\psi_3} \\
            &\rightarrow \ket{\psi_1} \otimes \ket{\psi_2} \otimes T X^{\psi_1} T^\dagger X^{\psi_2} H \ket{\psi_3} \\
            &\rightarrow \ket{\psi_1} \otimes \ket{\psi_2} \otimes X^{\psi_2} T X^{\psi_1} T^\dagger X^{\psi_2} H \ket{\psi_3} \\
            &\rightarrow \ket{\psi_1} \otimes \ket{\psi_2} \otimes T^\dagger X^{\psi_2} T X^{\psi_1} T^\dagger X^{\psi_2} H \ket{\psi_3} \\
            &\rightarrow \ket{\psi_1} \otimes \ket{\psi_2} \otimes X^{\psi_1} T^\dagger X^{\psi_2} T X^{\psi_1} T^\dagger X^{\psi_2} H \ket{\psi_3} \\
            &\rightarrow \ket{\psi_1} \otimes T^\dagger \ket{\psi_2} \otimes T X^{\psi_1} T^\dagger X^{\psi_2} T X^{\psi_1} T^\dagger X^{\psi_2} H \ket{\psi_3} \\
            &\rightarrow \ket{\psi_1} \otimes X^{\psi_1} T^\dagger \ket{\psi_2} \otimes H T X^{\psi_1} T^\dagger X^{\psi_2} T X^{\psi_1} T^\dagger X^{\psi_2} H \ket{\psi_3} \\
            &\rightarrow \ket{\psi_1} \otimes T^\dagger X^{\psi_1} T^\dagger \ket{\psi_2} \otimes H T X^{\psi_1} T^\dagger X^{\psi_2} T X^{\psi_1} T^\dagger X^{\psi_2} H \ket{\psi_3} \\
            &\rightarrow \ket{\psi_1} \otimes X^{\psi_1} T^\dagger X^{\psi_1} T^\dagger \ket{\psi_2} \otimes H T X^{\psi_1} T^\dagger X^{\psi_2} T X^{\psi_1} T^\dagger X^{\psi_2} H \ket{\psi_3} \\
            &\rightarrow T \ket{\psi_1} \otimes S X^{\psi_1} T^\dagger X^{\psi_1} T^\dagger \ket{\psi_2} \otimes H T X^{\psi_1} T^\dagger X^{\psi_2} T X^{\psi_1} T^\dagger X^{\psi_2} H \ket{\psi_3} \\
        \end{align*}
        \fi
        If $\psi_1 = 0$, then
        \begin{align*}
            T \ket{\psi_1}
            &= \left[ \begin{array}{cc}
                1 & 0 \\
                0 & \mbox{exp}(i \pi / 4)
            \end{array} \right]
            \left[ \begin{array}{c}
                1 \\
                0
            \end{array} \right] \\
            &= \left[ \begin{array}{c}
                1 \\
                0
            \end{array} \right] \\
            &= \ket{\psi_1} \\
            S (X^{\psi_1} T^\dagger)^2 \ket{\psi_2}
            &= \left[ \begin{array}{cc}
                1 & 0 \\
                0 & i
            \end{array} \right]
            \left[ \begin{array}{cc}
                1 & 0 \\
                0 & \mbox{exp}(-i \pi/ 4)
            \end{array} \right]^2 \ket{\psi_2} \\
            &= \left[ \begin{array}{cc}
                1 & 0 \\
                0 & i
            \end{array} \right]
            \left[ \begin{array}{cc}
                1 & 0 \\
                0 & -i
            \end{array} \right] \ket{\psi_2} \\
            &= I \ket{\psi_2} \\
            &= \ket{\psi_2} \\
            H (T X^{\psi_1} T^\dagger X^{\psi_2})^2 H \ket{\psi_3} &= H \left(
            \left[ \begin{array}{cc}
                1 & 0 \\
                0 & \mbox{exp}(i \pi / 4)
            \end{array} \right]
            \left[ \begin{array}{cc}
                1 & 0 \\
                0 & \mbox{exp}(-i \pi / 4)
            \end{array} \right]
            X^{\psi_2} \right)^2 H \ket{\psi_3} \\
            &= H (X^{\psi_2})^2 H \ket{\psi_3} \\
            &= H^2 \ket{\psi_3} \\
            &= \ket{\psi_3} \\
        \end{align*}
        and the overall transformation is
        \begin{align*}
            \ket{\psi_1 \psi_2 \psi_3} \rightarrow \ket{\psi_1 \psi_2 \psi_3}.
        \end{align*}
        Otherwise if $\psi_1 = 1$ and $\psi_2 = 0$, then
        \begin{align*}
            T \ket{\psi_1} 
            &= \left[ \begin{array}{cc}
                1 & 0 \\
                0 & \mbox{exp}(i \pi / 4)
            \end{array} \right]
            \left[ \begin{array}{c}
                0 \\
                1
            \end{array} \right] \\
            &= \left[ \begin{array}{c}
                0 \\
                \mbox{exp}(i \pi / 4)
            \end{array} \right] \\
            S (X^{\psi_1} T^\dagger)^2 \ket{\psi_2}
            &= \left[ \begin{array}{cc}
                1 & 0 \\
                0 & i
            \end{array} \right]
            \left( \left[ \begin{array}{cc}
                0 & 1 \\
                1 & 0
            \end{array} \right]
            \left[ \begin{array}{cc}
                1 & 0 \\
                0 & \mbox{exp}(-i \pi / 4)
            \end{array} \right] \right)^2
            \left[ \begin{array}{c}
                1 \\
                0
            \end{array} \right] \\
            &= \left[ \begin{array}{cc}
                1 & 0 \\
                0 & i
            \end{array} \right]
            \left[ \begin{array}{cc}
                0 & \mbox{exp}(-i \pi / 4) \\
                1 & 0
            \end{array} \right]^2
            \left[ \begin{array}{c}
                1 \\
                0
            \end{array} \right] \\
            &= \left[ \begin{array}{cc}
                1 & 0 \\
                0 & i
            \end{array} \right]
            \left[ \begin{array}{cc}
                \mbox{exp}(-i \pi / 4) & 0 \\
                0 & \mbox{exp}(-i \pi / 4)
            \end{array} \right]
            \left[ \begin{array}{c}
                1 \\
                0
            \end{array} \right] \\
            &= \left[ \begin{array}{cc}
                \mbox{exp}(-i \pi / 4) & 0 \\
                0 & i \mbox{exp}(-i \pi / 4)
            \end{array} \right]
            \left[ \begin{array}{c}
                1 \\
                0
            \end{array} \right] \\
            &= \left[ \begin{array}{c}
                \mbox{exp}(-i \pi / 4) \\
                0
            \end{array} \right] \\
            H (T X^{\psi_1} T^\dagger X^{\psi_2})^2 H \ket{\psi_3}
            &= H \left(
            \left[ \begin{array}{cc}
                1 & 0 \\
                0 & \mbox{exp}(i \pi / 4)
            \end{array} \right]
            \left[ \begin{array}{cc}
                0 & 1 \\
                1 & 0
            \end{array} \right]
            \left[ \begin{array}{cc}
                1 & 0 \\
                0 & \mbox{exp}(-i \pi / 4)
            \end{array} \right]
            \right)^2 H \ket{\psi_3} \\
            &= H \left(
            \left[ \begin{array}{cc}
                    0 & 1 \\
                    \mbox{exp}(i \pi / 4) & 0
            \end{array} \right]
            \left[ \begin{array}{cc}
                    1 & 0 \\
                    0 & \mbox{exp}(-i \pi / 4)
            \end{array} \right]
            \right)^2 H \ket{\psi_3} \\
            &= H
            \left[ \begin{array}{cc}
                    0 & \mbox{exp}(-i \pi / 4) \\
                    \mbox{exp}(i \pi / 4) & 0
            \end{array} \right]^2
            H \ket{\psi_3} \\
            &= HIH \ket{\psi_3} \\
            &= \ket{\psi_3}
        \end{align*}
        and the overall transformation is also
        \begin{align*}
            \ket{\psi_1 \psi_2 \psi_3} &\rightarrow
            \left[ \begin{array}{c}
                0 \\
                \mbox{exp}(i \pi / 4)
            \end{array} \right]
            \otimes \left[ \begin{array}{c}
                \mbox{exp}(-i \pi / 4) \\
                0
            \end{array} \right]
            \otimes \ket{\psi_3} \\
            &= \left[ \begin{array}{c}
                0 \\
                0 \\
                1 \\
                0
            \end{array} \right]
            \otimes \ket{\psi_3} \\
            &= \left[ \begin{array}{c}
                    0 \\
                    1
            \end{array} \right]
            \otimes \left[ \begin{array}{c}
                    1 \\
                    0
            \end{array} \right]
            \otimes \ket{\psi_3} \\
            &= \ket{\psi_1 \psi_2 \psi_3}.
        \end{align*}
        Finally, if $\psi_1 = \psi_2 = 1$, then (using some results from the previous case)
        \begin{align*}
            T \ket{\psi_1}
            &= \left[ \begin{array}{cc}
                0 \\
                \mbox{exp}(i \pi / 4)
            \end{array} \right] \\
            S (X^{\psi_1} T^\dagger)^2 \ket{\psi_2}
            &= \left[ \begin{array}{cc}
                    \mbox{exp}(-i \pi / 4) & 0 \\
                    0 & i \mbox{exp}(-i \pi / 4)
            \end{array} \right]
            \left[ \begin{array}{c}
                    0 \\
                    1
            \end{array} \right] \\
            &= \left[ \begin{array}{c}
                    0 \\
                    i \mbox{exp}(-i \pi / 4)
            \end{array} \right] \\
            H (T X^{\psi_1} T^\dagger X^{\psi_2})^2 H \ket{\psi_3}
            &= H \left(
            \left[ \begin{array}{cc}
                    1 & 0 \\
                    0 & \mbox{exp}(i \pi / 4)
            \end{array} \right]
            X
            \left[ \begin{array}{cc}
                    1 & 0 \\
                    0 & \mbox{exp}(-i \pi / 4)
            \end{array} \right]
            X
            \right)^2 H \ket{\psi_3} \\
            &= H
            \left(
            \left[ \begin{array}{cc}
                    0 & 1 \\
                    \mbox{exp}(i \pi / 4) & 0
            \end{array} \right]
            \left[ \begin{array}{cc}
                    0 & 1 \\
                    \mbox{exp}(-i \pi / 4) & 0
            \end{array} \right]
            \right)^2 H \ket{\psi_3} \\
            &= H
            \left[ \begin{array}{cc}
                    \mbox{exp}(-i \pi / 4) & 0 \\
                    0 & \mbox{exp}(i \pi / 4)
            \end{array} \right]^2
            H \ket{\psi_3} \\
            &= \frac{1}{2}
            \left[ \begin{array}{cc}
                    1 & 1 \\
                    1 & -1
            \end{array} \right]
            \left[ \begin{array}{cc}
                    -i & 0 \\
                    0 & i
            \end{array} \right]
            \left[ \begin{array}{cc}
                    1 & 1 \\
                    1 & -1
            \end{array} \right]
            \ket{\psi_3} \\
            &= \frac{1}{2}
            \left[ \begin{array}{cc}
                    -i & i \\
                    -i & -i
            \end{array} \right]
            \left[ \begin{array}{cc}
                    1 & 1 \\
                    1 & -1
            \end{array} \right]
            \ket{\psi_3} \\
            &= \left[ \begin{array}{cc}
                    0 & -i \\
                    -i & 0
            \end{array} \right]
            \ket{\psi_3} \\
            &= -iX\ket{\psi_3}
        \end{align*}
        and the transformation is
        \begin{align*}
            \ket{\psi_1 \psi_2 \psi_3} &\rightarrow
            \left[ \begin{array}{cc}
                0 \\
                \mbox{exp}(i \pi / 4)
            \end{array} \right]
            \otimes
            \left[ \begin{array}{c}
                    0 \\
                    i \mbox{exp}(-i \pi / 4)
            \end{array} \right]
            \otimes
            (-iX\ket{\psi_3}) \\
            &= -i \left[ \begin{array}{c}
                    0 \\
                    0 \\
                    0 \\
                    i
            \end{array} \right]
            \otimes X\ket{\psi_3} \\
            &= \left[ \begin{array}{c}
                    0 \\
                    1
            \end{array} \right]
            \otimes
            \left[ \begin{array}{c}
                    0 \\
                    1
            \end{array} \right]
            \otimes X \ket{\psi_3} \\
            &= \ket{\psi_1 \psi_2} \otimes X \ket{\psi_3}.
        \end{align*}
        Therefore, the circuit implements the Toffoli gate because its action matches that of the Toffoli gate for all inputs.
            
    \item[4.35.] The input $\ket{\psi}$ can be considered a linear combination of states in the two-qubit computational basis
        \[
            \sum_{\psi_1, \psi_2 \in \{0, 1\}} c_{\psi_1, \psi_2} \ket{\psi_1 \psi_2}
        \]
        where $c_{\psi_1, \psi_2} \in \mathbb{C}$. \par
        In the first circuit, the controlled-$U$ gate is applied first, giving
        \[
            \sum_{\psi_1, \psi_2 \in \{0, 1\}} c_{\psi_1, \psi_2} \ket{\psi_1} \otimes U^{\psi_1} \ket{\psi_2}.
        \]
        If the measurement gives $\psi_1$ for the first qubit, then (from the measurement postulate) the final state is
        \begin{align*}
            \ket{\psi_f} &= \frac{\sum_{\psi_2 \in \{0, 1\}} c_{\psi_1, \psi_2} U^{\psi_1} \ket{\psi_2}}{\sqrt{\sum_{\psi_2 \in \{0, 1\}} |c_{\psi_1, \psi_2} U^{\psi_1} \ket{\psi_2}|^2}} \\
            &= U^{\psi_1} \frac{\sum_{\psi_2 \in \{0, 1\}} c_{\psi_1, \psi_2} \ket{\psi_2}}{\sqrt{\sum_{\psi_2 \in \{0, 1\}} c_{\psi_1, \psi_2}^2 |U^{\psi_1} \ket{\psi_2}|^2}} \\
            &= U^{\psi_1} \frac{\sum_{\psi_2 \in \{0, 1\}} c_{\psi_1, \psi_2} \ket{\psi_2}}{\sqrt{\sum_{\psi_2 \in \{0, 1\}} c_{\psi_1, \psi_2}^2}}.
        \end{align*}
        In the second circuit, if the measurement operator applied on the first qubit gives $\psi_1$, then the $\ket{\psi}$ becomes
        \begin{align*}
            \frac{\sum_{\psi_2 \in \{0, 1\}} c_{\psi_1, \psi_2} \ket{\psi_2}}{\sqrt{\sum_{\psi_2 \in \{0, 1\}} |c_{\psi_1, \psi_2} \ket{\psi_2}|^2}} = \frac{\sum_{\psi_2 \in \{0, 1\}} c_{\psi_1, \psi_2} \ket{\psi_2}}{\sqrt{\sum_{\psi_2 \in \{0, 1\}} c_{\psi_1, \psi_2}^2}}.
        \end{align*}
        Applying the controlled-$U$ gate on the second qubit with $\psi_1$ as the control bit, the final state becomes
        \begin{align*}
            \ket{\psi_f} &= \frac{\sum_{\psi_2 \in \{0, 1\}} c_{\psi_1, \psi_2} U^{\psi_1} \ket{\psi_2}}{\sqrt{\sum_{\psi_2 \in \{0, 1\}} c_{\psi_1, \psi_2}^2}} \\
            &= U^{\psi_1} \frac{\sum_{\psi_2 \in \{0, 1\}} c_{\psi_1, \psi_2} \ket{\psi_2}}{\sqrt{\sum_{\psi_2 \in \{0, 1\}} c_{\psi_1, \psi_2}^2}},
        \end{align*}
        so the gates are equivalent.

\end{enumerate}

\end{document}
