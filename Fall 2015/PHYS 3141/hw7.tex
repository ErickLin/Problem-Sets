\documentclass[a4paper,12pt]{article}

\usepackage{amsmath, enumitem, fancyhdr, gensymb, mathrsfs, siunitx, textcomp}
\usepackage[margin=3cm]{geometry}
\allowdisplaybreaks

\def\dbar{{\mathchar'26\mkern-12mu \mathrm{d}}}
\pagestyle{fancy}
\rhead{Erick Lin}

\begin{document}

\section*{PHYS 3141 - Problem Set 7 Solutions}

\begin{enumerate}[label=\textbf{[\arabic*]}]
    \item
        \begin{enumerate}
            \item
                From the first law, $dU = TdS - PdV$. Taking the partial derivatives,
                \begin{align*}
                    \left( \frac{\partial U}{\partial S} \right)_V = T(S, V) \qquad \left( \frac{\partial U}{\partial V} \right)_S = -P(S, V).
                \end{align*}
                Meanwhile, the partial derivatives of $U(S, V)$ are given by
                \begin{align*}
                    \left( \frac{\partial U}{\partial S} \right)_V = \frac{2aS}{V} \qquad \left( \frac{\partial U}{\partial V} \right)_S = a S^2 \ln V
                \end{align*}
                so we have
                \begin{align*}
                    T(S, V) = \frac{2aS}{V} \qquad -P(S, V) = aS^2 \ln V.
                \end{align*}
        
            \item
                Taking the appropriate partial derivatives yields
                \begin{align*}
                    \left( \frac{\partial T}{\partial V} \right)_S = 2aS \ln V \qquad -\left( \frac{\partial P}{\partial S} \right)_V = 2aS \ln V
                \end{align*}
                which supports the proposition that the two expressions always represent the same quantity.
        \end{enumerate}

        \item
            \begin{enumerate}
                \item
                    Taking the partial derivatives,
                    \begin{align*}
                        \left( \frac{\partial V}{\partial S} \right)_U = \frac{T}{P} \qquad \left( \frac{\partial V}{\partial U} \right)_S = -\frac{1}{P}.
                    \end{align*}
                    Using the reciprocal rule and the triple product rule, we have
                    \begin{align}
                        P &= -\left( \frac{\partial U}{\partial V} \right)_S \\
                        T &= P \left( \frac{\partial V}{\partial S} \right)_U = -\left( \frac{\partial U}{\partial V} \right)_S \left( \frac{\partial V}{\partial S} \right)_U = \left( \frac{\partial U}{\partial S} \right)_V.
                    \end{align}
                    
                \item
                    If we assume smoothness, then
                    \begin{align*}
                        \left( \frac{\partial}{\partial S} \left( \frac{\partial U}{\partial V} \right)_S \right)_V = \left( \frac{\partial}{\partial V} \left( \frac{\partial U}{\partial S} \right)_V \right)_S
                    \end{align*}
                    and substituting (1) and (2),
                    \begin{align*}
                        -\left( \frac{\partial P}{\partial S} \right)_V = \left( \frac{\partial T}{\partial V} \right)_S.
                    \end{align*}
            \end{enumerate}

        \item
            Since $\Delta H = Q + V \Delta P$ and $\Delta U = Q - P \Delta V$, we have
            \begin{enumerate}
                \item
                    $\Delta H = \SI{26000}{\text{cal}} = \SI{1.1e5}{\J}$

                \item
                    $\Delta U = \SI{110000}{\J} - \SI{101325}{\Pa} (\SI{740}{\L} + \SI{0.011}{L} - \SI{0.028}{\L}) (\SI{10e-3}{\m\cubed\per\L)} = \SI{3.4e4}{\J}$.
            \end{enumerate}

        \item
            The derivation of the first Maxwell relation is given in Exercise 2. For the second Maxwell relation, we use the definition of enthalpy to obtain
            \begin{align*}
                dH = dU + PdV + VdP.
            \end{align*}
            Since $dU + PdV = \dbar Q = TdS$, the above equation becomes
            \begin{align*}
                dH = TdS + VdP
            \end{align*}
            and taking the partial derivatives, we have
            \begin{align*}
                \left( \frac{\partial H}{\partial S} \right)_P = T \qquad \left( \frac{\partial H}{\partial P} \right)_S = V.
            \end{align*}
            Assuming smoothness,
            \begin{align*}
                \left( \frac{\partial}{\partial P} \left( \frac{\partial H}{\partial S} \right)_P \right)_S = \left( \frac{\partial}{\partial S} \left( \frac{\partial H}{\partial P} \right)_S \right)_P
            \end{align*}
            and substituting,
            \begin{align*}
                \left( \frac{\partial T}{\partial P} \right)_S = \left( \frac{\partial V}{\partial S} \right)_P.
            \end{align*}
            For the third relation, we use the Helmholtz function to obtain
            \begin{align*}
                dF = dU - TdS - SdT.
            \end{align*}
            Since $dU - TdS = -\dbar W = -PdV$, the above equation becomes
            \begin{align*}
                dF = -PdV - SdT
            \end{align*}
            and taking the partial derivatives, we have
            \begin{align*}
                \left( \frac{\partial F}{\partial V} \right)_T = -P \qquad \left( \frac{\partial F}{\partial T} \right)_V = -S.
            \end{align*}
            Assuming smoothness,
            \begin{align*}
                \left( \frac{\partial}{\partial T} \left( \frac{\partial F}{\partial V} \right)_T \right)_V = \left( \frac{\partial}{\partial V} \left( \frac{\partial F}{\partial T} \right)_V \right)_T
            \end{align*}
            and substituting,
            \begin{align*}
                \left( \frac{\partial P}{\partial T} \right)_V = \left( \frac{\partial S}{\partial V} \right)_T.
            \end{align*}
            For the last relation, we use the Gibbs function to obtain
            \begin{align*}
                dG = dU + PdV + VdP - TdS - SdT = VdP - SdT
            \end{align*}
            and taking the partial derivatives, we have
            \begin{align*}
                \left( \frac{\partial G}{\partial P} \right)_T = V \qquad \left( \frac{\partial G}{\partial T} \right)_P = -S.
            \end{align*}
            Assuming smoothness,
            \begin{align*}
                \left( \frac{\partial}{\partial T} \left( \frac{\partial G}{\partial P} \right)_T \right)_P = \left( \frac{\partial}{\partial P} \left( \frac{\partial G}{\partial T} \right)_P \right)_T
            \end{align*}
            and substituting,
            \begin{align*}
                \left( \frac{\partial V}{\partial T} \right)_P = -\left( \frac{\partial S}{\partial P} \right)_T.
            \end{align*}

        \item
            ((a) and (b) follow from Exercise 4.)
            \begin{enumerate}
                \item
                    $P = -{\left( \frac{\partial F}{\partial V} \right)_T}$

                \item
                    $S = -{\left( \frac{\partial F}{\partial T} \right)_V}$

                \item
                    $U = F + TS = F - T\left( \frac{\partial F}{\partial T} \right)_V$

                \item
                    $C_V = \left( \frac{\partial Q}{\partial T} \right)_V = T \left( \frac{\partial S}{\partial T} \right)_V = -T \left( \frac{\partial^2 F}{\partial T^2} \right)_V$ \\
                    The last equality comes from differentiating (b).

                \item
                    $\kappa_T = -{\frac{1}{V} \left( \frac{\partial V}{\partial P} \right)_T} = \frac{1}{V} \left( \frac{\partial^2 V}{\partial F^2} \right)_T$ \\
                    The last equality comes from differentiating (a) and using the reciprocal rule.

                \item
                    $\alpha_P = \frac{1}{V} \left( \frac{\partial V}{\partial T} \right)_P$

                \item
                    $\alpha_V = \frac{1}{P} \left( \frac{\partial P}{\partial T} \right)_V = -{\left( \frac{\partial V}{\partial F} \right)_T} \left( \frac{\partial S}{\partial V} \right)_T = {\left( \frac{\partial V}{\partial F} \right)_T} \frac{\partial^2 F}{\partial V \partial T}$ \\
                    A Maxwell relation is used in the second equality, and the last equality comes from differentiating (b).
            \end{enumerate}

        \item
            Expressing $dU$ in terms of $dV$ and $dT$ gives
            \begin{align*}
                dU &= \left[ T \left( \frac{\partial S}{\partial V} \right)_T - P \right] dV + T \left( \frac{\partial S}{\partial T} \right)_V dT.
            \end{align*}
            Using $C_V = \left( \frac{\partial Q}{\partial T} \right)_V = T \left( \frac{\partial S}{\partial T} \right)_V$ and the Maxwell relation $\left( \frac{\partial S}{\partial V} \right)_T = \left( \frac{\partial P}{\partial T} \right)_V$, we have
            \begin{align*}
                dU &= \left[ T \left( \frac{\partial P}{\partial T} \right)_V - P \right] dV + C_V dT
            \end{align*}
            and substituting $\alpha_V \equiv \frac{1}{P} \left( \frac{\partial P}{\partial T} \right)_V$ gives
            \begin{align*}
                dU &= P(T \alpha_V - 1) dV + C_V dT.
            \end{align*}
            Finally, taking the partial derivative yields
            \begin{align*}
                \left( \frac{\partial T}{\partial V} \right)_U = \frac{P(1 - T \alpha_V)}{C_V}.
            \end{align*}

\end{enumerate}

\end{document}
