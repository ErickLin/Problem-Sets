\documentclass[a4paper,12pt]{article}

\usepackage{amsmath, enumitem, fancyhdr, gensymb, mathrsfs, siunitx, textcomp}
\usepackage[margin=1in]{geometry}
\allowdisplaybreaks

\def\dbar{{\mathchar'26\mkern-12mu \mathrm{d}}}
\pagestyle{fancy}
\rhead{Erick Lin}

\begin{document}

\section*{PHYS 3141 - Problem Set 6 Solutions}

\begin{enumerate}[label=\textbf{[\arabic*]}]
    \item
        \begin{enumerate}
            \item
                The isothermal processes occur at $T_1$ and $T_2$ (appearing as the horizontal sides of the rectangle) with $T_1 > T_2$, and the adiabatic processes occur at $S_1$ and $S_2$ (appearing as the vertical lines of the rectangle) with $S_2 > S_1$. Then
                \begin{align*}
                    \eta = 1 - \frac{Q_{\text{out}}}{Q_{\text{in}}} = 1 - \frac{\int_{S_1}^{S_2} T_2 dS}{\int_{S_1}^{S_2} T_1 dS} = 1 - \frac{T_2}{T_1}.
                \end{align*}

            \item
                Cycle B is more efficient -- based on a geometrical estimate, the ratio of the area under the bottom side of the triangle to the area under the top side of the triangle is less in cycle B than in cycle A, or $Q_{\text{out}, B} / Q_{\text{in}, B} < Q_{\text{out}, A} / Q_{\text{in}, A}$.
        \end{enumerate}

    \item
        $ $ \\[3in]

    \item
        \begin{enumerate}
            \item
                $\Delta S_{\text{resistor}} = 0$ because the thermodynamic state of the resistor is unchanged.

            \item
                From Joule heating, $Q$ is given by
                \begin{align*}
                    Q = I^2 Rt = \SI{2500}{J} \approx \SI{3000}{J}.
                \end{align*}
                Then
                \begin{align*}
                    \Delta S_{\text{universe}} = \Delta S_{\text{resistor}} + \Delta S_{\text{surroundings}} = 0 + \frac{Q}{T} = \frac{\SI{2500}{J}}{\SI{300}{K}} \approx \SI{8}{\J\per\K}.
                \end{align*}

            \item
                This time, the resistor undergoes a heating of
                \begin{align*}
                    Q = I^2 Rt = \SI{2500}{J} \approx \SI{3000}{J}.
                \end{align*}
                Because we also have that $Q = m c_P \Delta T$,
                \begin{align*}
                    \Delta T &= \frac{Q}{mc_P} = \frac{\SI{2500}{J}}{(\SI{10}{\g}) (\SI{0.84}{\J\per\g\per\K})} \approx \SI{300}{\K} \\
                    T_f &= T_i + \Delta T \approx \SI{600}{K}
                \end{align*}
                and
                \begin{align*}
                    \Delta S_{\text{resistor}} &= \int_{T_i}^{T_f} \frac{\dbar Q}{T} \\
                    &= \int_{T_i}^{T_f} \frac{m c_P dT}{T} \\
                    &= m c_P \ln\frac{T_f}{T_i} \\
                    &= (\SI{10}{\g}) (\SI{0.84}{\J\per\g\per\K}) \ln\frac{\SI{600}{K}}{\SI{300}{K}} \\
                    &= \SI{6}{\J\per\K}
                \end{align*}
 
            \item
                Because the thermodynamic state of the surroundings is unchanged,
                \begin{align*}
                    \Delta S_{\text{universe}} = \Delta S_{\text{resistor}} + \Delta S_{\text{surroundings}} \approx 0 + \SI{6}{\J\per\K} = \SI{6}{\J\per\K}.
                \end{align*}
        \end{enumerate}

    \item
        Recall that $C_P = \left( \frac{\dbar Q}{dT} \right)_P$, so under an isobaric process $\dbar Q = C_P dT$. Note that the temperature of the body changes from $T_i$ to $T_f$ before reaching equilibrium, while the temperature of the reservoir remains constant at $T_f$. Then we can calculate the entropy change as follows:
        \begin{gather*}
            \Delta S_{\text{Body}} = \int_{T_i}^{T_f} \frac{\dbar Q}{T} = \int_{T_i}^{T_f} \frac{C_P dT}{T} = C_P \ln \frac{T_f}{T_i} = -C_P \ln \frac{T_i}{T_f} = -C_P \ln(1 + x) \\
            \Delta S_{\text{System}} = -\frac{Q}{T_f} = -\frac{\int_{T_i}^{T_f} C_P dT}{T_f} = C_P \frac{-(T_f - T_i)}{T_f} = C_P x\\
            \Delta S_{\text{Universe}} = \Delta S_{\text{Body}} + \Delta S_{\text{System}} = C_P [x - \ln(1 + x)]
        \end{gather*}
        We claim that $x > \ln(1 + x)$ for all $x > 0$, the domain of $x$. To prove this assertion, differentiate both sides to obtain $1 > \frac{1}{1 + x}$, which is true for $x > 0$ and shows that the left side of the original inequality increases more quickly than the right. Because $x = \ln(1 + x)$ if $x = 0$, the left side is initially greater than the right for $x > 0$, and it must be the case that the hypothesis holds. Thus $\Delta S_{\text{Universe}}$ is positive.

    \item
        \begin{align*}
            \Delta S &= \int_{T_i}^{T_f} \frac{\dbar Q}{T} \\
            &= \int_{T_i}^{T_f} \frac{n c_V dT}{T} \\
            &= \int_{T_i}^{T_f} \frac{12nR \pi^4 T^3}{5 \Theta^3 T} \\
            &= \left[ \frac{4nR \pi^4 T^3}{5 \Theta^3} \right]_{T_i}^{T_f} \\
            &= \frac{4nR \pi^4 (T_f^3 - T_i^3)}{5 \Theta^3} \\
            &= \frac{4(\SI{1.2}{\g}) R \pi^4 [(\SI{350}{K})^3 - (\SI{10}{K})^3]}{5(\SI{12}{\g\per\mol})(\SI{2230}{K})^3} \\
            &\approx (\SI{0.03}{\mol}) R \\
            &\approx \SI{0.3}{\J\per\K}
        \end{align*}

    \item
        \begin{enumerate}
            \item
                Because the piston is adiabatic, $P_0 V_0^\gamma = P_f V_f^\gamma$ in each partition of the cylinder. Substituting $P_{f, \text{left}} = 27 P_0 / 8$ for the left partition and $\gamma = 3/2$, we have
                \begin{align*}
                    V_{f, \text{left}}^{3/2} &= \frac{8P_0 V_0^{3/2}}{27P_0} \\
                    V_{f, \text{left}} &= \frac{4}{9} V_0,
                \end{align*}
                and as a result $V_{f, \text{right}} = V_0 + V_0 - V_{f, \text{left}} = \frac{14}{9} V_0$.

            \item
                The final pressure in the right partition is
                \begin{align*}
                    P_{f, \text{right}} &= \frac{P_0 V_0^{3/2}}{V_{f , \text{right}}^{3/2}} = \frac{27 P_0 V_0^{3/2}}{14^{3/2} V_0^{3/2}} = \frac{27}{14^{3/2}} P_0.
                \end{align*}
                From the state equation for ideal gases,
                \begin{align*}
                    \frac{T_{f, \text{right}}}{P_{f, \text{right}}{V_{f, \text{right}}}} &= \frac{T_0}{P_0 V_0} \\
                    T_{f, \text{right}} &= \frac{P_{f, \text{right}}{V_{f, \text{right}}} T_0}{P_0 V_0} \\
                    &= \frac{27}{14^{3/2}} \left( \frac{14}{9} \right) T_0 \\
                    &= \frac{3}{14^{1/2}} T_0.
                \end{align*}

            \item
                \begin{align*}
                    T_{f, \text{left}} &= \frac{P_{f, \text{left}}{V_{f, \text{left}}} T_0}{P_0 V_0} \\
                    &= \frac{27}{8} \left( \frac{4}{9} \right) T_0 \\
                    &= \frac{3}{2} T_0
                \end{align*}

            \item
                From the first law, $Q = \Delta U + W$. The change in internal energy depends on the change in temperature and is given by
                \begin{align*}
                    \Delta U = \frac{3}{2} nR \Delta T = \frac{3}{2} nR \left( \frac{1}{2} T_0 \right) = \frac{3}{4} nRT_0.
                \end{align*}
                The work $W$ is given by
                \begin{align*}
                    W &= \int_{V_0}^{V_{f, \text{left}}} P dV \\
                    &= \int_{V_0}^{V_{f, \text{left}}} \frac{P_0 V_0^\gamma}{V^\gamma} dV \\
                    &= P_0 V_0^\gamma \frac{\left( V_{f, \text{left}}^{1 - \gamma} - V_0^{1 - \gamma} \right)}{1 - \gamma} \\
                    &= P_0 V_0^{3/2} \frac{\left( \left( \frac{4}{9} V_0 \right)^{-1/2} - V_0^{-1/2} \right)}{-1/2} \\
                    &= -P_0 V_0 \\
                    &= -nRT_0.
                \end{align*}
                Then we have
                \begin{align*}
                    Q &= \frac{3}{4} nRT_0 - nRT_0 = -\frac{1}{4} nRT_0.
                \end{align*}

            \item
                \begin{align*}
                    W &= \int_{V_0}^{V_{f, \text{right}}} P dV \\
                    &= \int_{V_0}^{V_{f, \text{right}}} \frac{P_0 V_0^\gamma}{V^\gamma} dV \\
                    &= P_0 V_0^\gamma \frac{\left( V_{f, \text{right}}^{1 - \gamma} - V_0^{1 - \gamma} \right)}{1 - \gamma} \\
                    &= P_0 V_0^{3/2} \frac{\left( \left( \frac{14}{9} V_0 \right)^{-1/2} - V_0^{-1/2} \right)}{-1/2} \\
                    &= 2 P_0 V_0 \left( 1 - \frac{3}{\sqrt{14}} \right) \\
                    &= 2 nRT_0 \left( 1 - \frac{3}{\sqrt{14}} \right)
                \end{align*}

            \item
                \begin{align*}
                    \Delta S_{\text{right}} &= \int_{T_0}^{T_{f, \text{right}}} \frac{\dbar Q}{T} \\
                    &= \int_{T_0}^{T_{f, \text{right}}} \frac{dU}{T} + \int_{T_0}^{T_{f, \text{right}}} \frac{\dbar W}{T} \\
                    &= \int_{T_0}^{T_{f, \text{right}}} \frac{3nRdT}{2T} + \int_{V_0}^{V_{f, \text{right}}} \frac{nR}{V} dV \\
                    &= \frac{3}{2} nR \ln\frac{T_{f, \text{right}}}{T_0} + nR\ln\frac{V_{f, \text{right}}}{V_0} \\
                    &= nR \left( \frac{3}{2} \ln\frac{3}{14^{1/2}} + \ln\frac{14}{9} \right)
                \end{align*}

            \item
                \begin{align*}
                    \Delta S_{\text{left}} &= \int_{T_0}^{T_{f, \text{left}}} \frac{\dbar Q}{T} \\
                    &= \int_{T_0}^{T_{f, \text{left}}} \frac{dU}{T} + \int_{T_0}^{T_{f, \text{left}}} \frac{\dbar W}{T} \\
                    &= \int_{T_0}^{T_{f, \text{left}}} \frac{3nRdT}{2T} + \int_{V_0}^{V_{f, \text{left}}} \frac{nR}{V} dV \\
                    &= \frac{3}{2} nR \ln\frac{T_{f, \text{left}}}{T_0} + nR\ln\frac{V_{f, \text{left}}}{V_0} \\
                    &= nR \left( \frac{3}{2} \ln\frac{3}{2} + \ln\frac{4}{9} \right)
                \end{align*}

            \item
                \begin{align*}
                    \Delta S_{\text{universe}} &= \Delta S_{\text{right}} + \Delta S_{\text{left}} \\
                    &= nR \left( \frac{3}{2} \ln\frac{3}{14^{1/2}} + \ln\frac{14}{9} + \frac{3}{2} \ln\frac{3}{2} + \ln\frac{4}{9} \right)
                \end{align*}
        \end{enumerate}

    \item
        Because $Q = mc_P(T_f - T_i) = -m'c_P'(T_f' - T_i')$ and $T_f, T_f'$ vary directly with $T$,
        \begin{align*}
            \dbar Q &= mc_P dT = -m'c_P' dT \\
            \Rightarrow mc_P &= -m'c_P'.
        \end{align*}
        Then
        \begin{align*}
            \Delta S_{\text{universe}} &= \int_{T_i}^{T_f} \frac{mc_PdT}{T} + \int_{T_i'}^{T_f'} \frac{m' c_P' dT}{T} \\
            &= \int_{T_i}^{T_f} \frac{mc_PdT}{T} - \int_{T_i'}^{T_f'} \frac{mc_PdT}{T} \\
            \frac{dS_{\text{universe}}}{dT} &= mc_P \left( \frac{1}{T_f} - \frac{1}{T_f'} \right)
        \end{align*}
        Note that $T_i$ and $T_i'$ are constant when differentiating above. This shows that in order for $dS_{\text{universe}} / dT = 0$, $T_f = T_f'$. By evaluating $\Delta S_{\text{universe}}$ at nearby points, this is shown to indeed be a maximum and not a minimum, and being the only maximum it is the global maximum.

    \item
        \begin{enumerate}
            \item
                Increase - heat is transferred to the system while the temperature remains constant.
            \item
                Decrease
            \item
                Increase because $\Delta S = \frac{mgh}{T}$ where $h$ is the initial height.
            \item
                Remain the same because no heat is transferred
            \item
                Increases because heat must flow out for the process to be isothermal
            \item
                Increase because the process is free expansion, which is irreversible.
            \item
                Increase because the process is irreversible according to the Clausius statement of the second law.
        \end{enumerate}

    \item
        \begin{enumerate}
            \item
                Since the latent heat of melting for aluminum is $\SI{398}{\J\per\g}$ and the melting point is $\SI{933}{\K}$,
                \begin{align*}
                    \Delta S = \frac{Q}{T} = \frac{(\SI{25}{\g}) (\SI{398}{\J\per\g})}{\SI{933}{\K}} \approx \SI{11}{\J\per\K}
                \end{align*}

            \item
                \begin{align*}
                    \Delta S = \frac{Q}{T} = \frac{(\SI{10}{\g}) (\SI{-2260}{\J\per\g})}{\SI{373}{\K}} \approx \SI{-60}{\J\per\K}
                \end{align*}

            \item
                \begin{gather*}
                    Q = m_{\text{Al}} c_{\text{Al}} (T_{f, \text{Al}} - T_{i, \text{Al}}) = -m_\text{w} c_\text{w} (T_{f, \text{w}} - T_{i, \text{w}}) \\
                    T_{f, \text{Al}} = T_{f, \text{w}} = T_f \\
                    (\SI{1000}{\g}) (\SI{0.902}{\J\per\g\per\K}) (T_f - \SI{353}{\K}) = -(\SI{2000}{\g})(\SI{4.186}{\J\per\g\per\K}) (T_f - \SI{293}{\K}) \\
                    T_f \approx \SI{300}{\K}
                \end{gather*}
        \end{enumerate}

    \item
        \begin{enumerate}
            \item
                $\dbar Q = TdS \Rightarrow Q = T_H(S_2 - S_1)$

            \item
                $Q = T_C(S_1 - S_2)$
        \end{enumerate}

\end{enumerate}

\end{document}
