\documentclass[a4paper,12pt]{article}

\usepackage{amsmath, enumitem, fancyhdr, gensymb, mathrsfs, siunitx, textcomp}
\usepackage[margin=3.5cm]{geometry}
\allowdisplaybreaks

\def\dbar{{\mathchar'26\mkern-12mu \mathrm{d}}}
\pagestyle{fancy}
\rhead{Erick Lin}

\begin{document}

\section*{PHYS 3141 - Problem Set 8 Solutions}

\begin{enumerate}[label=\textbf{[\arabic*]}]
    \item
        Using partial derivative rules and known identities for the case of an ideal gas,
        \begin{align*}
            \kappa_S &= -\frac{1}{V} \left( \frac{\partial V}{\partial P} \right)_S \\
            &= \frac{1}{V} \left( \frac{\partial V}{\partial S} \right)_P \left( \frac{\partial S}{\partial P} \right)_V \\
            &= \frac{1}{V} \left( \frac{\partial V}{\partial T} \right)_P \left( \frac{\partial T}{\partial S} \right)_P \left( \frac{\partial S}{\partial T} \right)_V \left( \frac{\partial T}{\partial P} \right)_V\\
            &= \frac{1}{V} \left( \frac{\partial V}{\partial T} \right)_P \left( \frac{T}{C_P} \right) \left( \frac{C_V}{T} \right) \left( \frac{\partial T}{\partial P} \right)_V \\
            &= \frac{1}{V} \left( \frac{C_V}{C_P} \right) \left( \frac{\partial V}{\partial T} \right)_P \left( \frac{\partial T}{\partial P} \right)_V \\
            &= -\frac{1}{V} \left( \frac{C_V}{C_P} \right) \left( \frac{\partial V}{\partial P} \right)_T \\
            &= -\frac{1}{V} \left( \frac{C_V}{C_P} \right) \left( \frac{-V^2}{nRT} \right) \\
            &= \frac{1}{V} \left( \frac{C_V}{C_P} \right) \left( \frac{V}{P} \right) \\
            &= \frac{1}{P} \left( \frac{C_V}{C_P} \right) \\
            &= \frac{1}{P} \left( \frac{C_V}{C_V + nR} \right) \\
            &= P^{-1} \left( 1 + \frac{nR}{C_V} \right)^{-1}
        \end{align*}

    \item
        \begin{enumerate}
            \item
                $\Delta U = Q - W = 0$

            \item
                A reversible process with the same initial and final states for the system is given by isothermal expansion from $V_i$ to $V_f$. Because $\Delta U = 0$, we have
                \begin{align*}
                    \Delta S = \frac{Q_\text{rev}}{T} = \frac{W_\text{rev}}{T} = \frac{\int_{V_i}^{V_f} P dV}{T} = \int_{V_i}^{V_f} \frac{nRdV}{V} = nR \ln\left( \frac{V_f}{V_i} \right).
                \end{align*}

            \item
                Because $\Delta U = 0$, $\Delta T = 0$ and so $P_f = P_i \left( \frac{V_i}{V_f} \right)$.

            \item
                Because $\Delta T = 0$, $T_f = T_i = \frac{P_i V_i}{nR}$.
        \end{enumerate}

    \item
        \begin{enumerate}
            \item
                $-W_{1 \to 2} = \frac{nR(T_2 - T_1)}{\gamma - 1}$

            \item
                $Q_{2 \to 3} = \Delta U_{2 \to 3} + W_{2 \to 3} = C_V \Delta T_{2 \to 3} + 0 = C_V (T_3 - T_2)$

            \item
                $W_{3 \to 4} = \frac{nR(T_3 - T_4)}{\gamma - 1}$

            \item
                $-Q_{4 \to 1} = -(\Delta U_{4 \to 1} + W_{4 \to 1}) = -(C_V \Delta T_{4 \to 1} + 0) = C_V (T_4 - T_1)$

            \item
                Lowest to highest: $T_1, T_2, T_4, T_3$

            \item
                $\eta(r) = 1 - r^{1 - \gamma}$

            \item
                $\eta(10) \approx 0.602$

        \end{enumerate}

    \item
        \begin{enumerate}
            \item
                Taking partial derivatives of the modified form of the fundamental relation $dS = (1/T)dU + (P/T)dV$ yields
                \begin{align}
                    \left( \frac{\partial S}{\partial U} \right)_V = \frac{1}{T} \qquad \left( \frac{\partial S}{\partial V} \right)_U = \frac{P}{T}.
                \end{align}
                We also have the equations of state for cavity radiation,
                \begin{align}
                    P = \frac{U}{3V} \qquad U = aVT^4.
                \end{align}
                Following substitution we have
                \begin{align*}
                    \left( \frac{\partial S}{\partial U} \right)_V = \left( \frac{aV}{U} \right)^{1/4} \qquad \left( \frac{\partial S}{\partial V} \right)_U = \frac{a^{1/4} U^{3/4}}{3V^{3/4}}
                \end{align*}
                which both integrate to
                \begin{align}
                    S(U, V) = \frac{4a^{1/4}U^{3/4}V^{1/4}}{3}.
                \end{align}
                Finally, the inverse of this function is
                \begin{align}
                    U(S, V) = \frac{(3S)^{4/3}}{4^{4/3} (aV)^{1/3}}.
                \end{align}

            \item
                Notice that from (1) and (4),
                \begin{align}
                    T &= \left( \frac{\partial U}{\partial S} \right)_V = \left( \frac{3S}{4aV} \right)^{1/3} \\
                    aVT^4 &= \frac{(3S)^{4/3}}{4^{4/3} (aV)^{1/3}} = U
                \end{align}
                which yields one of the equations of state. For the other, we use (1) and (3) to obtain
                \begin{align*}
                    P = T \left( \frac{\partial S}{\partial V} \right)_U = \cdots = \frac{U}{3V}.
                \end{align*}

        \end{enumerate}

    \item
        From (5), we have
        \begin{align*}
            TV^{1/3} = \left( \frac{3S}{4a} \right)^{1/3}
        \end{align*}
        and from (2) and (4), we have
        \begin{align*}
            PV^{4/3} = \frac{UV^{1/3}}{3} = \frac{3^{1/3}S^{4/3}}{4^{4/3} a^{1/3}}.
        \end{align*}
        Because the process is adiabatic, $S$ is constant; thus the above two quantities are constant also.

\end{enumerate}

\end{document}
