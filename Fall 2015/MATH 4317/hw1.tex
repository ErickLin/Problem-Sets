\documentclass[a4paper,12pt]{article}

\usepackage{amsfonts, amsmath}

\begin{document}

    \section*{MATH 4317 - HW1 Solutions}

    \begin{enumerate}

            \setcounter{enumi}{1}
            \item
                \begin{enumerate}
                    \item Using the field property of additive inverses, we have
                    \begin{align*}
                        -(a - b) + (a - b) &= 0 \\
                        b - b &= 0 \\
                        -a + a &= 0 \\
                    \end{align*}
                    Adding the three equations together produces the result
                    \begin{align*}
                        -(a - b) - b + a &= 0 \\
                        -(a - b) &= b - a
                    \end{align*}

                    \item Using the field properties of distributivity and additive inverses and their consequences, we have
                    \begin{align*}
                        (a - b)(c - d) &= a(c - d) - b(c - d) \\
                        &= ac - ad - bc + bd \\
                        -(ad + bc) + (ad + bc) &= 0 \\
                    \end{align*}
                    Adding both equations together produces the result
                    \begin{align*}
                        (ac + bd) - (ad + bc) &= 0
                    \end{align*}
                \end{enumerate}

            \item Because $-a, -b, -(a - b) \in {R}_+$ and ${R}_+$ is closed under multiplication, $\frac{-(a - b)}{(-a)(-b)} \in {R}_+$ as well, and 
            \begin{align*}
                \frac{-(a - b)}{(-a)(-b)} &= \frac{-a}{(-a)(-b)} - \frac{-b}{(-a)(-b)} \\
                &= -\frac{1}{b} + \frac{1}{a} 
            \end{align*}
            Since $1/a - 1/b \in \mathbb{R}_+$, $1/a - 1/b > 0$ by the trichotomy consequence of the order property.


            \setcounter{enumi}{5}
            \item
                From the given constraints, $x - a, b - y \in \mathbb{R}_+$ which implies that $(b - y) + (x - a) = b - a + x - y = b - a - (y - x) \in \mathbb{R}_+$. \par
                Also, $b - a, y - x \in \mathbb{R}_+$ which implies that $(b - a) + (y - x) = b - a - (x - y) \in \mathbb{R}_+$. \par
                Since $b - a - (y - x)$ and $b - a - (x - y)$ are both contained in $\mathbb{R}_+$, $b - a - |y - x| \in \mathbb{R}_+$ as well, or $|y - x| < b - a$.


            \setcounter{enumi}{9}
            \item
                \begin{enumerate}
                    \item
                        g.l.b.: $0$ \space \space l.u.b.: $1$ \\
                        $0$ is a lower bound because $0 \leq \frac{1}{n} \; \forall n \geq 1$. Also, for any $a > 0$, $\frac{1}{n} < a \; \forall n > N$ if we let $N = \frac{1}{a}$. Therefore, $0$ is the greatest lower bound. \par
                      $  1$ is an upper bound because $\frac{1}{n} = 1$ for $n = 1$, and $\frac{1}{n_2} < \frac{1}{n_1} \; \forall n_2 > n_1 \geq 1$. Because $1$ is contained in the set, it is also the least upper bound.

                    \item
                        g.l.b.: $\frac{1}{3}$ \space \space l.u.b.: $\frac{1}{2}$ \\
                        $\frac{1}{3}$ is a lower bound because $\frac{3^n - 1}{2(3^n)} = \frac{1}{3}$ for $n = 1$, and $\frac{3^{n_2} - 1}{2(3^{n_2})} > \frac{3^{n_1} - 1}{2(3^{n_1})} \; \forall n_2 > n_1 \geq 1$. Because $\frac{1}{3}$ is contained in the set, it is also the greatest lower bound. \par
                        $\frac{1}{2}$ is an upper bound because $\frac{1}{2} \geq \frac{3^n - 1}{2(3^n)} \; \forall n \geq 1$. Also, for any $b < \frac{1}{2}$, $\frac{3^n - 1}{2(3^n)} > b \; \forall n > N$ if we let $N = \log_3(\frac{1}{1 - 2b})$. Therefore, $\frac{1}{2}$ is the least upper bound.

                    \item
                        Let $a_n$ denote the terms of the sequence for $n \geq 1$. \par
                        g.l.b.: $\sqrt{2}$ \space \space l.u.b.: $2$ \\
                        $\sqrt{2}$ is a lower bound because $a_1 = \sqrt{2}$, and $a_{n_2} > a_{n_1} \; \forall n_2 > n_1 \geq 1$. Because $\sqrt{2}$ is contained in the set, it is also the greatest lower bound. \par
                        $\lim_{n \to \infty} a_n$ can be found by making the substitution $b = \sqrt{2 + b} \Rightarrow b^2 - b - 2 = 0$ and solving for the positive root $b = 2$. Because $\{ a_n \}$ is both increasing and convergent, by the monotone convergence theorem the sequence must be bounded, and its limit must be its least upper bound.
                \end{enumerate}

            \setcounter{enumi}{11}
            \item
                Let $y \in Y$. Then $\forall \; x \in X$, $x < y$, which means that $y$ is an upper bound of $X$. Because $X$ is bounded above, by the Axiom of Completeness $X$ also has a least upper bound, which is fulfilled by $a$. By a similar argument, $Y$ has a greatest lower bound, which will be denoted by $b$. \par
                Now let $z \in \mathbb{R}$. If $z \notin Y$ then $z \in X$, and if $z \notin X$ then $z \in Y$, because otherwise $z$ would be an element of $\mathbb{R}$ not contained in $X \cup Y$, which is a contradiction. In particular, if $a \in X$, then $a \notin Y$ and is a lower bound of $Y$ because $a < y \; \forall \; y \in Y$. However, if $a \notin X$, then $a \in Y$. The only possibility for $a = y \in Y$ is if $y$ is the greatest lower bound of $Y$, so $a = b$. \par
                In conclusion, if $a \in X$, then $z \in \mathbb{R} : z \leq a$ cannot be contained in $Y$ because $z \leq a < y \; \forall \; y \in Y$, so $z \in X$ and $X = \{ x \in \mathbb{R} : x \leq a \}$. If $a \notin X$, then $z \in \mathbb{R} : z < a$ cannot be contained in $Y$ because $a$ is the greatest lower bound for $Y$, so $z \in X$ and $X = \{ x \in \mathbb{R} : x < a \}$.

        \end{enumerate}

\end{document}
