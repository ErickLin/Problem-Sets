\documentclass[a4paper,12pt]{article}

\usepackage{amsfonts, amsmath}

\begin{document}

\section*{MATH 3225 - HW1 Solutions}

\begin{enumerate}
	
	\item Let $\mathbb{P}(k)$ denote the probability that an even number of sixes will occur in $k$ throws. $\mathbb{P}(0) = 1$ because the number of sixes that appear after $0$ throws is $0$, and the formula gives $\frac{1}{2} [1 + \left( \frac{2}{3} \right)^0] = 1.$ \par
	Now assume that $\mathbb{P}(k) = \frac{1}{2} [1 + \left( \frac{2}{3} \right)^k]$. With an additional throw, the parity of the number of sixes will change if the result is a six; otherwise, the parity will stay the same. Another fact is that the probability of having an odd number of sixes is the complement of the probability of having an even number of sixes. Adding up the probabilities of both cases, we have
	\begin{align*}
        \mathbb{P}(k + 1) &= \frac{5}{6} \mathbb{P}(k) + \frac{1}{6} \mathbb{P}^c(k) \\
		&= \frac{5}{6} \left( \frac{1}{2} \left[ 1 + \left( \frac{2}{3} \right)^k \right] \right) + \frac{1}{6} \left( 1 - \frac{1}{2} \left[ 1 + \left( \frac{2}{3} \right)^k \right] \right) \\
		&= \frac{5}{12} + \frac{5}{12} \left( \frac{2}{3} \right)^k + \frac{1}{6} - \frac{1}{12} - \frac{1}{12} \left( \frac{2}{3} \right)^k \\
		&= \frac{1}{2} + \frac{1}{3} \left( \frac{2}{3} \right)^k \\
		&= \frac{1}{2} + \frac{1}{2} \left( \frac{2}{3} \right)^{k + 1} \\
		&= \frac{1}{2} \left[ 1 + \left( \frac{2}{3} \right)^{k + 1} \right]
	\end{align*}
	Therefore, the formula holds for all nonnegative integral $k$, and in particular, for $n$ as well.

	\setcounter{enumi}{8}
	\item Imagine dividing the $2n$ coins into two equally sized partitions and choosing a particular set of $k$ coins from each partition to be heads, leaving the remaining as tails. The probability of this event occurring is $\left( \frac{1}{2} \right)^{2n}$. Summing for all possible combinations of $k$ coins and for all values of $k$, we have
	\begin{align*}
	P &= \sum_{k = 0}^{n} \binom{n}{k} \binom{n}{k} \left( \frac{1}{2} \right)^{2n} \\
	&= \left( \frac{1}{2} \right)^{2n} \sum_{k = 0}^{n} \binom{n}{k} \binom{n}{n - k} \\
	&= \left( \frac{1}{2} \right)^{2n} \binom{2n}{n}
	\end{align*}
	The final step results from the combinatorial identity in which the number of ways to choose $n$ elements from a set of $2n$ elements is equivalent to partitioning the set into two sets of $n$ elements, and then choosing $k$ elements from the first partition and $n - k$ elements from the second partition for all possible values of $k$.

	\item In the first circuit, either the top two switches, the middle switch, or the bottom two switches must be closed. The probability that a flow occurs from $A$ to $B$ is the complement of the probability that all three of these events do not occur, or
		\[ 1 - (1 - p^2)(1 - p)(1 - p^2). \]
	One way to find the probability for the second circuit is to enumerate all the possible cases, which is done in the following way: \par
	If the middle switch is open (with probability $1 - p$), then either the top two switches or the bottom two switches must be closed. The probability that a flow occurs is the complement of the probability that neither of these events occur, or $1 - (1 - p^2)^2$. \par
	If the middle switch is closed (with probability $p$), then we will need to look at the top-left and bottom-left switches as well. The probability that both switches are closed is $p^2$, and the probability that the top-left switch is closed and the bottom-left switch is open is $p(1 - p)$, as with the probability that the bottom-left switch is closed and the top-left switch is open. Regardless of which of these three cases is true, only at least one of the two switches on the right must be closed, which occurs with probability $1 - (1 - p^2)^2$. Then the probability that a flow occurs is $[p^2 + p(1 - p) + p(1 - p)][1 - (1 - p^2)^2] = p^2 (2 - p)^2$. \par
	The probability for all cases combined is
	\begin{align*}
		P &= (1 - p) [1 - (1 - p^2)^2] + p[p^2(2 - p)^2] \\
		&= 1 - p - (1 - p)(1 - p^2)^2 + p^3(2 - p)^2.
	\end{align*}
	
	\setcounter{enumi}{15}
	\item The 09.00 bus will arrive at 08.55 with probability $\frac{1}{2}$, and in this case the probability that no one is waiting at 09.00 is $e^{-5/5} = e^{-1}$. Otherwise, the bus will arrive at 09.05, and then the 08.45 bus must be considered. The 08.45 bus will arrive at 08.40 with probability $\frac{1}{2}$, and the probability that no one is waiting at 09.00 is then $e^{-20/5} = e^{-4}$. Otherwise, this bus will arrive at 08.50, and the probability of the same event is then $e^{10/5} = e^{-2}$. The probability for all cases combined is
	\begin{align*}
		P &= \frac{1}{2} e^{-1} + \frac{1}{2} \left( \frac{1}{2} e^{-4} + \frac{1}{2} e^{-2} \right) \\
		&= \frac{1}{2} e^{-1} + \frac{1}{4} e^{-2} + \frac{1}{4} e^{-4}
	\end{align*}
	Let $A$ denote the event that no one is waiting at 09.00, and let $B$ denote the event that the 09.00 bus left at 08.55 instead of 09.05. Then
	\begin{align*}
		\mathbb{P}(B | A) &= \frac{\mathbb{P}(B \cap A)}{A} \\
		&= \frac{\frac{1}{2} e^{-1}}{\frac{1}{2} e^{-1} + \frac{1}{4} e^{-2} + \frac{1}{4} e^{-4}} \\
		&\approx 0.827,
	\end{align*}
	The ratio of the probability to its complement is
	\[ \frac{\mathbb{P}(B | A)}{1 - \mathbb{P}(B | A)} \approx 4.789. \]

	\item If the outcome of the first toss is a head, then $r - 1$ successive heads will result in in $E$; otherwise, at least one tail appears among the $r - 1$ tosses, which can be used as the new starting point for $\mathbb{P}(E | A = \mbox{tail})$. \par
	$\mathbb{P}(E | A = \mbox{tail})$ is the complement of the probability that the first run of $s$ successive tails occurs earlier than the first run of $r$ successive heads, and can be written
	\[
		\mathbb{P}(E | A = \mbox{tail}) = 1 - [p^{s - 1} + (1 - p^{s - 1})(1 - \mathbb{P}(E | A = \mbox{head}))].
	\]
	Then through substitution,
	\begin{align*}
		\mathbb{P}(E | A = \mbox{head}) &= p^{r - 1} + (1 - p^{r - 1}) (1 - [p^{s - 1} + (1 - \mathbb{P}(E | A = \mbox{head}) \\
		&\qquad - p^{s - 1} + p^{s - 1}\mathbb{P}(E | A = \mbox{head}))]) \\
		&= p^{r - 1} + (1 - p^{r - 1})(-2p^{s - 1} - \mathbb{P}(E | A = \mbox{head}) \\
		&\qquad + p^{s - 1} \mathbb{P}(E | A = \mbox{head})) \\
		&= p^{r - 1} - 2p^{s - 1} - \mathbb{P}(E | A = \mbox{head}) + p^{s - 1} \mathbb{P}(E | A = \mbox{head}) \\
		&\qquad + 2p^{r + s - 2} + p^{r - 1} \mathbb{P}(E | A = \mbox{head}) - p^{r + s - 2} \mathbb{P}(E | A = \mbox{head}) \\
		&= \frac{p^{r - 1} - 2p^{s - 1} + 2p^{r + s - 2}}{2 - p^{r - 1} - p^{s - 1} + p^{r + s - 2}},
	\end{align*}
	and marginalizing out $A$, we have
	\begin{align*}
        \mathbb{P}(E) &= \mathbb{P}(E | A = \mbox{head}) \mathbb{P}(A = \mbox{head}) + \mathbb{P}(E | A = \mbox{tail}) \mathbb{P}(A = \mbox{tail}) \\
        &= p \left( \frac{p^{r - 1} - 2p^{s - 1} + 2p^{r + s - 2}}{2 - p^{r - 1} - p^{s - 1} + p^{r + s - 2}} \right) \\
        &+ (1 - p) \left( 1 - [p^{s - 1} + (1 - p^{s - 1})(1 - \frac{p^{r - 1} - 2p^{s - 1} + 2p^{r + s - 2}}{2 - p^{r - 1} - p^{s - 1} + p^{r + s - 2}})] \right)
	\end{align*}

	\setcounter{enumi}{18}
	\item The probability can be written as the ratio of the number of ways that the desired event can occur to the total number of possibilities. Then the equation is (Note: The number of black socks is $n - 3$.)
	\begin{align*}
		\frac{\binom{3}{1} \binom{n - 3}{1}}{\binom{n}{2}} &= 3 \left( \frac{\binom{3}{2}}{\binom{n}{2}} \right) \\
		3(n - 3) &= 3(3) \\
		n &= 6.
	\end{align*}
	The probability that John wears matching black socks is
	\begin{align*}
		\frac{\binom{3}{2}}{\binom{n}{2}} = \frac{3(2)}{n(n - 1)} = \frac{1}{5}.
	\end{align*}

\end{enumerate}

\end{document}
