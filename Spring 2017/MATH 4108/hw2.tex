\documentclass[12pt]{article}

%Packages Used%
\usepackage{amsmath}
\usepackage{amssymb}
\usepackage{latexsym}
\usepackage{amsfonts,setspace}
\usepackage{fullpage}
\usepackage{comment}
\usepackage{enumitem}
\usepackage[margin=1in]{geometry}


\setlength{\abovedisplayskip}{3mm}
\setlength{\belowdisplayskip}{3mm}
\setlength{\abovedisplayshortskip}{0mm}
\setlength{\belowdisplayshortskip}{2mm}
\setlength{\baselineskip}{12pt}
\setlength{\normalbaselineskip}{12pt}

\newcommand{\blank}{\underline{~~~~~~~~~}}
\newcommand{\RR}{\mathbb{R}}
\newcommand{\QQ}{\mathbb{Q}}
\newcommand{\CC}{\mathbb{C}}
\newcommand{\ZZ}{\mathbb{Z}}
\newcommand{\PP}{\mathbb{P}}
\newcommand{\FF}{\mathbb{F}}
\newcommand{\kk}{\mathbf{k}}
\newcommand{\cP}{\mathcal{P}}

\newcommand{\bu}{\mathbf{u}}
\newcommand{\bv}{\mathbf{v}}
\newcommand{\bx}{\mathbf{x}}
\newcommand{\by}{\mathbf{y}}
\newcommand{\bb}{\mathbf{b}}
\newcommand{\bc}{\mathbf{c}}
\newcommand{\bp}{\mathbf{p}}
\newcommand{\bq}{\mathbf{q}}
\newcommand{\be}{\mathbf{e}}
\newcommand{\bzero}{\mathbf{0}}
\DeclareMathOperator{\Nul}{Nul}
\DeclareMathOperator{\Mod}{mod}
\DeclareMathOperator{\ord}{ord}
\DeclareMathOperator{\GL}{GL}
\DeclareMathOperator{\SL}{SL}
\DeclareMathOperator{\Hom}{Hom}
\DeclareMathOperator{\End}{End}
\DeclareMathOperator{\Ind}{Ind}
\DeclareMathOperator{\im}{im}
\DeclareMathOperator{\tr}{tr}

\renewcommand{\And}{\wedge}
\newcommand{\Or}{\vee}
\newcommand{\Implies}{\Rightarrow}
\newcommand{\Not}{\sim}



\normalbaselines
\pagestyle{empty}
\raggedbottom
\begin{document}

\noindent
\textbf{Name: Erick Lin} \smallskip  \\
\textbf{Collaborators:} \smallskip \\ %%% List anyone with whom you discussed the problems
\textbf{Outside resources: Wellesley MATH 306 (for Exercise 9)} \smallskip \\ %%% List all resources used OTHER than the textbook, lecture notes, quizzes, worksheets, and previous homework assignments.

\begin{center}
{
Math 4108, Algebra II \\
HW 2 --- Due Feb 3, 2017 (Fri)
}
\end{center}

\noindent Read and think about the problems in Chapters 3 and 4 of the textbook (Howie).   Turn in the following problems.  Justify all your answers.

\begin{enumerate}
    \item Let $K = \QQ(\sqrt[3]{2})$. Find all elements $\alpha \in K$ such that $K = \QQ(\alpha)$. \par
        The roots of the minimum polynomial of $\sqrt[3]{2}$, $x^3 - 2$, also include $\sqrt[3]{2}e^{2i\pi/3}$ and $\sqrt[3]{2}e^{4i\pi/3}$, which means that $\QQ(\sqrt[3]{2}) \cong \QQ[x]/\langle x^3 - 2 \rangle \cong \QQ(\sqrt[3]{2}e^{2i\pi/3}) \cong \QQ(\sqrt[3]{2}e^{4i\pi/3})$. However, $\sqrt[3]{2}e^{2i\pi/3}, \sqrt[3]{2}e^{4i\pi/3} \notin K$.

    \item Let $K = \QQ(\sqrt{3},\sqrt{5})$.
    \begin{enumerate}
        \item Prove that $\sqrt{3} \notin \QQ(\sqrt{5})$. \par
            Suppose for the purpose of contradiction that there exist $a, b \in \mathbb{Q}$ such that $\sqrt{3} = a + b\sqrt{5}$, where $b \neq 0$ since $\sqrt{3}$ is irrational. Then $a^2 = (\sqrt{3} - b\sqrt{5})^2 = 3 - 2b\sqrt{15} + 5b^2$, which implies that $\sqrt{15} = (a^2 + 5b^2 + 3)/2b \in \mathbb{Q}$, a contradiction since $\sqrt{15}$ is also irrational.
        %https://math.stackexchange.com/questions/1601819/show-that-the-only-subfields-of-mathbbqi-sqrt5-is-mathbbq-mathb
        \item Find a basis of $K$ over $\QQ$. \par
            Since $\{ 1, \sqrt{3} \}$ is a basis for $\QQ(\sqrt{3})$ over $\mathbb{Q}$ and $\{ 1, \sqrt{5} \}$ is a basis for $\QQ(\sqrt{3}, \sqrt{5})$ over $\mathbb{Q}(\sqrt{3})$, the proof of Theorem 3.3 allows us to deduce that $\{ 1, \sqrt{3}, \sqrt{5}, \sqrt{15} \}$ is a basis for $K = \QQ(\sqrt{3},\sqrt{5})$.
        \item Show that the only subfields of $K$ are $\QQ$, $\QQ(\sqrt{3})$, $\QQ(\sqrt{5})$, $\QQ(\sqrt{15})$, and $K$ itself. \par
            Because $[K : \QQ] = 4$, all the subfields of $K$ other than itself and $\QQ$ must be of degree 2, the only nontrivial factor of $4$. Lastly, the existence of any subfield other than $\QQ(\sqrt{3})$, $\QQ(\sqrt{5})$, or $\QQ(\sqrt{15})$ contradicts the independence of the basis given in part (b).
        \item Find the minimum polynomial of $\sqrt{3}+\sqrt{5}$ over $\QQ$. \par
            %The roots of the polynomial must include $\pm \sqrt{3} \pm \sqrt{5}$.
            The minimum polynomial must be of degree $4$, the cardinality of the basis. We can thus see that
            \begin{align*}
                (x - \sqrt{3} - \sqrt{5})(x - \sqrt{3} + \sqrt{5})(x + \sqrt{3} - \sqrt{5})(x + \sqrt{3} + \sqrt{5}) = x^4 - 11x^2 + 4.
            \end{align*}
            is the minimum polynomial.
    \end{enumerate}

    \item Find the multiplicative inverse of $1+\sqrt{3}+\sqrt{5}+\sqrt{15}$ in $\QQ(\sqrt{3},\sqrt{5})$.
        \begin{align*}
            \frac{1}{1 + \sqrt{3} + \sqrt{5} + \sqrt{15}} \cdot \frac{1 - \sqrt{3}}{1 - \sqrt{3}} = \frac{1 - \sqrt{3}}{-2(1 + \sqrt{5})} \cdot \frac{1 - \sqrt{5}}{1 - \sqrt{5}} = \frac{1 - \sqrt{3} - \sqrt{5} + \sqrt{15}}{8}
        \end{align*}

    \item Show that $[\QQ(\sqrt{5}+\sqrt[3]{2}) : \QQ] = 6$. \par
        It is clear that $\sqrt{5} + \sqrt[3]{2} \in \QQ(\sqrt{5}, \sqrt[3]{2})$, and so $\QQ(\sqrt{5} + \sqrt[3]{2}) \subseteq \QQ(\sqrt{5}, \sqrt[3]{2})$. The minimum polynomial of $\sqrt{5}$ over $\mathbb{Q}$ is $x^2 - 5$, while the minimum polynomial of $\sqrt[3]{2}$ over $\mathbb{Q}$ is $x^3 - 2$, so Exercise 5 tells us that $[\QQ(\sqrt{5}, \sqrt[3]{2}) : \QQ] = 6$. Also, using an argument similar to the one in 2(a) yields $\sqrt{5} \notin \mathbb{Q}(\sqrt[3]{2})$ and $\sqrt[3]{2} \notin \QQ(\sqrt{5})$, which means that $\QQ(\sqrt{5} + \sqrt[3]{2})$ must be larger than these two fields. Examining possible degrees of the extensions shows that $\QQ(\sqrt{5} + \sqrt[3]{2}) = \QQ(\sqrt{5}, \sqrt[3]{2})$.

    \item Let $L$ be a field, $K$ be a subfield of $L$, and $a,b \in L$ be algebraic over $K$ of degrees $m$ and $n$ respectively.  Prove that if $m$ and $n$ are relatively prime, then $[K(a,b):K] = mn$. \par
        We know that $[K(a, b) : K] = [K(a, b) : K(a)][K(a) : K] = [K(a, b) : K(b)][K(b) : K]$, so it must be a multiple of both $m$ and $n$; the LCM of $m$ and $n$ is $mn$. Also, $[K(a, b) : K(a)] \leq [K(b) : K] = n$ and $[K(a, b) : K(b)] \leq [K(a) : K] = m$; thus, $mn$ is an upper bound on $[K(a, b) : K]$.

    \item Let $K$ be a field. Prove that the following conditions are equivalent. 
    \begin{enumerate}
        \item Every polynomial in $K[x]$ of degree $\geq 1$ has a root in $K$.
        \item Every polynomial in $K[x]$ of degree $\geq 1$ {\em splits} over $K$, that is, it factors as a product of linear polynomials.
        \item Every irreducible polynomial in $K[x]$ has degree $1$.
        \item There is no algebraic extension of $K$ except $K$ itself.
    \end{enumerate}
    A field $K$ is called {\em algebraically closed} if any (hence all) of the conditions above are satisfied. \par
    (a) implies (b) by induction, since a polynomial of degree $\geq 2$ with a root factors into a linear polynomial and a polynomial of degree $\geq 1$. (b) implies (c) since any polynomial of degree $\geq 2$ that splits must be reducible. (c) implies (d) since the minimum polynomial of any element must be of degree $1$, while any algebraic extension of $K$ other than itself involves a minimum polynomial of degree $\geq 2$. Lastly, (d) implies (a) because any polynomial of degree $\geq 2$ with no roots suggests the existence of an irreducible degree-2 polynomial, which in turn can be used to form an algebraic extension of $K$ of degree 2.

    \item Prove that an algebraically closed field must contain infinitely many elements. \par
        Suppose for the purpose of contradiction that an algebraically closed field has finitely many elements, which we may enumerate as $a_1, \cdots, a_n$. Then the polynomial $(x - a_1) \cdots (x - a_n) + 1$ of degree $n$ has no roots (since it evaluates to $1$ at every $a_i$), contradicting condition (a).

    \item Let $\overline{\QQ}$ denote the field of algebraic numbers:
    $$
        \overline{\QQ} = \{a \in \CC : a \textup{ is algebraic over }\QQ \}.
    $$
    Show that $\overline{\QQ}$ is algebraically closed.  You can use the Fundamental Theorem of Algebra, which says that $\CC$ is algebraically closed. \par
    The coefficients of any polynomial $a_n x^n + \cdots + a_0 \in \overline{\QQ}[x]$ are algebraic over $\mathbb{Q}$, so $\mathbb{Q}(a_0, \cdots, a_n)$ is an algebraic extension of $\QQ$. Thus, for any root $\alpha \in \mathbb{C}$ of the polynomial, $\mathbb{Q}(a_0, \cdots, a_n, \alpha)$ is an algebraic extension of $\QQ$, meaning $\alpha \in \overline{\QQ}$. \par
    {\em Note}:  An extension $L$ of a field $K$ is called an {\em algebraic closure} of $K$ if $L$ is algebraically closed and is algebraic over $F$.  This exercise shows that the algebraic numbers form an algebraic closure of the rational numbers.

    \item 
    \begin{enumerate}
        \item Find the minimum polynomial of $\cos\left( \frac{2 \pi}{5} \right)$. \par
            Expanding $(\cos\left(\frac{2\pi}{5}\right) + i\sin\left(\frac{2\pi}{5}\right))^5 = 1$ and equating the real parts, we have that $(x - 1)(4x^2 + 2x - 1)$ evaluated at $\cos\left(\frac{2\pi}{5}\right)$ is $0$, so the minimum polynomial is $4x^2 + 2x - 1$.
        \item Prove that a regular pentagon is constructible. \par
            Since $\cos\left(\frac{2\pi}{5}\right)$ is a root of $4x^2 + 2x - 1$, it must be $\frac{\sqrt{5} - 1}{4}$, which is constructible. Then the angle of $\frac{2\pi}{5}$ is also constructible, and so a regular pentagon is constructible.
    \end{enumerate}
\end{enumerate}

\end{document}
