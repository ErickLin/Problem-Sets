\documentclass[a4paper,12pt]{article}

\usepackage{amsfonts, amsmath, amssymb, amsthm, enumitem, fancyhdr, graphicx}
\usepackage[margin=1in, includehead, includefoot, heightrounded]{geometry}
\allowdisplaybreaks
\pagestyle{fancy}
\rhead{Erick Lin}

\newcommand*\dist{\mathop{\!\mathrm{d}}}
\newcommand{\norm}[1]{\left\lVert#1\right\rVert}
\renewcommand{\thesubsection}{\arabic{subsection}}
\DeclareMathOperator*{\argmax}{\arg\!\max}
\DeclareMathOperator*{\argmin}{\arg\!\min}
\DeclareMathOperator*{\vol}{vol}
\DeclareMathOperator*{\esssup}{ess\,sup}
\DeclareMathOperator*{\BV}{BV}
\DeclareMathOperator*{\Lip}{Lip}
\DeclareMathOperator*{\AC}{AC}
\newtheorem{theorem}{Theorem}
\newtheorem{lemma}[theorem]{Lemma}

\begin{document}

\section*{MATH 6337 -- HW9 Solutions}
\begin{enumerate}
    \item[5.1.7.]
        \boldmath\textbf{Given $f : [a, b] \to \mathbb{C}$, write $f = f_r + if_i$ where $f_r$ and $f_i$ are real-valued. Prove that
        \begin{align*}
            f \in \AC[a, b] \quad\Longleftrightarrow\quad f_r, f_i \in \AC[a, b].
        \end{align*}
        }\unboldmath \par
        ($\Rightarrow$) For every $\varepsilon > 0$, there exists a $\delta > 0$ such that for any finite or countably infinite collection of nonoverlapping subintervals $\{[a_j, b_j]\}$ of $[a, b]$ satisfying $\sum_j (b_j - a_j) < \delta$, we have
        \begin{align*}
             \sum_j |f(b_j) - f(a_j)| = \sum_j |f_r(b_j) - f_r(a_j) + i(f_i(b_j) - f_i(a_j))| < \varepsilon.
        \end{align*}
        Since $|f_r(b_j) - f_r(a_j)|, |f_i(b_j) - f_i(a_j)| \leq |f_r(b_j) - f_r(a_j) + i(f_i(b_j) - f_i(a_j))|$ for all $j$ by the Pythagorean theorem, $\sum_j |f_r(b_j) - f_r(a_j)|, \sum_j|f_i(b_j) - f_i(a_j)| < \varepsilon$ as well, meaning that $f_r, f_i \in \AC[a, b]$. \par
        ($\Leftarrow$) For every $\varepsilon > 0$, there exists $\delta_1, \delta_2 > 0$ such that for any finite or countably infinite collection of nonoverlapping subintervals $\{[a_j, b_j]\}$ of $[a, b]$,
        \begin{gather*}
            \sum_j (b_j - a_j) < \delta_1 \quad\Longrightarrow\quad \sum_j |f_r(b_j) - f_r(a_j)| < \frac{\varepsilon}{2} \\
            \sum_j (b_j - a_j) < \delta_2 \quad\Longrightarrow\quad \sum_j |f_i(b_j) - f_i(a_j)| < \frac{\varepsilon}{2}.
        \end{gather*}
        Since $|f(b_j) - f(a_j)| = |f_r(b_j) - f_r(a_j) + i(f_i(b_j) - f_i(a_j))| \leq |f_r(b_j) - f_r(a_j)| + |f_i(b_j) - f_i(a_j)|$ for all $j$ by the triangle inequality, if we take $\delta = \min\{ \delta_1, \delta_2 \}$, then
        \begin{gather*}
            \sum_j (b_j - a_j) < \delta \quad\Longrightarrow\quad \sum_j |f(b_j) - f(a_j)| < \frac{\varepsilon}{2} + \frac{\varepsilon}{2} = \varepsilon.
        \end{gather*}
        Hence, $f \in \AC[a, b]$.

    \item[5.1.10.]
        \boldmath\textbf{Prove that $f \in \AC[a, b]$ if and only if for every $\varepsilon > 0$ there exists a $\delta > 0$ such that for any \emph{finite} collection of nonoverlapping subintervals $\{[ a_j, b_j ]\}_{j = 1, \cdots, N}$ of $[a, b]$, we have
        \begin{align*}
            \sum_{j = 1}^N (b_j - a_j) < \delta \quad\Longrightarrow\quad \sum_{j = 1}^N |f(b_j) - f(a_j)| < \varepsilon.
        \end{align*}
        }\unboldmath \par
        The forward implication follows from the definition of $\AC[a, b]$. \par
        For the reverse implication, we must show that the countably infinite case reduces to the finite case, so consider a countably infinite collection of nonoverlapping subintervals $\{[ a_j, b_j ]\}_{j \in \mathbb{N}}$ of $[a, b]$. Since
        \begin{align*}
            \sum_{j = 1}^N (b_j - a_j) < \delta \quad\Longrightarrow\quad \sum_{j = 1}^N |f(b_j) - f(a_j)| < \varepsilon
        \end{align*}
        is true for all $N$ and the sequence obtained on the right is increasing and bounded by $\varepsilon$, the limit as $N \to \infty$ is
        \begin{align*}
            \sum_{j = 1}^\infty (b_j - a_j) < \delta \quad\Longrightarrow\quad \sum_{j = 1}^\infty |f(b_j) - f(a_j)| < \varepsilon.
        \end{align*}
        by convergence of monotone sequences. In other words, $f \in \AC[a, b]$.

    \item[5.3.5.]
        \boldmath\textbf{Assume that $g : [a, b] \to [c, d]$ and $f : [c, d] \to \mathbb{C}$ are continuous. Prove that
        }\unboldmath
        \begin{enumerate}
            \item
                \boldmath\textbf{if $f$ is Lipschitz and $g \in \AC[a, b]$, then $f \circ g \in \AC[a, b]$.
                }\unboldmath \par
                Let $K$ be a Lipschitz constant of $f$. Then for every $\varepsilon > 0$, there exists a $\delta > 0$ such that for any finite or countably infinite collection of nonoverlapping subintervals $\{[a_j, b_j]\}$ of $[a, b]$ satisfying $\sum_j (b_j - a_j) < \delta$,
                \begin{align*}
                    \sum_j |g(b_j) - g(a_j)| < \frac{\varepsilon}{K}.
                \end{align*}
                Since $|(f \circ g)(b_j) - (f \circ g)(a_j)| = |f(g(b_j)) - f(g(a_j))| \leq K|g(b_j) - g(a_j)|$ for all $j$, we have that
                \begin{align*}
                    \sum_j |(f \circ g)(b_j) - (f \circ g)(a_j)| \leq K\sum_j |g(b_j) - g(a_j)| < \varepsilon.
                \end{align*}
                Because $\varepsilon$ and $\{[a_j, b_j]\}$ are arbitrary, we conclude that $f \circ g \in \AC[a, b]$.
        \end{enumerate}

    \item[5.3.8.]
        \boldmath\textbf{Assume $f \in \AC[a, b]$ and there is a continuous function $g$ such that $f' = g$ a.e. Show that $f$ is differentiable everywhere on $[a, b]$ and $f'(x) = g(x)$ for every $x$. Show by example that the hypothesis of absolute continuity is necessary.
        }\unboldmath \par
        %Define $E = \{ f' = g \}$ and $Z = [a, b] \setminus E$. Since $|Z| = 0$, the Banach-Zaretsky Theorem states that $|f(Z)| = 0$. \par
        \iffalse
            From the Fundamental Theorem of Calculus,
            \begin{align*}
                f(x) - f(a) = \int_a^x f'(t) dt = \int_a^x g(t) dt, \quad x \in [a, b]
            \end{align*}
            since $f'(t) = g(t)$ a.e, and the integral of a continuous function is continuously differentiable. \par
        \fi
        Since $g$ is continuous, $g \in L^1[a, b]$, so by Lemma 5.1.6,
        \begin{align*}
            G(x) = \int_a^x g(t) dt, \quad x \in [a, b]
        \end{align*}
        is absolutely continuous on $[a, b]$. Then $f - g \in \AC[a, b]$ because for every $\varepsilon > 0$, there exists $\delta_1, \delta_2 > 0$ such that for any finite or countably infinite collection of nonoverlapping subintervals $\{[a_j, b_j]\}$ of $[a, b]$,
        \begin{gather*}
            \sum_j (b_j - a_j) < \delta_1 \quad\Longrightarrow\quad \sum_j |f(b_j) - f(a_j)| < \frac{\varepsilon}{2} \\
            \sum_j (b_j - a_j) < \delta_2 \quad\Longrightarrow\quad \sum_j |g(b_j) - g(a_j)| < \frac{\varepsilon}{2},
        \end{gather*}
        and since $|(f - g)(b_j) - (f - g)(a_j)| \leq |f(b_j) - f(a_j)| + |-1||g(b_j) - g(a_j)|$ for all $j$ by the triangle inequality, taking $\delta = \min\{ \delta_1, \delta_2 \}$ yields
        \begin{gather*}
            \sum_j (b_j - a_j) < \delta \quad\Longrightarrow\quad \sum_j |(f - g)(b_j) - (f - g)(a_j)| < \frac{\varepsilon}{2} + \frac{\varepsilon}{2} < \varepsilon.
        \end{gather*}
        Because $(f - G)' = f' - g = 0$ a.e. by the hypothesis, Corollary 5.3.4 states that $f - G = c$ for some constant $c$. Taking the derivatives of both sides again gives $f' = g$ everywhere. \par
        The Cantor-Lebesgue function $\varphi : [0, 1] \to [0, 1]$ is H\"older continuous with exponent $\alpha = \log_3 2$ (and hence uniformly continuous) on $[0, 1]$, and $\varphi' = 0$ a.e. However, $\varphi$ is not differentiable everywhere because otherwise, $\varphi' \neq 0$ on a set of measure zero, so $\varphi = \int_0^1 \varphi' = \int_0^1 0 = 0$ everywhere, a contradiction. (In particular, $\varphi$ is not differentiable on the Cantor set.)

    \item[5.3.11.]
        \boldmath\textbf{Define $g(x) = x^2\sin(1/x^2)$ for $x \neq 0$, and set $g(0) = 0$. Show that $g \in L^1[-1, 1]$, $g$ is differentiable everywhere on $[-1, 1]$, $g' \notin L^1[-1, 1]$, and $g \notin \AC[-1, 1]$.
        }\unboldmath \par
        %Exercise 4.2.3 (pg. 183) \par
        $g \in L^1[-1, 1]$ because
        \begin{align*}
            \int_{-1}^1 |g(x)| dx = 2 \int_0^1 |x^2\sin(1/x^2)| dx \leq 2 \int_0^1 x^2 dx = 2 \frac{x^3}{3} \bigg|_0^1 = \frac{2}{3} < \infty.
        \end{align*}
        $g$, being the product of the functions $x^2$ and $\sin(1/x^2)$ (which are differentiable when $x \neq 0$), is differentiable when $x \neq 0$; also, $g$ is differentiable at $x = 0$ because $-x \leq x\sin(1/x^2) \leq x$ everywhere so by the Sandwich Theorem,
        \begin{align*}
            \lim_{h \to 0} \frac{g(0 + h) - g(0)}{h} = \lim_{h \to 0} h\sin(1/h^2) = 0.
        \end{align*}
        Differentiating gives $g'(0) = 0$ and $g'(x) = x\sin(1/x^2) + x^2\cos(1/x^2)(-2/x^3) = x\sin(1/x^2) - 2\cos(1/x^2)/x$ for all $x \neq 0$. \par
        $g' \notin L^1[-1, 1]$ because using the change of variables $y = 1/x^2$,
        \iffalse
            From the same reasoning as in Exercise 4.2.21(a) (on the previous homework), the Mean Value Theorem for (Riemann) Integrals states that the above quantity diverges if and only if
        \fi
        \begin{align*}
            \int_{-1}^1 |g'(x)| dx &= 2\int_0^1 |x\sin(1/x^2) - 2\cos(1/x^2)/x| dx \\
            &= 2\int_1^\infty |\sin(y)/\sqrt{y} - 2\sqrt{y}\cos(y)| (x^3/2)dy = \infty.
        \end{align*}
        $g \notin \BV[-1, 1]$ by Exercise 4.2.21(a) with $a = 2$ and $b = 2$, and hence $g \notin \AC[-1, 1]$ by Lemma 5.1.3(b).
\end{enumerate}

\end{document}
