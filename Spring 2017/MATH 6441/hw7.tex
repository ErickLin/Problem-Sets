\documentclass[a4paper,12pt]{article}

\usepackage{amsfonts, amsmath, amssymb, amsthm, dsfont, enumitem, fancyhdr, graphicx}
\usepackage[margin=0.9in, includehead, includefoot, heightrounded]{geometry}
\allowdisplaybreaks
\pagestyle{fancy}
\rhead{Erick Lin}

\newcommand{\norm}[1]{\left\lVert#1\right\rVert}
\newcommand*\dist{\mathop{\!\mathrm{d}}}
\DeclareMathOperator{\im}{Im}
%\let\ker\relax %RedeclareMathOperator
%\DeclareMathOperator{\ker}{Ker}
\DeclareMathOperator{\Hom}{Hom}
\renewcommand{\thesubsection}{\arabic{subsection}}
\newcommand*\sq{\mathbin{\vcenter{\hbox{\rule{.3ex}{.3ex}}}}}
\newcommand{\iso}{\approx}
\newtheorem{theorem}{Theorem}
\newtheorem{lemma}[theorem]{Lemma}

\begin{document}

\section*{MATH 6441 -- HW7 Solutions}
\subsection*{Section 2.2}
\begin{enumerate}
    \item[27.]
        \boldmath\textbf{The short exact sequences $0 \to C_n(A) \to C_n(X) \to C_n(X, A) \to 0$ always split, but why does this not always yield splittings $H_n(X) \approx H_n(A) \oplus H_n(X, A)$?
        }\unboldmath \par
        While we have that $C_n(X) \approx C_n(A) \oplus C_n(X, A)$, the splitting is not necessarily preserved under the boundary maps $\partial$, and each homology group is of the form $\ker\partial / \im\partial$. \par
        In general, if there is no retraction $r : X \to A$, then the boundary maps in the long exact sequence for $(X, A)$ are not necessarily zero, so the long exact sequence cannot be split into short exact sequences on which the splitting lemma can be applied. \par
        One counterexample is $X = D^n$, $A = S^{n - 1}$, which would yield an impossible splitting $H_{n - 1}(D^n) \approx H_{n - 1}(S^{n - 1}) \oplus H_{n - 1}(D^n, S^{n - 1})$ (the left-hand side is the trivial group, while the first factor of the right-hand side is $\mathbb{Z}$).
\end{enumerate}

\subsection*{Section 3.1}
\begin{enumerate}
    \item[4.]
        \sloppy
        \boldmath\textbf{If one defines homology groups $h_n(X; G)$ as the homology groups of the chain complex
            \begin{align*}
                \cdots \to \Hom\left( G, C_n(X) \right) \to \Hom \left( G, C_{n - 1}(X) \right) \to \cdots,
            \end{align*}
            what are the groups $h_n(X; G)$ when $G = \mathbb{Z}$, $\mathbb{Z}_m$, and $\mathbb{Q}$?
        }\unboldmath \par
        \begin{lemma}
            $\Hom(\mathbb{Z}, A) \approx A$ for any abelian group $A$.
        \end{lemma}
        \begin{proof}
            $\mathbb{Z}$ is the free abelian group generated by the element $1$. Since homomorphisms are specified by where they send generators, each homomorphism can be identified to the element of $A$ to which $1$ is sent.
        \end{proof}
        Since $\Hom(\mathbb{Z}, C_n(X)) \approx C_n(X)$ for each $n$ by the lemma, the chain complex when $G = \mathbb{Z}$ is the same as the original chain complex,
        \begin{align*}
            \cdots \xrightarrow{\partial} C_n(X) \xrightarrow{\partial} C_{n - 1}(X) \xrightarrow{\partial} \cdots,
        \end{align*}
        and so the groups $h_n(X; \mathbb{Z})$ are the same as the singular homology groups $H_n(X) \approx \ker\partial_n / \im\partial_{n + 1}$. \par
        $\mathbb{Z}_m$ is the free abelian group on the generator $1$ of order $m$. Any member of $\Hom(\mathbb{Z}_m, C_n(X))$ must send $1$ to an element of $C_n(X)$ of order dividing $m$ --- this element must be the identity, because any nontrivial element in $C_n(X)$ is of infinite order. Thus, $\Hom(\mathbb{Z}_m, C_n(X)) = 0$ for each $n$, implying that $h_n(X; \mathbb{Z}_m) = 0$. \par
        Lastly, every element in $\mathbb{Q}$ is divisible by each positive integer, so any member of $\Hom(\mathbb{Q}, C_n(X))$ must send $1$ to a divisible element of $C_n(X)$ --- this element must be the identity, because any nontrivial element when viewed as a linear combination $\sum_i c_i a_i$ of the generators $a_i$ is indivisible by (for instance) $c_i + 1$. Thus, $\Hom(\mathbb{Q}, C_n(X)) = 0$ for each $n$, implying that $h_n(X; \mathbb{Q}) = 0$.
\end{enumerate}
\end{document}
